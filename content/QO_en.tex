\chapter{Alkanes}
\section{STRUCTURE AND NOMENCLATURE}
There is a large number of hydrocarbons having the formula \ce{C_{n}H_{2n+2}}. These compounds are called alkanes or paraffins, and methane, \ce{CH4} (Section 2.5), is the simplest of them. By increasing \emph{n} we obtain the formulas of a family of compounds, a homologous series. The first four terms of the series are:

\begin{figure}[h!]
\centering
\setatomsep{1.8em}
\chemname{\chemfig{H-C(-[2]H)(-[6]H)-H}}{Metano}
\chemname{\chemfig{H-C(-[2]H)(-[6]H)-C(-[2]H)(-[6]H)-H}}{Etano}
\chemname{\chemfig{H-C(-[2]H)(-[6]H)-C(-[2]H)(-[6]H)-C(-[2]H)(-[6]H)-H}}{Propano}
\chemname{\chemfig{H-C(-[2]H)(-[6]H)-C(-[2]H)(-[6]H)-C(-[2]H)(-[6]H)-C(-[2]H)(-[6]H)-H}}{Butano}
\end{figure}

The above compounds are all gaseous at room temperature, and the last two, liquefied and placed in cylinders, are widely used as fuel. The higher molecular weight counterparts - pentane, hexane, octane, nonane, etc. - are liquid (Table 3.1).

\begin{table}[h!]
\centering
\begin{tabular}{cccccc}
\hline 
N. de carbonos & Fórmula & Nome & Número total de isômeros possíveis & P.E. ($\degree$C) & P.F. ($\degree$C) \\
\hline
1 & \ce{CH4} & Metano & 1 & -162 & -183 \\ 
2 & \ce{C2H6} & Etano & 1 & -89 & -172 \\
3 & \ce{C3H8} & Propano & 1 & -42 & -187 \\
4 & \ce{C4H10} & Butano & 2 & 0 & -138 \\
5 & \ce{C5H12} & Pentano & 3 & 36 & -130 \\
6 & \ce{C6H14} & Hexano & 5 & 69 & -95 \\ 
7 & \ce{C7H16} & Heptano & 9 & 98 & -91 \\
8 & \ce{C8H18} & Octano & 18 & 126 & -57 \\
9 & \ce{C9H20} & Nonano & 35 & 151 & -54 \\
10 & \ce{C10H22} & Decano & 75 & 174 & -30 \\ 
11 & \ce{C11H24} & Undecano & - & 196 & -26 \\ 
12 & \ce{C12H26} & Dodecano & - & 216 & -10 \\
20 & \ce{C20H42} & Eicosano & 366.319 & 34 & +36 \\
30 & \ce{C30H62} & Tricontano & $4,11 \times 10^9$ & 446 & +66 \\
\hline
\end{tabular}
\caption{Os hidrocarbonetos normais.}
\label{Quadro:3.1}
\end{table}

\emph{TRIVIAL NOMENCLATURE. The nomenclature of simple organic molecules is not completely systematic because the compounds were already known and had their own names before their structures were known. For example, the name butane came from a related compound (butyric acid) that was first isolated from rancid butter. The higher homologues have a more systematic nomenclature, based on Greek numbers. These compounds and some others are sometimes called aliphatic hydrocarbons (from the Greek, aleifar, fat).}

Linear homologs containing 18 or more carbons are low melting point, and their blend, commercially known as paraffin wax, has long been used to seal jelly bottles, but is now more widely used as a moderator in nuclear reactors.

The bonds in all alkanes are fundamentally the same as in methane. The atoms are bound by electron pairs in \emph{sp$^3$} hybrid orbitals of the carbon atoms and orbital 1s of the hydrogen atoms. Each carbon has its substituents (atoms or groups of atoms attached to it) arranged in a tetrahedral arrangement.

It is often difficult to visualize the three-dimensional structure of a molecule. It then becomes very useful to have 'molecular models' that help to understand the structures. There are three main types of models, shown in Figure 3.1, which are an approximation of the structure and try to give an idea of ​​the physical size of the various atoms. Although none of the model types are perfectly accurate, they have some utility, depending on the molecular property we are examining. The reader should become familiar with each type of model. We will use the different types in this text, depending on the needs of each case.


\chapter{Nuclear Magnetic Resonance Spectroscopy}
\section{Identification of organic compounds}
When a student starts in organic laboratory chemistry his vision is very different from what he has when he studies the subject in a textbook like this. The physical aspect of organic compounds bears little relation to the lines, symbols and theories with which they are described in a book. When a chemist needs to identify a compound, its task can be very difficult and time consuming. It is of paramount importance that he knows what tools are available to assist him in identifying, as well as when and how to apply them, and also how to interpret the information they provide.

If two samples are equal in all physical and chemical properties, they are the same compound. Therefore, the first step in identifying a compound of unknown structure is to obtain as much information as possible about the compound. Its physical state is examined and data such as boiling and melting points, solubility characteristics, presence or absence of acidic or basic properties and index of refraction are obtained.

Any of the various spectral techniques can be used to obtain structural information about an unidentified compound.

The infrared spectrum can be interpreted in terms of the presence or absence of functional groups. From the nuclear magnetic resonance spectrum, the number nature and 'environment' of the hydrogens in a molecule can be determined. Of these two spectra the molecular skeleton structure can often be deduced. A mass spectrum provides data on molecular weight and formula and on the arrangement of specific groups in the molecule. Ultraviolet spectra resulting from electronic excitations are obtained from compounds containing unsaturated bonds. These tools provide different types of data that are very effectively used together and with additional physical and chemical data.

The organic chemist has available a wide range of spectroscopic, physical and chemical techniques that can be used to identify and characterize compounds and to determine structures. When he has obtained and studied the physical, chemical, and spectroscopic data, he will know much about the unknown compound and the class of functional group to which he belongs and may suggest a structure for the compound.

The second step in the identification of an unknown compound is the literature search. The properties of the unknown compound can be compared with existing compilations of published data for all compounds already studied. In addition, derivatives may be prepared by well-characterized reactions and their physical and chemical properties can be compared to the corresponding published values. The spectral data can be compared with reference spectra obtained from known compounds: if two samples are the same compound, their spectra will be superimposable. If programs and computer access are available, much of the published data research can be done quickly.

If at the end of the search in the chemical literature a compound can not be identified, the chemist must then assume that it is a new compound and is faced with the task of characterizing it, elucidating its structure and publishing results. From the available data you now have, the chemist can be sure about the structure in question. To confirm the postulated structure, the chemist may prefer to synthesize the compound from known structure materials through well-understood reactions, and then check whether the synthetic material and the unknown compound are identical. Another way of attacking the problem of smaller posts can be unambiguously identified. One can often arrive at an unknown structure from the structures of smaller compounds if the reactions used in the degradation steps are well understood. It is important to be reminded that a proposed structure is only accepted if it is consistent with all known data on the compound.

Infrared (IR) spectroscopy, nuclear magnetic resonance (NMR) spectroscopy, ultraviolet (UV) spectroscopy, and mass spectrometry have all had great impact on chemistry and are widely used by organic chemists. These techniques will be discussed in this textbook with emphasis on their applications as tools for the identification of organic compounds. Nuclear magnetic resonance spectroscopy, which is the most informative and widely used technique that the organic chemist has today to study the molecular structure, will be presented in this chapter. Chapter 9 is dedicated to infrared spectrometry; ultraviolet spectroscopy is discussed in Chapter 29 and mass spectrometry in Chapter 32 (Section 32.4). In addition, Chapter 32 NMR is discussed in greater detail and illustrations are presented of how data from different spectral techniques are pooled to solve identification problems.

The NMR spectrum of a compound can be determined directly on the pure liquid. If the compound is a solid, the spectrum is determined in solution. The reason for this is that the spectrum must be determined on molecules that have free motion and thus equalize their interactions with neighboring molecules. Spectra of the type discussed herein can not be determined in solids. A wide variety of materials are satisfactory solvents for the determination of NMR spectra. If one wants to study protons, which is the usual situation, it is preferred to use a solvent that does not contain protons that may interfere. Carbon tetrachloride and deuterated solvents, such as D$_{2}$O or deuterated chloroform, are also commonly used.

\section{Guidance of a core in an external magnetic field}
All nuclei have charge and mass. Those that have an odd mass number or an odd atomic number also have a spin; that is, they have angular momentum. For example, $^1_1$H, $^2_1$H, $^{13}_6$C, $^{14}_7$N and $^{17}_8$O have spins, whereas the same does not occur with $^{12}_6$C and $^{16}_8$O. Any nucleus that has a spin can be studied by NMR, but this chapter will be limited to discussion of the nucleus of the proton, which in practice has given the most useful information.

The most common isotope of carbon ($^{12}_6$C) does not have NMR spectrum because it has no nuclear spin. This is unfortunate because the skeleton of most organic compounds is ultimately carbon. The next most abundant isotope of carbon ($^{13}_6$C) has an NMR spectrum that can be of considerable use. Since this isotope constitutes only 1 percent natural carbon wax, there has been a technical difficulty in obtaining $^{13}$C spectra of sufficient intensity to be useful. This difficulty has been solved recently and $^{13}$C spectra are expected to become increasingly important in the future (see Section 32.3).

A spinning charge, such as that of the core $^1_1H$, generates a magnetic field having a magnetic moment ($\mu$) associated therewith. Such a core can be regarded as the bar of a magnet. When an external magnetic field ($H_0$) is applied, the nucleus attempts to align its magnetic moment along the direction of the field, just as the bar magnet of a compass aligns with the Earth's magnetic field. The spin-quantum number of the nucleus is designated I; there are $2I+1$ orientations and corresponding possible energy levels for a magnetic core relative to the external field.

A proton, $^1_1H$ has a spin quantum number of $\frac{1}{2}$; thus, it has $2\times\frac{1}{2}+1=2$ possible orientations; parallel ($\uparrow$) and antiparallel ($\downarrow$) to the external magnetic field. In the absence of a magnetic field, each proton has the same spin nuclear energy and the spins are oriented in chaotic directions. In the presence of a magnetic field, the proton spins are aligned parallel or antiparallel to the field and the energy difference between these two orientations ($\Delta E$) is proportional to the intensity of the external magnetic field, $H_0$.

\begin{equation}
\Delta E = k H_0
\end{equation}

where $k = h\gamma/2\pi$; $\gamma$ is gyromagnetic ratio (a constant for a given nucleus); $H_0$ is the intensity of the external magnetic field; and h is Planck's constant. This variation in the energy level spacing ($\Delta E$) as a function of the applied field strength ($H_0$) is shown in Figure 5.1 for the $^1_1H$ core. The lowest energy level in Figure 5.1 corresponds to the parallel alignment ($\uparrow$) to the applied field and the highest energy level corresponds to the antiparallel alignment ($\downarrow$).

Figure 5.1 Separation of spin energy level for the hydrogen nucleus as a function of an external magnetic field ($H_0$).

For a given field strength, the proton can go from one energy level to the other by absorption or emission of a discrete amount of energy.

\begin{equation}
\Delta E = hv
\end{equation}

where $v$ is the frequency of radiation being absorbed or emitted. Combining equations 1 and 2, we find:

\begin{equation}
v = \frac{\gamma}{2\pi} H_0
\end{equation}

As can be seen in these equations, when protons are placed in a magnetic field that has a fiat intensity, there will be a defined frequency separating the two energy levels. In practice, a field of about 14,100 Gauss (G) requires a frequency of 60 megahertz (MHz) of energy (from the radiofrequency region of the electromagnetic spectrum) for the transition between the orientations. In a field of 23,500 G, 100 MHz are required; for a field of 47,000 G, 200 MHz are needed (Figure 5.1). It is important to note that 60 MHz corresponds to a very small amount of energy ($6\times10^{-3} cal/mol$). This means that the number of molecules in the ground state is slightly higher than the number of molecules in the excited state.

To obtain an NMR spectrum of a sample, it is placed in the magnetic field on the spectrometer, and a radiofrequency field is applied, a stream being passed through a coil wrapped around the sample (Figure 5.2). The magnetic field ($H_0$) is gradually increased and the excitation or 'oscillation' of the cores from one orientation to the other is detected as an induced voltage, resulting from the energy absorption of the radiofrequency field. An NMR spectrum (Figure 5.3) is a voltage plot induced against magnetic field scanning. The area under a 'peak' depends on the total number of cores that are 'oscillating'.

The energy absorbed by a nucleus can be released by spin-spin relaxation, in which the spin energy is transferred to a neighboring nucleus, or by spin-net relaxation, in which spin energy is converted into thermal energy. The nuclei are thus excited from the lowest to the highest spin state by a radiofrequency field. They spontaneously return to the lower energy state to be excited again, and so on.

\section{Shielding of hydrogen nuclei}
If all hydrogen nuclei absorbed energy at the same field strength for a given frequency, NMR spectroscopy would only be a method for quantitative proton analysis. In fact, it is much more than this. The field strength required for energy absorption by a given proton depends on its near environment, that is, on the molecular structure. By observing the field strength in which a proton absorbs, it is possible to deduce something about the local molecular structure.

Figure 5.3 Ethane NMR spectrum.

If our hydrogen atom is part of a molecule and the molecule is placed in a magnetic field, the field induces a circulation of electrons around the proton in a plane perpendicular to the external field. This circulating charge, in turn, generates a magnetic field induced in the region of the nucleus that is generally opposite to the external field (Figure 5.4). The electrons that surround the proton are said to shield the proton if the induced field opposes the external field, as in Figure 5.a. In this case, the electrons shield the core of the effects of the external field. On the other hand, electrons are said to disintegrate the nucleus if the induced field increases the external field. Hydrogen nuclei in different environments have greater or lesser electron density around them, and are shielded or disintegrated in different proportions by circulating electrons. The result is that the proton is subjected to a liquid or effective magnetic field.

\begin{equation}
H_{efetivo} = H_{externo}-H_{induzido}
\end{equation}

\emph{Figura 5.4 Shield ($H_{induced}$) caused by electrons circulating around the nucleus in plane perpendicular to the external field ($H_0$).}

When the magnetic field scan occurs, not all protons oscillate at the same field strength. The intensity of the field in which they oscillate depends on how much they are shielded, which, in turn, depends on the chemical environments.

From Figure 5.4 it can be seen that the shielding will depend on the electron density surrounding the nucleus of the hydrogen. For example, the protons in methyl iodide (CH3I) are more shielded than the methyl protons in methanol (CH3OH) because the electronegativity of oxygen is greater than that of iodine. The electron-attracting effect of oxygen is greater and reduces the electron density around methyl more than iodine. From the NMR spectrum of a compound we can therefore say the relative number of different protons present by the relative areas under the peaks and we can also say something about the environment of each proton.