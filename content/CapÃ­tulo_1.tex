\chapter{Introdução}

\section{AS PRIMEIRAS MOLÉCULAS ORGÂNICAS}

Se quisermos voltar ao início, é preciso regredir três bilhões de anos, ao tempo em que a vida apareceu na Terra. A Terra de então tinha todas as condições necessárias para o aparecimento da vida: temperatura estável, nem calor nem frio, energia solar abundante, massa suficiente para reter uma atmosfera propícia e os poucos ingredientes que compõem todas as formas de vida — carbono, hidrogênio, oxigênio e nitrogênio. Estes quatro elementos perfazem 98 por cento de todo tecido vivo. Isto é muito interessante, pois os elementos mais abundantes na crosta terrestre são oxigênio, silício e alguns metais leves. Os elementos acima, exceto o oxigênio, são elementos-traço, não passando de 1 por cento da crosta terrestre. 

Como começou a vida? Como os complexos compostos de carbono que a constituem puderam se formar a partir dos átomos e moléculas simples que existiam, quando nosso planeta tinha apenas um bilhão e meio de anos de idade? Em 1923, Oparin, químico russo, sugeriu que as primeiras moléculas orgânicas, os chamados precursores da vida, apareceram em um mundo que dispunha de pouco ou nenhum oxigênio. A atmosfera era, então, formada por vapor d'água, dióxido de carbono, nitrogênio, amoníaco (\ch{NH3}) e metano (\ch{CH4}). O sol iluminava a Terra, nuvens formavam-se, os raios e a chuva caíam. As substâncias radioativas decaíam, adicionando sua energia à do ambiente. Segundo Oparin, foi neste ambiente caótico que as primeiras moléculas orgânicas formaram-se e a vida foi possível. Os gases simples decompuseram-se e reações químicas mais complexas foram ocorrendo. 

Durante os anos cinquenta, Miller realizou algumas experiências para testar as idéias de Oparin. Em seu laboratório na Universidade de Chicago, Miller colocou em uma aparelhagem metano, amoníaco, água e hidrogênio de modo a simular uma atmosfera semelhante àquela que, segundo Oparin, rodearia a Terra primitiva. Quando Miller fez passar uma centelha na mistura para simular as descargas elétricas dos raios descobriu que, entre outras moléculas orgânicas, formavam-se amino-ácidos. Este resultado é notável, uma vez que todas as proteínas, componentes importantes da matéria viva, são formadas por amino-ácidos ligados entre si. 

Os compostos orgânicos simples, formados durante um grande período de tempo, dissolveram-se no oceano que, assim, se enriqueceu gradualmente, com uma enorme variedade de materiais orgânicos. Ainda não se sabe como estas primeiras moléculas orgânicas evoluíram até formar as células vivas. Com o tempo, o sistema desenvolveu-se de maneira a se organizar para o crescimento e a divisão de forma ordenada em duas partes idênticas, isto é, para a reprodução. Uma coisa, porém, é certa: neste processo de evolução, em colaboração com água, luz do sol e alguns poucos outros elementos, o carbono tem um papel central. 

\section{POR QUE CARBONO?}

Carbono é o elemento essencial, isto é, em torno do qual desenvolveu-se a química da vida. As proteínas, por exemplo, apenas um dos tipos de compostos de carbono, existem em uma variedade surpreendente de formas e funções. São moléculas complicadas, com pesos moleculares que vão de alguns milhares a milhões. Conhece-se a estrutura completa de apenas cinquenta proteínas, e só recentemente foi possível sintetizar uma das mais simples por métodos químicos em laboratório. A variedade de proteínas atualmente em função nos sistemas vivos é surpreendente. Usando apenas vinte amino-ácidos cada espécie existente na desenvolveu seu próprio conjunto de proteínas. Até mesmo uma bactéria fisiologicamente simples como \textit{Escherichia coli} contém cerca de 5000 compostos químicos diferentes, dos quais cerca de três mil são diferentes proteínas. O ser humano contém cerca de cinco milhões de proteínas diferentes e nenhuma delas é encontrada em \textit{E. coli}, ou em qualquer outro organismo vivo. Quando se considera o grande número de espécies diferentes que existem na Terra, é realmente surpreendente o número e variedade das proteínas. Os biólogos calculam o número de espécies em cerca de 1.200.000. Isto significa que cerca de $10^{12}$ tipos de proteínas participam dos processos vitais em curso na superfície da Terra. O carbono é o responsável por esta fantástica variedade. 

\begin{figure}[htbp]
    \centering
    \adjustbox{margin=1em,width=\textwidth,set height=4cm,set depth=4cm,frame,center}{Dummy}
    \caption{Aparelhagem usada por Miller para simular as condições que se acredita terem existido na Terra primitiva. Metano e amoníaco circulam continuamente entre o frasco inferior aquecido (o "oceano") e o frasco superior (a "atmosfera" que é atravessado por uma descarga elétrica.}
    \label{fig1_1}
\end{figure}

Por que o carbono é tão apropriado aos processos vitais? Por que nenhum dos outros quase cem elementos? Encontram-se as respostas quando se examina a estrutura atômica do carbono, pois é esta estrutura que lhe permite a formação de uma variedade de compostos muito maior do que os demais elementos. O carbono tem quatro elétrons na camada eletrônica externa. Cada um desses elétrons pode ser partilhado com outros elementos que sejam capazes de completar suas camadas eletrônicas por partilha de elétrons, formando ligações covalentes. (No Capítulo 2, discutiremos ligações covalentes com mais detalhes.) Nitrogênio, hidrogênio e oxigênio estão entre os elementos que podem se ligar desta maneira. um único átomo de carbono pode partilhar um máximo de quatro pares de elétrons para formar compostos do tipo do metano: 

\begin{figure}[htbp]
    \centering
    \adjustbox{margin=1em,width=\textwidth,set height=4cm,set depth=4cm,frame,center}{Dummy}
    \caption{Células de \textit{Escherichia Coli}, uma bactéria intestinal do ser humano, durante o processo do divisão celular. Tais células são compostas principalmente por água; as moléculas mais abundantes a seguir são as proteínas, que se constituem em cerca de 15 por cento do peso da célula. As proteínas servem de elementos estruturais de célula e participam das reações pelas quais a célula se mantém e se reproduz.}
    \label{fig1_2}
\end{figure}

\begin{tightcenter}
    \setchemfig{atom sep=2em}\chemfig[][]{H-C(-[2]H)(-[6]H)-H}
\end{tightcenter}

A característica mais importante do átomo de carbono, que o distingue de todos os demais elementos (exceto silício) e que explica seu papel fundamental na origem e evolução da vida, é sua capacidade de partilhar elétrons com outros átomos de carbono para formar ligações carbono-carbono. Este fenômeno simples é a base da química orgânica. É ele que permite a formação de estruturas de carbono lineares, ramificadas, cíclicas e semelhantes a gaiolas, com a participação de hidrogênio, oxigênio, nitrogênio e outros átomos capazes de formar ligações covalentes. Apenas aqueles poucos elementos que contêm quatro elétrons em sua camada de valência são capazes de formar ligações covalentes de forma repetitiva com o mesmo elemento. Destes, silício é o único elemento, além do carbono, capaz de formar tais ligações de forma relativamente estável. Porém, compostos contendo ligações silício-silício não resistem à atmosfera oxigenada da Terra, oxidando-se para formar sílica (\ch{SiO2}), o constituinte principal da areia e do quartzo, materiais incapazes de sustentar a vida. Assim, pelo menos na Terra, apenas o carbono é capaz de fornecer a espinha dorsal dos componentes moleculares dos seres vivos. 

\section{A QUÍMICA DO CARBONO E O PLANETA TERRA}

Foram precisos cerca de 4,5 bilhões de anos de luz solar e compostos de carbono para formar a vida atualmente existente na Terra. O ser humano, um neófito; existe na Terra há apenas alguns milhões de anos e, no espaço de somente cem anos, aprendeu a transformar compostos de carbono em medicamentos, combustíveis e produtos industriais em larga escala.

Até recentemente, nossa capacidade de efetuar estas transformações industriais com moléculas orgânicas era considerada uma bênção que permitia ao ser humano e suas máquinas exercerem poder sobre o meio ambiente. Conseguimos drogas maravilhosas, fibras fantásticas, detergentes miraculosos, supercombustíveis. Conseguimos aspirinas, pílulas de controle da natalidade, brinquedos plásticos, refrigerantes e comida suficiente para a maior parte das pessoas na maior parte do tempo. 

Mas, agora, temos o ar poluído. Peixes estão morrendo em lagos e rios onde não mais ousamos nadar. As praias e os mares estão ameaçados por uma mortífera combinação de óleo cru e dejetos industriais. Relatórios governamentais americanos indicam que cerca de 500000 novos tipos de moléculas sintéticas, transportadas pelos rios do mundo, terminam no mar a cada ano. O chumbo da gasolina já é encontrado nas neves distantes das regiões polares. A gordura de focas e pinguins está contaminada por DDT. 

O problema do DDT merece um comentário à parte. Este composto químico controlou a população de insetos do mundo a tal ponto que a Terra é agora capaz de produzir comida suficiente para alimentar a população humana. Mas este resultado altamente positivo tem seu lado negativo: os níveis de DDT na comida estão atingindo proporções perigosa à saúde. Poderíamos deixar de usar DDT (e deixar morrer de fome uma proporção maior da humanidade), mas uma melhor solução a longo prazo precisa ser encontrada. 

\begin{figure}[htbp]
    \centering
    \adjustbox{margin=1em,width=\textwidth,set height=4cm,set depth=4cm,frame,center}{Dummy}
    \caption{A idade industrial trouxe poluição e prosperidade.}
    \label{fig1_3}
\end{figure}

\begin{figure}[htbp]
    \centering
    \adjustbox{margin=1em,width=\textwidth,set height=4cm,set depth=4cm,frame,center}{Dummy}
    \caption{Uma consequência importante de recente evolução cultural da humanidade foi a poluição do ar, da água e da terra por dejetos industriais e da agricultura.}
    \label{fig1_4}
\end{figure}

A indústria química orgânica mudou nosso mundo para melhor e para pior. Poluição e superpopulação são problemas sérios que se estão tornando ainda mais sérios. São problemas científicos e políticos. A parte científica tem solução, embora dispendiosa. A parte politica é menos clara. É uma questão de como utilizar nossas riquezas - uma questão de prioridades. Como estudantes de química orgânica vocês fariam bem em ter estes problemas na cabeça. Muitos de vocês serão, provavelmente, biólogos, físicos, professores ou pesquisadores em química. Nestas e em outras posições, vocês terão possibilidades de resolver alguns dos problemas que outros, antes de vocês, ajudaram a criar. A qualidade da vida, talvez a própria vida, depende de sua ligação e respeito para com a Terra. Aqueles que têm algum conhecimento de química orgânica terão uma oportunidade de trabalhar para tal e a obrigação de fazê-lo. Nós esperamos sua participação nesta tarefa. 
\begin{figure}[htbp]
    \centering
    \adjustbox{margin=1em,width=\textwidth,set height=4cm,set depth=4cm,frame,center}{Dummy}
    \caption{O desenvolvimento de novas cepas de trigo, arroz e milho aumentou a produtividade agrícola no mundo, proporcionando uma solução parcial e temporária para os problemas de subnutrição e superpopulação.}
    \label{fig1_5}
\end{figure}