\chapter{Grupos Funcionais Formados por Ligações Simples}
\section{ÁCIDOS E BASES: UMA REVISÃO}

Em 1887, o químico sueco Arrhenius propôs que se desse o nome de acido às substancias cujas soluções aquosas contivessem um excesso de íons hidrogênios. \ch{H+}, e que se chamasse base as substancias que em soluções aquosas contivessem um excesso de íons \ch{OH-}. Hoje em dia sabemos que as reações químicas ocorrem de modo a formar os produtos de configuração eletrônica mais estável. Assim, não deveríamos esperar que o átomo de hidrogênio de um acido se dispusesse a ceder o par de elétrons que o liga ao restante da molécula, perdendo a configuração mais estável, semelhante ao gás nobre hélio, e transformando-se em um núcleo de hidrogênio (próton) em solução aquosa. Na verdade, a fracão de prótons que permanece livre em água é de para $10^{100}$ (isto é, um próton livre em uma piscina do tamanho do Universo conhecido cheia de água!). A reação que acontece quando cloreto de hidrogênio se dissolve em água pode ser representada com se segue: 

\begin{figure}[H]
    \centering
    \ch{HCl + H2O -> H3O^+}, \ch{H5O2^+}, \ch{H7O3^+}, \ch{H9O4^+}, etc. \ch{+ Cl^-}
\end{figure}

\noindent A formula \ch{H3O+} é usada habitualmente para representar o próton hidratado, embora normalmente os químicos utilizem apenas a formula \ch{H+}.

Podemos perceber que a formulação de Arrhenius, embora profunda para a época em que foi enunciada, é de utilidade bastante limitada, pois só se aplica a soluções aquosas. Um conceito mais amplo, e igualmente importante, foi apresentado em 1923, independentemente, por Br{\o}nsted e Lowry. Eles propuseram que deveríamos considerar ácidos quaisquer substancias capazes de doar prótons a outras, isto é, doadores de prótons, e que deveríamos considerar bases quaisquer substancias capazes de aceitar os prótons doados pelos ácidos, isto é, aceptores de prótons.

Para exemplificar o ponto de vista de Br{\o}nsted e Lowry, considere a reação do cloreto de hidrogênio com água:

\begin{figure}[H]
    \centering
    \ch{HCl + H2O <=> H3O^+ + Cl^-}
\end{figure}

Interpreta-se a reação acima do seguinte modo: no sentido da esquerda para a direita o cloreto de hidrogênio (acido) doa um próton à água (base). No sentido oposto, \ch{H3O+} (acido) doa um próton ao íon cloreto. \ch{Cl-} (base), Um par de ácidos e bases que se formam mutuamente pelo intercambio de um próton, como no caso acima, é chamado um par conjugado acido-base.

\begin{figure}[H]
    \centering
    \ch{!(\text{Ácido\enspace conjugado})( HA )}\ch{<=> [ -H^+ ][ H^+ ]}\ch{!(\text{Base\enspace conjugada})( A^- )}
\end{figure}

\noindent Quanto mais forte um acido, mais fraca será sua base conjugada e vice-versa. 

Um corolário importante do conceito de Br{\o}nsted e Lowry é que o equilíbrio em qualquer reação de transferência de prótons favorecera a formação do ácido mais fraco e da base mais fraca, às custas do acido mais forte e dar base mais forte.

\begin{table}[H]
    \centering
    \begin{tabular}{ccccccc}
        Ácido mais forte & + & Base mais forte & \ch{<=>} & Ácido mais fraco & + & Base mais fraca  \\
        \ch{HCl} & + & \ch{H2O} & \ch{<=>} & \ch{H3O+} & + & \ch{Cl-} \\
         \ch{H2O} & + & \ch{NH2-} & \ch{<=>} & \ch{NH3} & + & \ch{OH-} \\
         \ch{CH3OH} & + & \ch{CH3-} & \ch{<=>} & \ch{CH4} & + & \ch{CH3O-}
    \end{tabular}
\end{table}

\noindent Os compostos que apresentamos a seguir estão ordenados de acordo om sua acidez ou basicidade.

\begin{table}[H]
    \centering
    \begin{tabular}{lllrl}
                   & \multicolumn{2}{l}{Ácidos fracos} & \multicolumn{2}{r}{Ácidos fortes} \\
        Acidez     & \multicolumn{4}{c}{\ch{CH4\enspace <\enspace NH3\enspace <\enspace H2O\enspace <\enspace NH4+ \enspace <\enspace H2S\enspace <\enspace HF}}                        \\
                   & \multicolumn{2}{l}{Bases fortes} & \multicolumn{2}{r}{Bases fracas} \\
        Basicidade & \multicolumn{4}{c}{\ch{CH3- \enspace>\enspace NH2- \enspace>\enspace HO- \enspace>\enspace NH3 \enspace>\enspace HS- \enspace>\enspace F-}}                       
    \end{tabular}
\end{table}

Apesar de sua amplidão, este conceito ainda esbarra em uma limitação severa imposta pela palavra 'próton' propondo, em 1923, que acido fosse definido como sendo toda especie capaz de aceitar um par de elétrons e que uma base fosse definida como toda especie capaz de ceder um par de elétrons. Segundo o conceito de acidez e basicidade de Lewis toda reação acido-base não passa do partilhamento de um par de elétrons entre uma base e um acido. Frequentemente o resultado da reação é a formação de uma ligação covalente entre o acido e a base. A transferência de prótons é apenas um caso particular do conceito de Lewis. Vale a pena ressaltar que embora a formulação do Leis possa nos parecer, hoje, tão obviamente racional, ela revolucionou toda a compreensão da Química, especialmente da Química Orgânica. 

A conceituação de base é praticamente a mesma para Lewis e para Br{\o}nsted-Lowry. As especies capazes de partilhar seu par de elétrons com um aceptor (acido de Lewis) geralmente faro o mesmo com um próton (acido de Br{\o}nsted-Lowry). Por outro lado, a definição de Lewis expande enormemente o conceito de acido, como bem mostra o exemplo abaixo:

1. Íons positivos:

\begin{table}[H]
    \centering
    \begin{tabular}{ccccc}
        Ácido &  & Base &  & Complexo de coordenação \\
        \ch{Ag+} & + & \ch{2 $n$ NH3} & \ch{->} & \ch{[NH3-Ag-NH3]+} \\
        \ch{NO2^+} & + & \ch{CH2=CH2} & \ch{->} & \ch{^{+}CH2-CH2NO2} \\
        \ch{CH3^+} & + & \ch{CH3OCH3} & \ch{->} & \ch{(CH3)3O+}
    \end{tabular}
\end{table}

2. Compostos contendo um átomo com octeto incompleto: estes compostos estão entre os mais importantes ácidos de Lewis e muitos são especialmente úteis como catalisadores em reações químicas.

\begin{table}[H]
    \centering
    \begin{tabular}{ccccccc}
        Ácido &  & Base &  & \multicolumn{3}{c}{Complexo de coordenação} \\ [1ex]
        \chemfig{\lewis{2:4:6:,Fe}(-[2,0.4,,,draw=none]Cl)(-[4,0.5,,,draw=none]Cl)(-[6,0.4,,,draw=none]Cl)} & + & \ch{Cl2} & \ch{->} & \chemfig{\lewis{0:2:4:6:,Fe}(-[2,0.4,,,draw=none]Cl)(-[4,0.5,,,draw=none]Cl)(-[6,0.4,,,draw=none]Cl)(-[0,0.5,,,draw=none]Cl)} & + & \ch{Cl+} \\ [3ex]
        \ch{AlCl3} & + & \ch{(CH3)3CCl} & \ch{->} & \ch{ACl^-4} & + & \ch{(CH3)3C^+} \\ [1ex]
        \ch{BF3} & + & \ch{(CH3CH2)2O} & \ch{->} & \multicolumn{3}{c}{\ch{(CH3CH2)2^+O-^{-}BF3}}
    \end{tabular}
\end{table}

Vale a pena, neste ponto, rever alguns conceitos familiares importantes.

A acidez de uma solução aquosas é definida em termos da constante de equilíbrio \Ka da reação:

\begin{figure}[H]
    \centering
    \ch{HA + H2O <=> H3O+ + A-}
\end{figure}

\noindent Se admitirmos os coeficientes de atividade como sendo iguais a 1,

\begin{figure}[H]
    \centering
    $\Ka = \dfrac{[H_{3}O^{+}][A^{-}]}{[HA]}$
\end{figure}

\noindent (onde os colchetes indicam concentração) e $\pKa = -log \Ka$, vemos, então, que o acido será tão mais forte quanto maior for o valor de \Ka \enspace e menor for o de \pKa. O valor \pKa\enspace é muito importante porque é diretamente proporcional à energia-livre padrão, \gibbs*{}, envolvida na ionização. A equação

\begin{figure}[H]
    \centering
    $\pKa = \dfrac{\gibbs*{}}{2.303RT}$
\end{figure}

\noindent mostra esta proporcionalidade. $R$ é a constante dos gases e $T$ a temperatura absoluta.

A basicidade pode ser definida de maneira análoga usando-se a equação

\begin{figure}[H]
    \centering
    \ch{B + H2O <=> BH+ + OH-} e $\Ka = \dfrac{[BH^{+}][OH^{-}]}{[B]}$
\end{figure}

\noindent Um modo alternativo é exprimir a basicidade de $B$ em fincão da acidez de \ch{BH+}. Quanto mais fraca a base $B$, mais forte será seu acido conjugado \ch{BH+}. Assim, na reação 

\begin{figure}[H]
    \centering
    \ch{BH+ + H2O <=> B + H3O+}
\end{figure}

\noindent a acidez de \ch{BH+} é medida por \Ka:

\begin{figure}[H]
    \centering
    $\Ka = \dfrac{[H_{3}O^{+}][B]}{[BH^{+}]}$
\end{figure}

\noindent Das formulas acime segue-se que \Ka, para \ch{BH+} e \Kb, para $B$, se relacionam por

\begin{figure}[H]
    \centering
    $\Ka\Kb=\Kw$
\end{figure}

\noindent onde \Kw\enspace é produto iônico de água a 25$\degree$C, $\Kw = 1,0 \times 10^{-14}$, ou, por

\begin{figure}
    \centering
    $\pKa + \pKb = 14,00$
\end{figure}

\noindent No presente texto usaremos \Ka\enspace e \pKa, de modo que em lugar de discutirmos \pKb\enspace para o \ch{NH3} (4,76), por exemplo, discutiremos \pKa\enspace para o ácido conjugado \ch{NH4+} (9,24).

Decorre disto tudo que o \pKa\enspace de um ácido é o \pH, em soluções aquosa em que $[AH^+] = [A^-]$, isto é, em uma solução aquosa neutralizada pela metade. Os valores de \pKa\enspace estabelecem uma escala de acidez relativa com referência a uma base de comparação - a água. Quanto maior for o valor de \pKa\enspace mais fraco será o ácido. Fora da faixa 1-13 de valores de \pKa\enspace os valores não mais se relacionam ao grau de ionização. Os ácidos muito fortes (\pKa\enspace < 1) estão completamente ionizados em água, ocorrendo um 'nivelamento' de acidez, isto é, o ácido mais forte existente nestas soluções é o \ch{H3O+}. Ácidos com \pKa\enspace maiores do que 13 não mostram qualquer acidez que possa ser medida com facilidade em relação à água. A acidez, nestes casos, tem que ser determinada por métodos indiretos. Voltaremos a comentar o assunto na Seção 12.3.

\section{GRUPOS FUNCIONAIS}

Um átomo ou grupos de átomos que caraterizam uma classe de compostos orgânicos e determinam as propriedades da mesma são chamados grupos funcionais. Os álcoois, por exemplo, têm formula geral ROH (onde R é um grupamento alquila) e seu grupo funcional é a hidroxila, -OH. Os grupos funcionais são os responsáveis pelas propriedades físicas e químicas mais características dos membros da classe de compostos. Essencialmente, os grupos funcionais são a parte não-hidrocarbônica das moléculas. 

Os grupos funcionais simples são formados pela ligação de um heteroátomo (halogênio, oxigênio, enxofre, nitrogênio, etc.) ao carbono de um grupamento alquila, através de uma ligação simples. Neste capítulo, nós nos preocuparemos com a estrutura, nomenclatura e propriedades das classes mais comuns de compostos orgânicos cujos grupos funcionais só contenham ligações simples (ligações sigma). Mais adiante (Seção 4.18) já estaremos em condições de generalizar algumas das propriedades destes compostos que são afetadas pelas forcas intermoleculares devidas aos grupos funcionais. Quanto aos grupos funcionais que contenham ligações múltiplas, nós os deixaremos para capítulos subsequentes.

À medida que continuarmos a explorar a Química Orgânica, duas generalizações importantes se tornarão evidentes:

\begin{enumerate}
    \item A química dos compostos orgânicos é um resultado da química dos grupos funcionais presentes. Os resíduos de hidrocarbonetos agem como fatores modificadores e não como fatores primários, seja nas propriedades física, seja no comportamento químico dos compostos considerados
    \item O comportamento químico das substâncias pode ser compreendido com base na estrutura eletrônica dos átomos ou grupos funcionais.
\end{enumerate}

Antes de passarmos à discussão das classes individuais de compostos, vale a pena discutir um pouco mais o problema das relações entre geometria e hibridação.

\section{MAIS SOBRE HIBRIDAÇÃO}

No Capítulo 2 discutirmos a hibridação sp3 do átomo de carbono, onde usávamos, para formar os quatro orbitais, um orbital atômico $s$ e três orbitais atômicos $p$. Muitos outros tipos de orbitais podem se formar e, na verdade, não há limites para o número deles, pois basta que adicionemos o novo orbital aos já existentes e os utilizemos em uma nova operação de hibridação. Por exemplo, podemos deixar de lado um dos três orbitais $p$ (digamos o orbital $p_z$) e combinar os outros dois ($p_y$ e $p_x$) com o orbital $s$ para obter três orbitais híbridos $sp^2$. A forma de um orbital $sp^2$ é semelhante à de um orbital $sp^3$: há uma grande densidade eletrônica de um dos lados do núcleo e uma pequena densidade do outro lado. Os três orbitais $sp^2$ estão situados em um plano ($xy$) e seus eixos fazem 120$\degree$ entre si. O orbital $p_z$ é perpendicular a este plano (Figura 4.1).

Assim como foi feito para os orbitais $sp^3$, podemos deixar de utilizar dois orbitais $p$ e combinar apenas um deles com o orbital $s$ para obter dois orbitais $sp$, cujos eixos formam um angulo de 180$\degree$ entrei si (Figura 4.2)

%\begin{figure}[H]
%    \centering
%    \chemsetup[orbital]{overlay, opacity = 0.75, p/scale = 1.6, s/color = blue!50, s/scale = 1.6}
%    \chemfig{\orbital{s}-[:-20]{\orbital[scale=2]{p}}{\orbital[half,angle=0]{p}}{\orbital[angle=170,half]{p}}{\orbital[angle=-150,half]{p}}(-[:-150]\orbital{s})-\orbital{s}}
%\end{figure}

Os orbitais híbridos podem formar ligações simples - ligações sigma ($\sigma$) - relativamente fortes, porque a densidade eletrônica dos orbitais localiza-se principalmente entre os núcleos que formam a ligação. Isto significa que na distancia internuclear de equilíbrio ocorre entrosamento suficiente entre os orbitais (Figura 4.3).

Uma vez que, a uma distancia fixa, o entrosamento aumenta com o aumento da contribuição relativa do orbital $s$ ao orbital híbrido, podemos ordenar as ligações no sentido decrescente das forcas de ligação:

Para compreender porque certas geometrias são observadas comumente em moléculas lembremo-nos que somente dois elétrons podem ocupar o mesmo orbital - e mesmo assim se tiverem spins com sinais opostos (principio da exclusão de Pauli). Por causa deste principio, assim das leis da eletrostática, os elétrons dos diferentes orbitais ocupados se repelem. O arranjo mais favorável no es paco tridimensional será aquele em que os orbitais estiverem o mais afastados possível. Para dois orbitais é linha reta, para três, o trigonal plano, para quatro, o tetraedro regular e para cinco, a bipirâmide trigonal (Figura 4.4).

\section{ESTRUTURAS MOLECULARES}

Já que os ângulos dos orbitais, até agora, estão restritos aos valores 90$\degree$, 109$\degree$, 120$\degree$ e 180$\degree$, provenientes das hibridações discutidas acima, é razoável esperar que os ângulos de ligação, exceto quando deformados por exigências de tensão nos anéis , se restrinjam também àqueles valores. Na prática, entretanto, os estados de hibridação e os ângulos de ligação resultantes em compostos acíclicos variam muito, o que faz com que os casos discutidos anteriormente sejam muito especiais. Para moléculas reais, a hibridação e a geometria são reguladas pela imposição de manter-se a menor energia total para a molécula. Isto é obtido através de um compromisso entre os inúmeros fatores energéticos, incluídos os estados de hibridação. A seção anterior poderia nos levar a crer que a hibridação leva sempre a um numero apropriado de orbitais híbridos idênticos. Na verdade não é necessário que isto aconteça sempre. Por exemplo, consideremos um átomo hibridado com orbitais sp2. Não há necessidade dos três orbitais serem idênticos. Um ou dois deles podem ter um maior caráter $p$ desde que os restantes tenham, correspondentemente, um menor caráter $p$. As pequenas mudanças na hibridação são acompanhadas por mudanças nos ângulos e comprimentos de ligação. Um aumento de caráter $s$ em dois orbitais no mesmo átomo tende a aumentar o angulo entre eles (Quadro 4.1).

Em geral pode-se fazer previsões razoáveis sobre a estrutura das moléculas considerando-se as seguintes regras:

\begin{enumerate}
    \item Forma-se o maior numero do ligações em torno de cada átomo, levando-se em conta a teoria de Lewis.
    \item Minimizam-se as repulsões eletrônicas.
\end{enumerate}

A primeira regra já foi discutida durante a análise da estrutura do mento (Seção 2.5) O hidrocarboneto mais simples é \ch{CH4} a não \ch{:CH2}. Embora nossos argumentos tenham, então, sido apresentados de modo diferente, o resultado pode ser resumido se dissermos que é melhor, do ponto de vista energético, ter oito elétrons em ligação (como no metano) do que apenas quatro em em ligação (as duas ligações sigma do \ch{:CH2}) e mais quatro em orbitais atômico (do no \ch{:CH2}) e dois no átomos de hidrogênio):

Para compreendermos como a regra 2 se aplica a uma determinada molécula é preciso considerar as repulsões entre os pares de elétrons que formam as várias ligações e, também, as repulsões entre os pares de elétrons em ligações e os para de elétrons livres (não ligados). Em nossa análise não é necessário levar em conta as repulsões entre os núcleos. Deve-se esperar que um par de elétrons libre produze uma densidade eletrônica maior próximo ao núcleo ao qual está ligado do que o faria em um orbital de ligação, pela simples razão de que, no segundo caso, a atração do par de elétrons é compensada pela existência do segundo núcleo (Figura 4.5) Da mesma forma, pares de elétrons livres no mesmo átomo se repelem mutuamente muito mais do que o fazem quando em orbitais de ligação. Por causa diferentes dos ângulos ideais, isto é, dos ângulos que esperaríamos observar na base de uma hibridação ideal. Em resumo, a importância das repulsões que se verificam entre pares de elétrons de um átomo decresce na seguinte ordem: par livre-par livre > par livre-par em ligação > par em ligação-par em ligação. Uma boa ilustração para o que foi dito é variação dos ângulos de ligação na série isoeletrônica metano, amoníaco, água (Figura 4.6).

Mesmo em compostos de carbono compatíveis com a hibridação $sp3$, o angulo de ligação de 109 é raramente encontrado. As configurações geométricas exatas só costumam ser encontradas naqueles caso em em que a substituição no átomo em consideração é completamente simétrica, como no caso de metano e tetracloreto de carbono (CCl4). Os fatores que interferem com essa expectativa idealizada são (1) substituição assimétrica em torno do átomo central, frequentemente acompanhada de pequenas modificações na distribuição dos caráteres $s$ e $p$ dos diversos orbitais híbridos envolvidos: (2) tensão de anel e (3) tensão estérica (Figura 4.7).

Um modelo de hibridação $sp3$ proporciona uma base razoável para previsões acerca dos ângulos de ligação em torno dos elementos seguintes, quando aparecem com átomos centrais ligados a outros grupos exclusivamente por legações sigma:

Como os halogênios costumam ser monovalentes, não precisamos nos preocupar com os seus estados de hibridação na predição de ângulos de ligação. A compressão de ângulos de ligação em relação ao modelo $sp3$ pela repulsão de pares isolados é importante nos elementos enxofre e fosforo, do segundo período da Tabela Periódica.

\section{HALOGENETOS DE ALQUILA}

Halogenetos de alquila são derivados estruturalmente dos hidrocarbonetos por substituição de um átomo de hidrogênio por um átomo de halogênio. Qualquer átomo de hidrogênio de um hidrocarboneto pode ser substituído por halogênio. Na verdade todos os átomos de hidrogênio podem ser trocados ao mesmo tempo e, por exemplo, compostos totalmente fluorados, ou fluorocarbonetos, são muito importantes graças à sua alta estabilidade térmica. 

Os químicos costumam usar a notação \textbf{RX} para representar os halogenetos de alquila. \textbf{R} representa qualquer grupo alquila e \textbf{X} qualquer halogênio, exceto quando especificado diferentemente. Comecemos nosso estudo dessas moléculas revendo as configurações do estado fundamental dos halogênios. Observemos que, em cada caso, ao átomo de halogênio, eletronegativo, falta um elétron que lhe permita a configuração de um á gás nobre. Isto significa que é razoável esperar que formem moléculas estáveis através de apenas uma ligação simples iônica ou covalente.

Além disso, tais halogênios terão pares de elétrons livres, logo, podemos antecipar que tais halogênios covalentes (e iônicos) funcionarão como bases de Lewis.

Fluoreto, cloreto, brometo e iodeto de metila são formados pelo entrosamento de um orbital $sp^3$ do carbono com os orbitais $2p$, $3p$, $4p$ e $5p$ do flúor, cloro, bromo e iodo, respectivamente. A força das ligações C-X diminui com o aumento do peso atômico do átomo X. Isto é uma consequência do princípio geral que estipula que o entrosamento dos orbitais é maior quando a ligação é feita entre átomos do mesmo período da Tabela Periódica e decresce em eficacia à medida que aumenta a diferença de número quântico principal entre os elétrons envolvidos na ligação. A justificativa para isto é o tamanho relativo dos orbitais envolvidos (Figura 4.8). O orbital $2sp^3$, relativamente pequeno, não consegue um bom entrosamento com os orbitais $p$, maiores, de maneira a formar uma ligação forte.

\section{NOMENCLATURA DOS HALOGENETOS DE ALQUILA}

os halogenetos de alquila simples sao nomeados como derivados alquilados do halogenetos de hidrogênio. No sistema UIQPA eles são considerados como derivados halogenados do hidrocarbonetos. Nos exemplos deste capítulo apresentaremos os nomes vulgares entre parêntesis, abaixo dos nomes do sistema UIQPA. Na nomenclatura vulgar (ou trivial) os prefixos abreviados $n$-, $s$- e $t$- significam, respectivamente, normal, secundário e terciário (ver a Seção 3.1). 

Com o sistema UIQPA, a tarefa de dar nomes aos compostos que contêm exclusivamente funções univalentes (funções simples ligadas ao carbono por ligações simples, tais como -Cl (cloro), -Br (bromo), -NO2 (nitro), etc.), que podem ser expressos pelo uso exclusivo de prefixos, decorre facilmente em consequência do processo de nomeação do hidrocarboneto principal. O princípio da numeração que dá aos substituintes da cadeia do composto principal os menores números possíveis é sempre seguido.

O mais importante dos periódicos de indexação e resumo de trabalhos científicos e patentes industriais em Química, o Chemical Abstracts, escreve o nome dos substituintes em ordem alfabética, qualquer que seja sua numeração (Seção 3.1). A ordem na qual um nome composto é escrito não tem grande importância para a compreensão da estrutura dos compostos que deseja designar mas é, por outro lado, extremamente importante para fins de indexação.

O Chemical Abstracts é publicado semanalmente pela Sociedade Americana de Química (American Chemical Society). A revista sumaria patentes químicas e todos os artigos originais de Química que aparecem nas revistas cientificas de todos os países do mundo. Duas vezes por ano são editados índices completos de autores, assuntos e formulas. A cada cinco anos é editado um indica cumulativo, O uso do Chemical Abstracts é a única maneira pratica de conduzir uma pesquisa bibliográfica sistemática de qualquer assunto em Química. Sem o uso do Chemical Abstracts dificilmente o ritmo de desenvolvimento da Química seria tao acelerado como é atualmente. Para ter-se uma idéia de como o conhecimento em Química se desenvolve basta considerar que, para um período tipico recente de seis meses, o Chemical Abstracts publicou 11.500 paginas contendo 122.00 resumos, além de um índice de 6.3000 paginas cobrindo o período.

Certos grupos hidrocarbonetos que têm mais de uma posição possível para substituição e que ocorrem com maior frequência possuem nomes vulgares ou triviais:

Os termos \textit{geminal} ($g$-) (do latim \textit{geminus}, gêmeo) e \textit{vicinal} ($v$-) ( do latim \textit{vicinus}, vizinho) são usados muitas vezes para designar posições relativas de substituintes, 1-1 e 1-2, respectivamente, nas discussões de aspectos estruturais de moléculas.

\section{PROPRIEDADES DE HALOGENETOS DE ALQUILA}

as propriedades físicas de alguns halogenetos de alquila representativos são dadas no Quadro 4.2. Muitos dos halogenetos são líquidos. Os brometos, iodetos e poli-halogenetos têm, em geral, densidade maior do que 1. Os halogenetos de alquila são insolúveis em água mas são miscíveis em todas as proporções com os hidrocarbonetos líquidos.

\begin{table}[H]
    \centering
    \caption{My caption}
    \label{my-label}
    \begin{tabular}{ccccc}
        \toprule
        Nome & Formula & Pf & Pe & Massa especifica (liquido) \\
        \midrule
        Fluoreto de metila & \ch{CH3F} & -142 & -79 & 0,877 \\
        Cloreto de metila & \ch{CH3Cl} & -97 & -23,7 & 0,920 \\
        Brometo de metila & \ch{CH3Br} & -93 & 4,6 & 1,732 \\
        Iodeto de metila & \ch{CH3I} & -64 & 42,3 & 2,279 \\
        Cloreto de etila & CH3CH2Cl &  &  &  \\
        Brometo de etila & CH3CH2I &  &  &  \\
        Cloreto de n-propila & CH3CH2CH2Cl &  &  &  \\
        Cloreto de isopropila & (CH3)2CHCl &  &  &  \\
        Brometo de n-butila &  &  &  &  \\
        Brometo de isobutila &  &  &  &  \\
        Brometo de s-butila &  &  &  &  \\
        Brometo de t-butila & \ch{(CH3)3CBr} & -20 & 73,3 & 1,222 \\
        Brometo de n-octadecila & \ch{CH3(CH2)17Br} & 34 & 170 & \\
        \bottomrule
    \end{tabular}
\end{table}

\noindent\emph{USOS DOS HALOGENETOS DE ALQUILA. Cloretos de alquila são bons solventes para muitos materiais orgânicos. Muitos solventes policlorados, estáveis e disponíveis com facilidade, são usados para finalidades gerais e industriais em Química. Dentre os mais importantes estão o cloreto de metileno (\ch{CH2Cl2}), o clorofórmio (\ch{CHCl3}) e o tetracloreto de carbono. Este último tem sido utilizado em extintores de incêndio, o que não é, aliás, muito recomendável, já que a temperaturas elevadas o tetracloreto de carbono pode reagir com o oxigênio do ar para dar fosgênio (\ch{COCl2}), que é um gás altamente tóxico. O Teflon, \ch{F(CF2)_{n}F}, onde $n$ é um número inteiro muito grande, é um material muito útil graças a sua extrema estabilidade térmica e química. Ele é escorregadio ao tao e é, por isso, utilizado para a confecção de pecas moveis que não necessitam de lubrificação. Um de seus empregos mais curiosos está na construção de edificações à prova de terremotos, pois sapatas deslizantes de teflon, incluídas nas fundações, permitem que o prédio se movimente sem desgastar seus suportes. Também é utilizado no revestimento interno de frigideiras já que as frituras tendem a não aderir à superfície de teflon. Muitos fluorocarbonetos, conhecidos como Freons, são usados como fluidos refrigerantes não tóxicos em sistemas de refrigeração e como propelentes em aerossóis.}

\section{COMPOSTOS OXIGENADOS}

Podemos formar derivados orgânicos da água trocando, formalmente, cada um de seus hidrogênios por um grupamento alquila. Quando um hidrogênio é substituído formamos um álcool (\textbf{ROH}). Uma vez que o oxigênio é muito eletronegativo, poderá suportar facilmente uma carga negativa; o hidrogênio de um álcool é, em consequência, fracamente ácido (\pKa\enspace = 15,5 a 19), do mesmo modo que o hidrogênio da água (\pKa\enspace = 15,3).

Quando os dois hidrogênios da água são substituídos por grupamentos alquila, obtemos um éter (\textbf{ROR}). Os éteres assemelham-se geometricamente à água e aos álcoois e nao apresentam nada de especial no modo de formar as ligações. Podemos antecipar que o angulo de ligação COC deve aumentar ligeiramente com o tamanho dos substituintes, o que é experimentalmente comprovado. 

Os derivados orgânicos do peroxido de hidrogênio são também conhecidos. Tanto os peróxidos (\textbf{ROOR}) quanto os hidroperóxidos (\textbf{ROOH}) são bastante instáveis e às vezes explosivos. Raramente nos preocupamos em isolá-los, embora sejam intermediários importantes nos processos de oxidação e combustão de substâncias orgânicas.

Uma vez que derivados oxigenados divalentes têm pares de elétrons livres, comportam-se como bases de Lewis. Em consequência, são conhecido inúmeros íons de oxigênio tricovalente análogos ao hidrogênio (H3O+). Cátions do tipo ROH2+, formados da protonação de álcoois, e R2OH+, da protonação de éteres, não são isolados normalmente, existindo apenas em soluções ácidas e, transitoriamente, como intermediários em reações químicas. Os sais trissubstituídos de oxônio, R3O+Y-, embora menos estáveis do que os sais de amônio e sulfônio, podem ser isolados de meios inertes desde que os anions sejam pouco reativos e muito estáveis, como o fluoroboreto (BF4-). Os éteres formam complexos estáveis com os ácidos de Lewis: o sal trifluoro-boreto de dietil-éter, (CH3CH2)2O+--BF3, é um liquido estável que entra em ebulição a 126$\degree$C. O átomo de oxigênio tem ângulos de ligação próximos de 109$\degree$ em compostos deste tipo.

\section{NOMENCLATURA DE ÁLCOOIS}

O nome trivial dos álcoois simples consiste da palavra álcool seguida do nome do grupamento alquila correspondente. No sistema UIQPA troca-se o final $o$ do nome do hidrocarboneto correspondente pelo sufixo -$ol$, e a posição do grupo hidroxila (-OH) é indicada pelo menor número possível. Observe que a maior cadeia que contenha o átomo de carbono ao qual está ligada a hidroxila é a cadeia principal, não precisando ser necessariamente a maior cadeia de carbonos.

Frequentemente deparamos com a classificação, muito conveniente, dos álcoois em primários, secundários ou terciários, conforma o carbono da hidroxila esteja ligado a um, dois ou três carbonos. Nos exemplos seguintes, os alcoóis $n$-butílico e isobutílico são primários, o álcool $s$-butílico, secundário e o álcool $t$-butílico, terciário.

Muitos compostos contêm mais de um grupo hidroxila. O nome geral para esta classe de compostos é álcool poli-hídrico ou poliol. Compostos que contêm duas hidroxilas são chamados glicóis ou dióis, e os que têm três hidroxilas, trióis. Note que por razoes de eufonia o -$o$ final do hidrocarboneto não é omitido quando o sufixo começa com consoante. Assim, etano dá etanol, mas o diol é etanodiol.

Quando um composto contém um grupo hidroxila além de outras funções é conveniente usar o prefixo $hidroxi$- no lugar de -$ol$.

\section{PROPRIEDADES DOS ALCOÓIS}

A Tabela \ref{quadro_4_3} sumaria propriedades físicas de alguns álcoois comuns. Os álcoois metílico, etílico, n-propílico, isopropílico, terbutílico e muitos álcoois poli-hídricos são completamente miscíveis com a água. Outros álcoois são apreciavelmente solúveis, dependendo do número de átomos de carbono e de hidroxilas. Em geral a presença de uma hidroxila é suficiente para permitir que uma estrutura de três ou quatro carbonos seja solúvel. De um modo geral os álcoois têm ponto de ebulição bem maior do que os hidrocarbonetos e muitos outros compostos de mesmo peso molecular (Quadro 4.4). A grande solubilidade em água e os altos pontos de ebulição podem ser atribuídos à formação de ligações hidrogênio ou pontes de hidrogênio (Seção 4.18).

\begin{table}[H]
    \centering
    \caption{Álcoois.}
    \label{quadro_4_3}
    \begin{tabular}{ccccc}
        \toprule
        Nome & Formula & Pf & Pe & Massa especifica (líquido) \\
        \midrule
        Álcool metílico & \ch{CH3F} & -142 & -79 & 0,877 \\
        Álcool etílico & \ch{CH3Cl} & -97 & -23,7 & 0,920 \\
        Alcool metilico & \ch{CH3Br} & -93 & 4,6 & 1,732 \\
        Iodeto de metila & \ch{CH3I} & -64 & 42,3 & 2,279 \\
        Cloreto de etila & CH3CH2Cl &  &  &  \\
        Brometo de etila & CH3CH2I &  &  &  \\
        Cloreto de n-propila & CH3CH2CH2Cl &  &  &  \\
        Cloreto de isopropila & (CH3)2CHCl &  &  &  \\
        Brometo de n-butila &  &  &  &  \\
        Brometo de isobutila &  &  &  &  \\
        Brometo de s-butila &  &  &  &  \\
        Brometo de t-butila & \ch{(CH3)3CBr} & -20 & 73,3 & 1,222 \\
        Brometo de n-octadecila & \ch{CH3(CH2)17Br} & 34 & 170 & \\
        \bottomrule
    \end{tabular}
\end{table}

\noindent EMPREGO DOS ÁLCOOIS. \emph{Todos os álcoois listados no Quadro 4.3 têm importância comercial como solventes, intermediários químicos ou ambas as coisas. Como veremos nos capítulos seguintes, os álcoois são compostos muito versáteis, podendo ser transformados em quase todas as demais classes de compostos alifáticos. De todos, o mais importante é o álcool etílico, conhecido comumente como álcool. É um solvente industrial e farmacêutico muito importante, sendo, ainda, muito utilizado como meio reacional em química e como ingrediente ativo em muitas bebidas agradáveis. Um outro membro da série, o etileno-glicol, é usado como anticongelante, aditivo para motores em países muito frios.}

Em 1974, os impostos do governo norte-americano sobre as bebidas alcoólicas chegaram a US\$ 5.358.477.000,00! O imposto alcança cerca de US 4,60 por litro de álcool absoluto. A fermentação é ainda o processo mais importante para a obtenção de álcool. É um processo bioquímico catalisado por enzimas, e que ocorre em duas etapas: a primeira envolvendo a conversão de polissacarídeos (açucares, Capítulo 25) em monossacarídeos, partindo de quaisquer fontes naturais (a fonte principal ainda é a cana-de-açúcar) e a segunda envolvendo a conversão do monossacarídeo obtido em álcool etílico. 