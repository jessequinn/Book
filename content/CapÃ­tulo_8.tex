\chapter{Grupos Funcionais que Contêm Oxigênio Ligado Duplamente a um Átomo de Carbono: o Grupo Carbonila}

Nos Capítulos 3 e 7 discutimos com algum detalhe as estruturas e propriedades dos alcanos e alquenos. No Capítulo 4 discutimos os compostos que contêm um ou mais heteroátomos em ligações simples. Iniciaremos agora o estudo do grupo de compostos que contêm ligações múltiplas envolvendo heteroátomos. Começaremos pela ligação dupla carbono-oxigênio, retornando ao assunto no Capítulo 10, que introduz outras ligações múltiplas envolvendo heteroátomos.

\section{O GRUPO CARBONILA}

O arranjo \ch{R2C=O} é chamado \textit{grupo carbonila}, e os compostos que contêm este arranjo de átomos são chamados de \textit{compostos carbonilados}. A Figura \ref{figurar_8_1} mostra as estruturas das classes funcionais mais comuns de compostos que contêm o grupo carbonila:


\begin{figure}[H]
    \centering
    \chemnameinit{}
    \chemname{\setchemfig{atom sep=2em}\chemfig{R-C(=[2]O)-H}}{Aldeído}
    \qquad
    \chemname{\setchemfig{atom sep=2em}\chemfig{R-C(=[2]O)-R^1}}{Cetona}
    \qquad
    \chemname{\setchemfig{atom sep=2em}\chemfig{R-C(=[2]O)-OH}}{Ácido carboxílico}
    \qquad\qquad
    \chemname{\chemname{\setchemfig{atom sep=2em}\chemfig{R-C(=[2]O)-OOH}}{Ácido peróxi-carboxílico}}{(Ácido percarboxílico)}
    \par\bigskip
    \begin{flushleft}
        Derivados de ácidos:
    \end{flushleft}
    \par\bigskip
    \chemnameinit{}
    \chemname{\setchemfig{atom sep=2em}\chemfig[][]{R-C(=[2]O)-OR^1}}{Éster}
    \qquad
    \chemname{\setchemfig{atom sep=2em}\chemfig{R-C(=[2]O)-NR^1_2}}{Amida}
    \qquad
    \chemname{\setchemfig{atom sep=2em}\chemfig{R-C(=[2]O)-O-C(=[2]O)-R}}{Anidrido}
    \qquad\qquad
    \chemname{\setchemfig{atom sep=2em}\chemfig{R-C(=[2]O)-X}}{Halogeneto de acila}
    \caption{Classes funcionais que contêm o grupo carbonila.}
    \label{figurar_8_1}
\end{figure}

\noindent Os grupos funcionais têm o nome particular de:

\begin{tightcenter}
    \schemestart
        \chemnameinit{}
        \chemname[4ex]{\setchemfig{atom sep=2em}\chemfig{C(-[3])(-[5])=O}}{Grupo carbonila}
        \qquad\qquad
        \chemnameinit{}
        \chemname[2.5ex]{\setchemfig{atom sep=2em}\chemfig{-C(=[1]O)(-[7]OH)}}{Grupo carboxila}
        \qquad\qquad
        \chemnameinit{}
        \chemname[1.9ex]{\setchemfig{atom sep=2em}\chemfig{-C(=[1]O)(-[7]O^{-})}}{Íon carboxilato}
        \qquad\qquad
        \chemname[2.3ex]{\setchemfig{atom sep=2em}\chemfig{R-C(=[2]O)-}}{Grupo acila}
    \schemestop
\end{tightcenter}

O composto carbonilado mais simples é o formaldeído, \ch{H2CO}, um gás muito solúvel em água. Os compostos carbonilados de baixo peso molecular são líquidos. Os ésteres têm odor de frutas, e as cetonas cíclicas grandes têm odor muito agradável, sendo, habitualmente, componentes de perfumes de alto preço. Os ácidos carboxílicos, anidridos e halogenetos de acila têm, geralmente, odores desagradáveis.

A ligação \ch{C=O} em compostos das classes apresentadas na Figura \ref{figurar_8_1} é, do mesmo modo que a ligação \ch{C=C} de olefinas, composta por uma ligação $\sigma$ e uma ligação $\pi$. O grupo carbonila pode ser representado como sendo formado pelo entrosamento de um orbital $sp^2$ do carbono com um orbital $2p_x$ do oxigênio para formar a ligação $\sigma$ e pelo entrosamento lateral dos orbitais $2p_z$ do carbono e do oxigênio para formar uma ligação $\pi$ (Figura \ref{figura_8_2}). Há um par de elétrons livres no orbital $2s$ e outro par no orbital $2p_y$ do oxigênio.

\begin{figure}[H]
    \centering
    \begin{tikzpicture}[help lines/.style={thin,draw=black!50}]
        %\draw[help lines] (0,0) grid (7,7);
        \path (2,3.2) node (m) {C} ($ (m) + (350:2) $) node (o) {O};
        \path ($ (m) + (90:0.9) $) node (ma) {} ($ (o) + (90:0.9) $) node (oa) {};
        \draw[] (m) -- ++(270:2);
        \draw[] (m) -- ++(90:1);
        \draw[] (o) -- ++(270:2);
        \draw[] (m) -- ++(170:1);
        \draw[dashed] (m) -- (o);
        \draw[] (o) -- ++(90:1);
        \draw[] (m) -- ++(30:2);
        \draw[] (o) -- ++(30:2);
        \draw[] (m) -- ++(210:2);
        \draw[] (o) -- ++(210:2) node[align=center] {\enspace\enspace\footnotesize{y}};

        \filldraw[fill=white,thick] ($ (m) + (90:0.8) $) ellipse (0.6 and 0.4);

        \draw[->,>=stealth] (ma) -- ++(90:1);

        \filldraw[fill=white,thick] ($ (m) + (270:0.8) $) ellipse (0.6 and 0.4);

        \filldraw[fill=white,thick] ($ (o) + (30:0.8) $) ellipse (0.4 and 0.6);

        \draw[->,>=stealth] (o) -- ++(350:1.5) node[align=right] {\enspace\footnotesize{x}};

        \filldraw[fill=white,thick] ($ (o) + (90:0.8) $) ellipse (0.6 and 0.4);

        \draw[->,>=stealth] (oa) -- ++(90:1) node[align=right] {\enspace\enspace\footnotesize{z}};

        \filldraw[fill=white,thick] ($ (o) + (270:0.8) $) ellipse (0.6 and 0.4);

        \node (t1) at ($ (m) + (90:1) $) {};
        \node (t2) at ($ (o) + (90:1) $) {};
        \node (b1) at ($ (m) + (270:1) $) {};
        \node (b2) at ($ (o) + (270:1) $) {};
        \draw[dashed] (t1) -- (t2);
        \draw[dashed] (b1) -- (b2);

        \filldraw[fill=white,thick] ($ (o) + (210:0.8) $) ellipse (0.4 and 0.6);
    \end{tikzpicture}
    \caption{Grupo carbonila planar com o oxigênio ($p^3$) não-hibridado. Não aparece o par de elétrons livres do orbital $2s$ do oxigênio.}
    \label{figura_8_2}
\end{figure}

A ligação \ch{C=O} é uma ligação muito forte, 176-179 kcal mol$^{-1}$, um pouco mais forte do que duas ligações \ch{C-O} (2 $\times$ 85,5 kcal mol$^{-1}$) mas, apesar disto, é uma ligação dupla muito reativa. A alta reatividade é devida à diferença de eletronegatividade entre o carbono e o oxigênio, que leva a uma contribuição importante da forma de ressonância polar na qual o oxigênio é negativo e o carbono positivo. Estudos de momento de dipolo indicam que a contribuição da forma polar pode chegar a 50 por cento:

\begin{tightcenter}
\schemestart
    \setchemfig{atom sep=2em}\chemfig[][]{C(-[3])(-[5])=\lewis{0:6:,O}}
    \qquad\arrow{<->}
    \setchemfig{atom sep=2em}\chemfig[][]{\chemabove{C}{\hspace{+5mm}\scriptstyle +}(-[3])(-[5])=\lewis{0:2:6:,\chemabove{O}{\hspace{+5mm}\scriptstyle -}}}
\schemestop
\end{tightcenter}

Em compostos nos quais o átomo ligado ao grupo carbonila tem orbitais $p$ ocupados, a ligação é complicada pela possibilidade de deslocalização adicional dos elétrons (ressonância).

\begin{tightcenter}
\schemestart
    \setchemfig{atom sep=2em}\chemfig[][]{-C(=[1]\lewis{0:2:,O})(-[7]\lewis{0:4:6:,Y})}
    \qquad\arrow{<->}
    \setchemfig{atom sep=2em}\chemfig[][]{-C(=[1]\lewis{0:2:6:,\chemabove{O}{\hspace{+5mm}\scriptstyle -}})(-[7]\lewis{0:6:,\chemabove{Y}{\hspace{+5mm}\scriptstyle +}})}
\schemestop
\end{tightcenter}

\noindent A importância da forma dipolar aumenta quando eletronegatividade de Y decresce na ordem halogênio, oxigênio, nitrogênio. Observe que, quando a contribuição da forma dipolar aumenta, o caráter de ligação dupla do grupo carbonila diminui.

Um sistema $\pi$ de quatro centros é possível quando Y é um grupo vinda.

\begin{tightcenter}
\schemestart
    \setchemfig{atom sep=2em}\chemfig[][]{-C(=[2]O)-CH=CH-}
    \arrow{<->}
    \setchemfig{atom sep=2em}\chemfig[][]{-C(-[2]O)=CH-\chemabove{C}{\scriptstyle +}H-}
\schemestop
\end{tightcenter}

Em uma estrutura deste tipo, as ligações duplas são conjugadas como as do butadieno. O leitor deve voltar a este assunto (Seções 7.12 e 7.13).

\section{ACIDEZ E BASICIDADE DE COMPOSTOS CARBONILADOS}

\noindent\textbf{O grupo carbonila como ácido de Lewis}

Há muitos compostos orgânicos nos quais um átomo em ligação múltipla pode aceitar parcialmente um par de elétrons, geralmente com o deslocamento síncrono de elétrons na ligação múltipla. (O caso no qual uma ligação múltipla \textit{doa} um par compartilhado de elétrons foi discutido na Seção 7.15.) Podemos adicionar à nossa lista anterior de ácidos de Lewis (Seção 4.1) \textit{os grupos contendo ligações múltiplas possuindo uma região de baixa densidade eletrônica}.



Observe que o átomo com o qual a base se coordena não tem um orbital vazio, porém um orbital fica disponível pelo deslocamento de elétrons. O mais importante destes ácidos de Lewis é o grupo carbonila. As reações de grupos carbonila como ácidos de Lewis, envolvendo adições à ligação \ch{C=O}, estão entre as mais importantes da química orgânica.




Uma classificação importante das reações iônicas de moléculas orgânicas é baseada na descrição do reagente "que ataca". Pela definição de Lewis, um ácido é qualquer espécie capaz de aceitar um par de elétrons e uma base é um doador de par de elétrons. Os reagentes que são ávidos por elétrons (e, portanto, ácidos de Lewis) são chamados \textit{eletrófilos} (que gostam de elétrons) e os reagentes que são doadores de elétrons (e, portanto, bases de Lewis) são chamados \textit{nucleófilos} (que gostam de núcleos). As reações iônicas são, em consequência, classificadas como \textit{eletrofílicas} ou \textit{nucleofílicas}, de acordo com o tipo de reagente envolvido. Por exemplo, a equação anterior descreve uma \textit{adição nucleofílica} ao grupo carbonila.

Estas palavras, um tanto elegantes, descreveriam uma situação nova? A resposta é sim. A \textit{basicidade} é medida em termos da posição de \textit{equilíbrio} entre um doador de elétrons e um ácido (geralmente o próton). A \textit{nucleofilia}, por outro lado, é medida em termos da \textit{velocidade} da reação do nucleófilo com o substrato. (A palavra \textit{substrato} é um termo geral para a molécula que está participando da reação em questão.) Considere os exemplos do íon metóxido, \ch{CH3O^{-}}, e do metil-mercapteto, \ch{CH3S^{-}}. Metóxido é a base mais forte, mas o mercapteto é o nucleófilo mais potente. Falaremos mais sobre basicidade e nucleofilia nos próximos capítulos.

\noindent\textit{Bases}
\begin{tightcenter}
    \schemestart
        \ch{CH3O- + H2O}
        \arrow{<->>}
        \ch{CH3OH + HO-}
    \schemestop\par\bigskip
    \schemestart
        \ch{CH3S- + H2O}
        \arrow{<<->}
        \ch{CH3SH + HO-}
    \schemestop
\end{tightcenter}

\noindent\textit{Nucleófilos}

É fácil observar que os ácidos práticos e de Lewis são, ambos, catalisadores eficientes em adições nucleofílicas ao grupo carbonila.



A forma protonada da carbonila tem mais carga positiva no carbono do que a carbonila original, como mostramos. O ataque por um nucleófilo é, assim, facilitado.

\noindent\textbf{O grupo carbonila como base de Lewis}

O oxigênio do grupo carbonila possui dois pares de elétrons livres e, além disto, uma boa parte da densidade eletrônica dos elétrons ligantes $\sigma$ e $\pi$. Em consequência, o oxigênio funciona como base de Lewis, embora seja $10^{12}$ a $10^{18}$ vezes menos básico do que o nitrogênio de uma amina.


Deste modo, enquanto a maior parte dos compostos carbonilados, exceto os de peso molecular muito baixo, é insolúvel em água, eles se dissolvem em \ch{H2SO4} concentrado, com a formação de \ch{R2C=OH+}.

\noindent\textbf{Compostos carbonilados como ácidos próticos}

Vimos que um grupo carbonila pode atuar como ácido ou base de Lewis. Os compostos carbonilados que contêm um hidrogênio ligado a um dos átomos adjacentes ao grupo carbonila também podem funcionar como ácidos próticos.



Os ácidos carboxílicos estão dentre os ácidos orgânicos mais fortes. O ácido acético (\ch{CH3COOH}) é $10^{11}$ vezes mais ácido do que o etanol. Embora a forma de ressonância sem cargas seja da maior importância para a maioria dos compostos orgânicos, as formas com cargas algumas vezes contribuem significativamente para a estrutura, como no caso dos ácidos carboxílicos.

A forma à esquerda é consideravelmente mais importante do que a da direita.

Quando o ácido carboxílico se ioniza, o ânion resultante tem duas formas de ressonância, as quais são equivalentes.


Como discutimos nas \textit{regras para o uso do método de ressonância} (Capítulo 7), a ressonância será tanto mais efetiva quanto mais próximas forem as energias das duas formas de ressonância. Ela estabiliza o ácido carboxílico, porém estabiliza muito mais o ânion.

A estabilização extra do produto no lado direito da equação:



\noindent faz com que o equilíbrio se desloque para a direita mais do que aconteceria se não houvesse ressonância. A grande acidez dos ácidos carboxílicos comparada com os álcoois pode ser atribuída:

\noindent principalmente a este efeito de ressonância.
\noindent Entre os compostos do tipo:

\noindent os ácidos carboxílicos são os mais ácidos. As amidas, \ch{RCONH2}, são fracamente ácidas, compostos essencialmente neutros, como deveríamos esperar das eletronegatividades relativas do nitrogênio e do oxigênio. No extremo oposto, quando Y é um átomo de carbono em compostos tais como nas cetonas, não há um sistema deslocalizado de três átomos no composto inicial. Entretanto, há no ânion a possibilidade de deslocalização eletrônica mesmo que as duas formas de ressonância não sejam de igual energia.



Assim, um hidrogênio no carbono adjacente ao grupo carbonila de uma cetona é muito mais ácido (pKa $\sim$ 19) que um hidrogênio de um alcano (pKa $\sim$ 40), mas é bem menos ácido do que o próton de um ácido carboxílico (pKa $\sim$ 5). Como uma cetona é menos ácida do que a água (pKa = 16), nós a consideramos, geralmente, como sendo neutra, porém esta pequena acidez é importante nas propriedades químicas deste tipo de composto (Seções 18.12-18.14).

Quando um hidrogênio está ligado a um carbono colocado entre dois grupos carbonila, o composto é consideravelmente mais ácido (pKa $\sim$ 10) que um composto carbonilado normal, graças à maior estabilidade conferida ao ânion pela ressonância devida ao grupo carbonila adicional,



\section{TAUTOMERIA CETO-ENÓLICA}

A protonação de qualquer um dos oxigênios em um ânion carboxilato produz a mesma estrutura, o ácido carboxílico. No caso do íon enolato, a protonação pode ocorrer no carbono, produzindo uma cetona, ou no oxigênio, produzindo um \textit{enol} (\textit{en-}, \ch{C=C}; \textit{-ol}, OH), o qual é equivalente a um álcool vinílico. Na maioria das vezes, uma cetona está em equilíbrio com o enol correspondente.



\noindent Interconversão ceto-enólica está sujeita à catálise por ácido ou base ou, mais precisamente, uma combinação. O processo pode ocorrer por etapas ou de maneira concertada. A base pode remover o próton do carbono, formando um íon enolato, que, por sua vez, é protonado no oxigênio, formando o enol.

\noindent \textit{Catalisador básico}

\noindent Ou então a protonação pode ocorrer no oxigênio, formando o ácido conjugado da cetona, seguindo-se a abstração do próton do carbono pela base.

\noindent\textit{Catalisador ácido}

\noindent Ou, ainda, a protonação e a desprotonação podem ocorrer simultaneamente, e é desta maneira que o processo é geralmente escrito:

\noindent Exceto na ausência completa de ácido ou base, as formas cetônica e enólica se interconvertem rapidamente, existindo em um equilíbrio móvel cuja posição depende dos detalhes estruturais do composto e das condições ambientes (solvente, temperatura, concentração, etc.).

Observe que as formas ceto e enol de um composto são moléculas distintas. Elas não devem ser confundidas com formas de ressonância, que não têm existência independente. Um nome especial foi criado para descrever o relacionamento entre as formas ceto e enol. Elas são chamadas \textit{tautômeras} e sua interconversão é chamada de \textit{tautomeria}. Os tautômeros são interconvertidos fácil e rapidamente sob condições normais, As forma, ceto e enol da ciclo-hexanona:



\noindent que diferem entre si apenas na posição relativa do átomo de hidrogênio, são tautômeras enquanto que metileno-ciclo-hexano e 1-metil-ciclo-hexeno:



\noindent que existem independentemente sob condições normais, são isômeros estruturais e não tautômeros. A diferença entre isômeros estruturais e tautômeros é de grau, não de qualidade.


Para aldeídos e cetonas simples, o equilíbrio está muito deslocado para o lado da forma ceto. Por outro lado, compostos 1,3-dicarbonilados contêm uma proporção alta da forma enólica.

\begin{tightcenter}
    \chemnameinit{}
    \chemname{\setchemfig{atom sep=2em}\chemfig{CH_3-C(=[2]O)-CH_3}}{Acetona}
    \qquad
    \chemname{\setchemfig{atom sep=2em}\chemfig{CH_3-C(=[2]O)-CH_2-C(=[2]O)-O-CH_2-CH_3}}{Aceto-acetato de etila}
    \qquad
    \chemname{\setchemfig{atom sep=2em}\chemfig{CH_3-C(=[2]O)-CH_2-C(=[2]O)-CH_3}}{Acetil-acetona}
\end{tightcenter}

\noindent A existência da forma enólica em proporções apreciáveis em um composto 1,3-dicarbonilado é o resultado da estabilização por conjugação da ligação dupla carbono-carbono do enol com o segundo grupo carbonila, acrescida, em certos casos, de estabilização adicional por formação de ligação hidrogênio interna. As formas enólicas da maioria dos compostos 1,3-dicarbonilados existem em estruturas anelares, formadas por ligações hidrogênio internas, hamadas, geralmente, \textit{anéis quelato} (do grego \textit{quela}, garra). Excepcionalmente estáveis e geralmente cristalinos, sais quelatados formam-se entre compostos $\beta$-dicarbonilados e vários íons metálicos. É necessário que o íon metálico possua orbitais não-ocupados, de baixa energia, disponíveis para coordenação.

\begin{tightcenter}
    \chemnameinit{}
    \chemname[2ex]{\setchemfig{atom sep=2em}\chemfig{[:30]C(-[5]H_3C)*6(-\chembelow{C}{H}=C(-CH_3)-O-H-[,,,,dash pattern=on 2pt off 2pt]O=)}}{Enol da acetil-acetona}
    \qquad\qquad
    \chemname{\setchemfig{atom sep=2em}\chemfig{C*6(-A-B?*6(-Z-X-Y-W-T?)-C-D-E-)}}{Acetil-acetonato de cobre}
\end{tightcenter}

\par\bigskip
\noindent SEPARAÇÃO DE TAUTÔMEROS. \textit{Muito tempo antes de se entender a natureza da tautomerização, os químicos já estavam envolvidos pelo desafio de isolar o mais simples dos enóis, o álcool vinílico. Até hoje isto não pôde ser conseguido. O equilíbrio favorece em muito o aldeído.}

\textit{O aceto-acetato de etila foi sempre considerado como o caso clássico da tautomeria ceto-enólica. Em 1911, o químico alemão Knorr conseguiu separar e isolar ambas as a formas cetônica e enólica do acero-acetato de etila: a forma cetônica cristaliza-se de soluções a -78$\degree$C. Quando se passa cloreto de hidrogênio seco em uma solução do sal de sódio do aceto-acetato de etila a -78$\degree$C, obtém-se a forma enólica na forma de um sólido vítreo. Entretanto, ao voltar à temperatura ambiente, a mistura dos dois tautômeros é novamente obtida em equilíbrio, mesmo no material assim isolado.}

\section{ALDEÍDOS E CETONAS}

Se um carbono da carbonila está ligado a dois hidrogênios ou a um hidrogênio e um grupo alquila, o composto resultante é um \textit{aldeído}. Em uma \textit{cetona}, a carbonila está ligada a dois grupos alquila.

\begin{tightcenter}
    \setchemfig{atom sep=2em}\chemfig{C(-[5]H)(-[7]H)(=[2]O)}
    \qquad
    \setchemfig{atom sep=2em}\chemfig{C(-[5]R)(-[7]H)(=[2]O)}
    \qquad
    \setchemfig{atom sep=2em}\chemfig{C(-[5]R)(-[7]R)(=[2]O)}
    \qquad
    \setchemfig{atom sep=2em}\chemfig{C(-[5]R)(-[7])(=[2]O)}
\end{tightcenter}


Os aldeídos são obtidos pela oxidação dos álcoois primários, enquanto que as cetonas são obtidas pela oxidação de álcoois secundários. Os aldeídos estão sujeitos a oxidação subsequente, produzindo ácidos carboxílicos. Na química orgânica, a \textit{oxidação} geralmente envolve a remoção de hidrogênio associada frequentemente à adição de oxigênio ou algum outro elemento eletronegativo. A \textit{redução} quase sempre envolve a adição de hidrogênio ou a remoção de um elemento eletronegativo como o oxigênio.


Os espectros de RMN de aldeídos e cetonas fornecem informações estruturais de importância, embora o grupo carbonila não produza sinais no espectro de RMN. Os núcleos de hidrogênio vizinhos a um grupo carbonila apresentam desblindagem significativa, resultante do caráter elétron-atraente do carbono da carbonila carregado positivamente. Os hidrogênios de grupos metila adjacentes ao grupo carbonila em aldeídos e cetonas aparecem perto de $\delta$ 2, e os hidrogênios de grupos metileno (\chemfig{—CH2—CO-}) aparecem em campo ligeiramente mais baixo, perto de $\delta$ 2,5 (Figuras 8.3 e 8.4). Uma desblindagem muito forte é observada no caso do hidrogênio dos aldeídos, ligados diretamente ao carbono da carbonila. Tais hidrogênios aparecem perto de $\delta$ 10 (Figura 8.3). Como praticamente nenhum outro tipo de núcleo de hidrogênio aparece a campo tão baixo, a espectroscopia de RMN nos proporciona um método excelente para a identificação de aldeídos.

\begin{figure}[H]
    \centering
    \adjustbox{margin=1em,width=\textwidth,set height=4cm,set depth=4cm,frame,center}{Dummy}
    \caption{Espectro de RMN do acetaldeído em solução de \ch{CDCl3} (deslocamento de 2,0 ppm). Observe a ressonância a baixo campo característica do quarteto aldeído a $\delta$ 9,80.}
    \label{fig8_3}
\end{figure}

\par\bigskip
\noindent OBSERVAÇÃO TÉCNICA SOBRE ESPECTROS DE RMN. \textit{Os papéis registradores mais comumente usados para espectros de RMN cobrem uma faixa de 500 Hz ($\delta$ 0 a 8,33). Uma recalibração do ponto de partida (usualmente $\delta$ O) é necessária para colocar no papel qualquer pico que sai da faixa padrão. A recalibração não se aplica à curva inferior na Figura \ref{figura_8_3}, que cobre toda a extensão do gráfico, mas sim à porção superior do traço, começando na margem esquerda. Para se determinar o deslocamento químico dos prótons mostrados em um traçado deslocado deste tipo, o deslocamento obtido da recalibração em ppm é simplesmente adicionado aos valores mostrados na parte inferior do gráfico. Por exemplo, na Figura \ref{fig8_3} o próton aldeídico está aparentemente centrado em $\delta$ 7,80. O deslocamento dado pela recalibração foi de $\delta$ 2,0 ppm. O deslocamento químico do próton do aldeído é então 7,80 + 2,0 ou $\delta$ 9,80. O quarteto maior é uma expansão do multiplete a $\delta$ 9,80.}
\par\bigskip

\begin{figure}[H]
    \centering
    \adjustbox{margin=1em,width=\textwidth,set height=4cm,set depth=4cm,frame,center}{Dummy}
    \caption{Espectro do RMN da metil-etil-cetona.}
    \label{fig8_4}
\end{figure}

\section{NOMENCLATURA DE ALDEÍDOS E CETONAS}

Os nomes UIQPA de aldeídos são formados substituindo-se a terminação \textit{-o} dos hidrocarbonetos pelo sufixo \textit{-al}. O carbono aldeídico é sempre o de número 1. Os nomes vulgares são derivados dos nomes dos ácidos carboxílicos correspondentes (Seção 8.8) substituindo-se a terminação \textit{-ico} ou \textit{-bico} por \textit{-aldeído}. Nomes vulgares são quase sempre usados para aldeídos com até cinco carbonos.

\begin{tightcenter}
    \chemnameinit{}
    \chemname{\chemname{\ch{CH3CH2COOH}}{\footnotesize{Ácido propiônico}}}{\footnotesize{propanóico}}
    \qquad
    \chemnameinit{}
    \chemname{\chemname{\ch{CH3CH2CHO}}{\footnotesize{Propionaldeído}}}{\footnotesize{propanal}}
    \qquad
    \chemnameinit{}
    \chemname{\chemname{\ch{CH3CH(CH3)CH2CHO}}{\footnotesize{Isovaleraldeído}}}{\footnotesize{3-metil-butanal}}
\end{tightcenter}

Nos nomes UIQPA, as posições dos substituintes na estrutura principal são designadas por números. Na nomenclatura vulgar indicam-se, normalmente, as posições dos substituintes por intermédio de letras gregas, começando com a no carbono \textit{adjacente} ao grupo funcional principal, seguido por beta ($\beta$), gama ($\gamma$), delta ($\delta$), epsilon ($\epsilon$) etc.; ômega ($\omega$) é algumas vezes usado para designar o último carbono na cadeia, seja qual for o seu número.

\begin{tightcenter}
    \chemnameinit{}
    % need to find away to add greek symbols here
    \setchemfig{atom sep=2em}\chemfig{C-C-\chemabove{C}{\footnotesize BETA}-\chemabove{C}{\footnotesize ALPHA}-X}
    \qquad
    \chemname{\ch{CH3CH2CHBrCHO}}{\footnotesize{$\alpha$-bromo-butiraldeído}}
\end{tightcenter}

Os nomes UIQPA de cetonas derivam-se dos nomes dos hidrocarbonetos adicionando-se o sufixo \textit{-ona} ao nome da maior cadeia de carbono que contenha o grupo carbonila. A locação do grupo carbonila é dada pelo menor número possível. O número é colocado imediatamente antes do nome principal ou, quando usado juntamente com outros sufixos, imediatamente antes do sufixo \textit{-ona}. Em sistemas complicados, quando além da cetona existem outros grupos funcionais presentes, costuma-se usar o prefixo \textit{-oxo} acompanhado do número apropriado. Os nomes vulgares das cetonas são formados nomeando-se os grupos hidrocarbônicos ligados à carbonila e adicionando-se a palavra cetona. Às vezes as cetonas são conhecidas pelos nomes triviais que refletem sua origem.

\begin{tightcenter}
    \chemnameinit{}
    \chemname{\chemname{\setchemfig{atom sep=2em}\chemfig{CH_3-CH_2-C(=[2]O)-CH_3}}{\footnotesize Metil-etil-cetona}}{\footnotesize (2-butanona)}
    \qquad
    \chemname{\chemname{\setchemfig{atom sep=2em}\chemfig{CH_2=CH-C(=[2]O)-CH_3}}{\footnotesize Metil-vinil-cetona}}{\footnotesize (3-buteno-2-ona)}
    \qquad
    \chemname{\chemname{\setchemfig{atom sep=2em}\chemfig{C(-[3]CH_3)(-[5]CH_3)=CH-C(=[2]O)-CH_3}}{\footnotesize Óxido de mesitila}}{\footnotesize (4-metil-3-peteno-2-ona)}
    \qquad
    \chemnameinit{}
    \chemname{\setchemfig{atom sep=2em}\chemfig{Cl-CH_2-C(=[2]O)-CH_3}}{\footnotesize Cloro-acetona}
    \qquad
    \chemname{\setchemfig{atom sep=2em}\chemfig{CH_3-CH_2-C(=[2]O)-CH(-[2]Cl)-CH_3}}{\footnotesize 2-cloro-3-pentanona}
    \qquad
    \chemname{\chemname{\setchemfig{atom sep=2em}\chemfig{CH_3-C(=[2]O)-CH_2-CH_2-C(=[2]O)-OH}}{\footnotesize 4-oxo-pentanóico}}{\footnotesize (ácido levulínico)}
\end{tightcenter}

\section{PROPRIEDADES DE ALDEÍDOS E CETONAS}

As propriedades físicas de um certo número de importantes aldeídos e cetonas encontram-se na Tabela \ref{tab8_1}. A maioria dos aldeídos e cetonas simples têm momentos de dipolo em torno de 2,7 D. Associação por interação dos dipolos explica a magnitude dos pontos de ebulição destes compostos. Os pontos de ebulição são intermediários entre os dos hidrocarbonetos e os dos álcoois de pesos moleculares semelhantes. Os aldeídos de baixo peso molecular têm odores desagradáveis, enquanto que as cetonas, largamente distribuídas na natureza, geralmente têm odores muito agradáveis. Várias cetonas naturais e sintéticas são usadas em perfumes e aromatizantes. Outras são importantes substâncias medicinais ou componentes de sistemas biológicos.

\begin{table}[H]
    \centering
    \caption{Propriedades físicas de aldeídos e cetonas.}
    \label{tab8_1}
    \begin{tabular}{cccc}
        \toprule
        Nome & Estrutura & PF ($\degree$C) & PE ($\degree$C) \\
        \midrule
        \multicolumn{1}{l}{\underline{Aldeídos}} & & & \\
        Formaldeído & \ch{CH2O} & -117 & -19 \\
        Acetaldeído & \ch{CH3CHO} & -123 & 21 \\
        Propionaldeído & \ch{CH3CH2CHO} & -81 & 48 \\
        Butiraldeído & \ch{CH3(CH2)2CHO} & -97 & 75 \\
        Valeraldeído & \ch{CH3(CH2)3CHO} & -51 & 93 \\
        Caproaldeído & \ch{CH3(CH2)4CHO} & -56 & 129 \\
        Acroleína & \ch{CH2=CHCHO} & -87 & 53 \\
        \underline{Cetonas} & & & \\
        Acetona & \ch{CH3COCH3} & -95 & 56 \\
        Metil-etil-cetona & \ch{CH3COCH2CH3} & -86 & 80 \\
        Dietil-cetona & \ch{CH3CH2COCH2CH3} & -39 & 102 \\
        3-Hexanona & \ch{CH3CH2COCH2CH2CH3} & & 124 \\
        $t$-Butil-metil-cetona & \ch{(CH3)3CCOCH3} & -53 & 106 \\[2ex]
        Ciclo-pentanona & \setchemfig{atom sep=2em}\chemfig{*5(--(=O)---)} & -58 & 106 \\[2ex]
        Ciclo-hexanona & \setchemfig{atom sep=2em}\chemfig{[:-30]*6(---(=O)---)} & -31 & 156 \\[2ex]
        Metil-vinil-cetona & \ch{CH3COCH=CH2} & & 80 \\
        Óxido de mesitila & \ch{CH3COCH=C(CH3)3} & -53 & 130 \\
        Biacetila & \ch{CH3COCOCH3} & -2 & 88 \\
        Acetil-acetona & \ch{CH3COCH2COCH3} & -23 & 138 \\
        \bottomrule
    \end{tabular}
\end{table}

\par\bigskip
\noindent USOS DE COMPOSTOS CARBONIZADOS. \emph{A biacetila é o principal ingrediente aromatizante da margarina. A muscona, extraída das glândulas de secreção externa do almíscar macho, é um ingrediente importante dos perfumes. A cânfora, obtida da madeira da canforeira (Cinnamomum camphora), nativa do Vietnã, Japão e China, foi utilizada para fins medicinais por muitos séculos e ainda o é, embora aparentemente não tenha valor terapêutico maior. A testosterona, o hormônio sexual responsável pelo desenvolvimento das características masculinas do homem e outros mamíferos, é obtida comercialmente pela extração de testículos de touro.}

\emph{Formol, uma solução aquosa a 37\% de formaldeído, é um preservativo bem conhecido de espécimens biológicos. Mais de três milhões de toneladas de formol são fabricados a cada ano nos Estados Unidos da América, principalmente para o uso da indústria de plásticos e resinas. O acetaldeído é também um intermediário químico importante, sendo usado na síntese industrial de compostos orgânicos.}

\emph{A acetona é, sem dúvida, a mais importante das cetonas. Mais de um milhão de toneladas são consumidos anualmente nos Estados Unidos da América. É um excelente solvente para compostos orgânicos. A metil-etil-cetona é também muito usada como solvente industrial.}
\par\bigskip

\section{ESTRUTURA DOS ÁCIDOS CARBOXÍLICOS}

O grupo carboxila, —COOH, formalmente uma combinação de uma carbonila e um grupo hidroxila, representa o estado de oxidação de um carbono primário imediatamente acima do de um aldeído. O resultado mais visível da combinação dos dois grupos é um grande aumento na acidez da hidroxila. Embora a ionização esteja longe de ser completa, o grupo carboxila ioniza-se suficientemente em água a ponto de tornar vermelho o papel de tornas-sol. O porquê da acidez da carboxila foi discutido na Seção 8.2. Os ácidos formam ligações hidrogênio ainda mais fortes do que os álcoois, porque as ligações O—H estão mais fortemente polarizadas e a ponte de hidrogênio pode se ligar ao oxigênio mais negativo da carbonila ao invés de se ligar a um oxigênio de outra hidroxila. Os ácidos carboxílicos nos estados sólido e líquido, e mesmo, em certa proporção, no estado de vapor, existem na forma de dímeros cíclicos.

A densidade eletrônica no hidrogênio da carboxila é muito baixa e o próton aparece, como seria de esperar, em campo anormalmente baixo na RMN, na região $\delta$ 11-13. No espectro de RMN do ácido acético em solução de clorofórmio deuterado, o singlete do grupamento metila é observado em $\delta$ 2,10 e o singlete do próton ácido em $\delta$ 11,37. O espectro de RMN de um ácido carboxílico comum, simples, é apresentado na Figura 8.5.

\section{NOMENCLATURA DOS ÁCIDOS CARBOXÍLICOS}

Os nomes vulgares de alguns ácidos carboxílicos importantes são apresentados na Tabela \ref{tab8_2}. Como estes nomes e alguns de seus derivados serão constantemente usados, o leitor deve tentar memorizá-los. Acima de seis carbonos, apenas os ácidos com número par de átomos de carbono são importantes, visto que eles são os únicos que ocorrem em quantidades significantes em fontes naturais. Como veremos mais tarde (Capítulo 27), a natureza constrói estes ácidos de cadeias longas pela combinação de resíduos de ácido acético.

\begin{figure}[H]
    \centering
    \adjustbox{margin=1em,width=\textwidth,set height=4cm,set depth=4cm,frame,center}{Dummy}
    \caption{Espectro de RMN de um ácido carboxílico simples.}
    \label{fig8_5}
\end{figure}

É igualmente importante conhecer-se os nomes vulgares dos ácidos dicarboxílicos simples apresentados na Tabela \ref{tab8_2}.

\par\bigskip
\noindent NOTA HISTÓRICA. \emph{Muitos dos ácidos carboxílicos comuns foram inicialmente isolados de fontes naturais, especialmente de gorduras, daí serem frequentemente chamados de "ácidos graxos". Os nomes vulgares, usados antes de serem conhecidas suas estruturas químicas, referem-se à origem natural e não às estruturas. Assim, a irritação causada por uma mordida de formiga é devida em parte ao ácido fórmico (do latim formica, formiga); o principal ingrediente do vinagre é o ácido acético (do latim acetum, vinagre); o ácido butírico (do latim butirum, manteiga) dá o odor característico da manteiga rançosa; o ácido valérico foi isolado da raiz da valeriana (do latim valere, ser forte); e os ácidos capróico, caprílico e cáprico são os responsáveis pelo odor tão pouco social das cabras.}
\par\bigskip

Os ácidos que contêm um grupo isopropila na extremidade de uma cadeia de hidrocarboneto normal podem ser nomeados pela adição do prefixo \textit{-iso} ao nome vulgar do ácido com o mesmo número total de carbonos.

\begin{tightcenter}
\chemnameinit{}
    \chemname{\ch{CH3CH2CH2CH2COOH}}{\footnotesize Ácido valérico}
    \qquad
    \chemname{\setchemfig{atom sep=2em}\chemfig{CH_3-CH(-[2]CH_3)-CH_2-COOH}}{\footnotesize Ácido isovalérico}
\end{tightcenter}

No sistema UIQPA a terminação \textit{-o} da maior cadeia hidrocarbônica que carboxila é substituída por \textit{-óico}. O carbono da carboxila é sempre o de número 1. Observe que, na nomenclatura vulgar com o uso de letras gregas, o carbono \textit{adjacente} à carboxila é chamado de $\alpha$, e não o carbono da carboxila. O nome sistemático raramente é um ácido que possua menos de cinco átomos de carbono.

Quando um grupo carboxila aparece como substituinte de uma cadeia principal, pode-se usar o prefixo \textit{carboxi-} e a numeração adequada. Uma outra maneira é usar o nome do hidrocarboneto principal antecedido da palavra \textit{ácido} e seguido da palavra \textit{carboxílico}.

\begin{table}[H]
    \centering
    \caption{Ácidos carboxílicos..}
    \label{tab8_2}
    \begin{tabular}{ccccc}
        \toprule
        Ácido & Estrutura & PF ($\degree$C) & PE ($\degree$C) & pKa \\
        \midrule
        \multicolumn{1}{l}{\underline{Ácidos monocarboxílicos}} & & & & \\
        Fórmico & \ch{HCOOH} & 8 & 100,5 & 3,77 \\
        Acético & \ch{CH3COOH} & 17 & 118 & 4,76 \\
        Propiônico & \ch{CH3CH2COOH} & -22 & 141 & 4,88 \\
        Butírico & \ch{CH3CH2CH2COOH} & -5 & 163 & 4,82 \\
        Valérico & \ch{CH3CH2CH2CH2COOH} & -35 & 187 & 4,81 \\
        Capróico (hexanóico) & \ch{CH3(CH2)4OOH} & -2 & 163 & 4,85 \\
        Caprílico (octanóico) & \ch{CH3(CH2)6COOH} & 16 & 237 & 4,85 \\
        Cáprico (decanóico) & \ch{CH3(CH2)8COOH} & 31 & 269 & - \\
        Láurico (dodecanóico) & \ch{CH3(CH2)10COOH} & 43 & - & - \\
        Mirístico (tetradecanóico) & \ch{CH3(CH2)12COOH} & 54 & - & - \\
        Palmítico (hexadecanóico) & \ch{CH3(CH2)14COOH} & 64 & - & - \\
        Esteárico (octadecanóico) & \ch{CH3(CH2)16COOH} & 70 & - & - \\
        Glicólico & \ch{HOCH2COOH} & 79 & - & 3,83 \\
        Láctico & \ch{CH3CHOHCOOH} & 18 & - & 3,87 \\
        Acrílico & \ch{CH2=CHCOOH} & 13 & 141 & 4,26 \\
        \multicolumn{1}{l}{\underline{Ácidos dicarboxílicos}} & & & & \\
        Oxálico & \ch{HOOC-COOH} & 190 & - & 1,46 \\
        Malônico & \ch{HOOC-CH2-COOH} & 135 & - & 2,80 \\
        Succínico & \ch{HOOC-(CH2)2-COOH} & 187 & 235 & 4,17 \\
        Glutárico & \ch{HOOC-(CH2)3-COOH} & 98 & 303 & - \\
        Adípico (hexanodióico) & \ch{HOOC-(CH2)4-COOH} & 152 & 340 & - \\
        Pimélico (heptanodióico) & \ch{HOOC-(CH2)5-COOH} & 105 & - & - \\
        Maléico & \ch{HOOC-CH=CH-COOH} (\textit{cis}) & 131 & - & - \\
        Fumárico & \ch{HOOC-CH=CH-COOH} (\textit{trans}) & 287 & - & - \\
        \bottomrule
    \end{tabular}
\end{table}

Os nomes dos ânions carboxilato (\ch{RCOO-}) são formados a partir do nome vulgar dos ácidos, perdendo-se a palavra \textit{ácido}, e o sufixo -\textit{ico} e colocando-se o sufixo -\textit{ato}. Os sais de ácidos carboxílicos são nomeados escrevendo-se primeiro o nome do ânion carboxilato e depois o do cátion.

\begin{tightcenter}
    \chemname{\ch{CH3COO^-NH4^+}}{\footnotesize Acetato de amônio}
    \qquad\qquad\qquad
    \chemname{}{\footnotesize Ciclo-hexano-carboxilato de sódio}
    \qquad\qquad\qquad
    \chemname{\chemname{\ch{HOOCCH2CH2COO^-Na^+}}{\footnotesize Hidrógeno-succinato de sódio}}{\footnotesize (succinato ácido de sódio)}
\end{tightcenter}

\section{PROPRIEDADES DOS ÁCIDOS CARBOXÍLICOS}

Os dados de ponto de fusão e de ebulição de alguns ácidos carboxílicos são apresentados rio Tabela \ref{tab8_2}. Estes dados mostram que estas substâncias têm pontos de fusão e de ebulição muito mais altos do que os de outras classes de compostos com peso molecular semelhante. Isto é atribuído à forte associação intermolecular através de ligações hidrogênio (seções 4.18 e 8.7). A solubilidade dos ácidos em água é mais ou menos semelhante à dos álcoois, animas e outros compostos que são solvatados pela água através da formação de ligações hidrogênio. Os ácidos fórmico, acético, propiônico e butírico são completamente miscíveis em água.

\par\bigskip
\noindent ODOR DOS ÁCIDOS CARBOXÍLICOS. \emph{Os ácidos fórmico e acético têm cheiro agudo, irritante e paladar azedo, ácido. Os ácidos de quatro a oito carbonos têm odores excepcionalmente desagradáveis. Por estranho que pareça, entretanto, em pequenas concentrações eles são responsáveis por muitas fragrâncias deliciosas; o queijo Roquefort não seria o mesmo sem o ácido valérico.}

\emph{A grande sensibilidade olfativa dos cães é bem conhecida. Um cão pode diferenciar uma pessoa da outra porque pode detectar a composição aproximada da mistura de ácidos carboxílicos de baixo peso molecular que são um produto do metabolismo do indivíduo e que estão sempre presentes em quantidades traço na pele. O metabolismo de cada pessoa é um pouco diferente e a composição dos ácidos graxos na pele é, portanto, diferente. Os seres humanos, com seu sentido acurado de visão, tendem a se identificar pela visão. Já um cão possui a visão relativamente pouco desenvolvida, mas, em compensação, o senso de olfato altamente desenvolvido, o qual é usado para o mesmo fim.}
\par\bigskip

Os ácidos carboxílicos formam carboxilatos de sódio solúveis por reação com hidróxido de sódio aquoso ou bicarbonato de sódio:

\begin{tightcenter}
    \ch{RCOOH + NaHCO3 -> RCOO^-Na^+ + CO2 + H2O}
\end{tightcenter}

\noindent Os sais de sódio dos ácidos que contêm 12 ou mais carbonos são pouco solúveis em água, sendo muito utilizados por isto como sabões (Seção 8.12). Os sais dos ácidos carboxílicos têm as propriedades que seriam de se esperar: altos pontos de fusão e baixa solubilidade em solventes orgânicos.

\par\bigskip
\noindent USOS DOS ÁCIDOS CARBOXÍLICOS. \emph{Todos os ácidos apresentados na Tabela \ref{tab8_2} são disponíveis para uso comercial como intermediários sintéticos. O mais importante de todos é o ácido acético, usado como reagente e solvente, tanto em processos industriais como em laboratório. O ácido acético é comercializado na forma de ácido acético glacial (\sim 99,5 por cento), assim chamado porque em dias muito frios ele se transforma em um sólido com aspecto de gelo. Mais um milhão de toneladas são consumidas anualmente nos Estados Unidos da América.}

\emph{Muitos ácidos e seus derivados são encontrados na natureza e desempenham papéis importantes no metabolismo animal e vegetal. O ácido acético, o produto final de fermentação, é um constituinte fundamental para a biossíntese de uma grande variedade de produtos naturais, desde ácidos graxos até a borracha natural (Capítulo 27).}
\par\bigskip

\section{ÉSTERES E LACTONAS}

Em uma \textit{reação de condensação}, duas moléculas se unem, geralmente com perda de água ou outra molécula simples. O produto de uma reação de condensação entre um ácido carboxílico e um álcool é um \textit{éster}:


Antigamente a esterificação era considerada como análoga à neutralização, e os ésteres são ainda nomeados como se fossem "sais de alquila" dos ácidos carboxílicos.

\begin{tightcenter}
    \chemnameinit{}
    \chemname{\ch{CH3COOH}}{\footnotesize Ácido acético}
    \qquad
    \chemname{\ch{CH3COONa}}{\footnotesize Acetato de sódio}
    \qquad
    \chemname{\ch{CH3COOCH2CH3}}{\footnotesize Acetato de etila}
    \qquad\par\bigskip
    \chemname{\ch{CH3CH2COOCH2CH=CH2}}{\footnotesize Propionato de alila }
    \qquad
    \chemname{\ch{CH2=CHCOOC(CH3)3}}{\footnotesize Acrilato de $t$-butila}
    \qquad
    \chemname{\ch{ClCH2COOCH=CH2}}{\footnotesize Cloroacetato de vinila}
\end{tightcenter}

Algumas vezes o grupo éster pode ser considerado como um substituinte em um composto principal.

Os ésteres são algo menos enólicos do que as cetonas, mas um $\beta$-ceto-éster, tal como a 2-metoxí-carbonil-ciclo-hexanona, contém pelo menos uma pequena percentagem do enol no equilíbrio.

Um éster cíclico é conhecido como \textit{lactona}. O tamanho do anel de uma lactona é designado pela letra grega que corresponde à posição de hidroxila com a qual o ácido foi condensado.

\begin{tightcenter}
    \chemnameinit{}
    \chemname{\setchemfig{atom sep=2em}\chemfig{CH_2(-[6]OH)-CH_2-C(=[1]O)(-[7]OH)}}{\footnotesize ácido \beta-hidroxi-propiônico}
    \qquad\qquad
    \chemname{\setchemfig{atom sep=2em}\chemfig{O*4(-C(=O)-CH_2-H_2C-)}}{\footnotesize \beta-propiolactona}
\end{tightcenter}

\noindent As lactonas mais frequentemente encontradas contêm anéis de cinco e seis membros, livres de tensão, as $\gamma$- e $\delta$-lactonas. Os $\gamma$- e $\delta$-hidroxi-ácidos raramente existem como tal. Quando sintetizado, tais compostos perdem água espontaneamente e formam lactonas. As $\gamma$- e $\delta$-lactonas são muito difundidas na natureza, especialmente em plantas. Os anéis de três membros de $\alpha$-lactonas estão sujeitos à tensão muito alta e ocorrem apenas como intermediários transientes em reações.

\begin{tightcenter}
    \chemnameinit{}
    \chemname{\setchemfig{atom sep=2em}\chemfig{O*4(-C(=O)-CH_2-CH(-CH_2-CH_3)-)}}{\footnotesize \beta-valerolactona}
    \qquad\qquad
    \chemname{\setchemfig{atom sep=2em}\chemfig{O*5(-C(=O)-CH2-CH2-CH(-CH_3)-)}}{\footnotesize \gamma-valerolactona}
    \qquad\qquad
    \chemname{\setchemfig{atom sep=2em}\chemfig{CH_2*6(-CH_2-O-C(=O)-CH_2-CH_2-)}}{\footnotesize \delta-valerolactona}
\end{tightcenter}

Os ésteres são geralmente insolúveis em água e têm pontos de ebulição ligeiramente maiores do que os dos hidrocarbonetos de peso molecular semelhante (Tabela \ref{tab8_3}). Os ésteres voláteis têm aromas característicos de frutas. Os ésteres, assim como as cetonas, são responsáveis pelo sabor e fragrância de muitas frutas, flores e de aromatizantes artificiais.

\begin{table}[H]
    \centering
    \caption{Ésteres.}
    \label{tab8_3}
    \begin{tabular}{cccc}
        \toprule
        Nome & PE ($\degree$C) & Nome & PE ($\degree$C)  \\
        \midrule
        Formiato de metila & 32 & Acetato de $n$-butila & 127 \\
        Acetato de metila & 57 & Propionato de etila & 99 \\
        Acetato de etila & 77 & Butirato de $n$-butila & 166 \\
        Acetato de $n$-propila & 102 & Isovalerato de isopentila & 194 \\
        \bottomrule
    \end{tabular}
\end{table}

\par\bigskip
\noindent\emph{A delicadeza de muitos sabores e fragrâncias naturais é devida a misturas complexas. Assim, por exemplo, mais de 100 substâncias contribuem para o sabor dos morangos frescos. Os aromatizantes artificiais de baixo preço, tais como os usados caramelos e balas, consistem, normalmente, de um só composto ou são no máximo misturas muito simples. O odor e o sabor do acetato de isopentila são semelhantes aos de bananas, do}
