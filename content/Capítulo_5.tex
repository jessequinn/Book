\chapter{Espectroscopia de Ressonância Magnética Nuclear}
\section{Identificação de compostos orgânicos}
Quando um estudante se inicia na química orgânica de laboratório sua visão é muito diferente da que tem quando estuda o assunto em um livro-texto como este. O aspecto físico dos compostos orgânico tem pouca relação com as linhas, símbolos e teorias com os quais são descritos em um livro. Quando um químico precisa identificar um composto, sua tarefa pode ser muito difícil e vagarosa. É de capital importância que ele saiba que ferramentas estão disponíveis para ajudá-lo na identificação, assim como quando e de que modo aplicá-las e, também, como interpretar a informação que fornecem. 

Se duas amostras são iguais em todas as propriedades físicas e químicas, elas são o mesmo composto. Portanto, o primeiro passo na identificação de um composto de estrutura desconhecida é obter o maior número de informações possível sobre o composto. Examina-se seu estado físico e obtêm-se dados tais como pontos de ebulição e fusão, características de solubilidade, presença ou ausência de propriedades ácidas ou básicas e índice de refração. 

Qualquer uma das varias técnicas espectrais podem ser usadas na obtenção de informações estruturais sobre um composto não-identificado.

O espectro no infravermelho pode ser interpretado em termos de presença ou ausência de grupos funcionais. A partir do espectro de ressonância magnética nuclear, o numero natureza e 'ambiente' dos hidrogênios em uma molécula pode ser determinado. Destes dois espectros a estrutura do esqueleto molecular pode, frequentemente, ser deduzida. Um espectro de massas fornece dados sobre peso e fórmula moleculares e sobre os arranjos de grupos específicos na molécula. Espectros no ultravioleta, resultantes de excitações eletrônicas, são obtidos de compostos que contêm ligações insaturadas. Estas ferramentas fornecem diferentes tipos de dados que são muito efetivamente usados em conjunto e com dados adicionais físicos e químicos.

O químico orgânico tem disponível uma ampla gama de técnicas espectroscópicas, físicas e químicas que podem ser usadas para identificar e caracterizar compostos e para determinar estruturas. Quando ele houver obtido e estudado os dados físicos, químicos e espectroscópicos, saberá muito sobre o composto desconhecido e a classe de grupo funcional à qual ele pertence e poderá sugerir uma estrutura para o composto.

A segunda etapa na identificação de um composto desconhecido é a pesquisa na literatura. As propriedades do composto desconhecido podem ser comparadas com compilações existentes de dados publicados para todos os compostos já estudados. Além disto, derivados podem ser preparados por reações bem caracterizadas e suas propriedades físicas e químicas podem ser comparadas com os valores publicados correspondentes. Os dados espectrais podem ser comparados com espectro-referência, obtidos de compostos conhecidos: se duas amostras são o mesmo composto, seus espectros serão superponíveis. Se programas e acesso a computador são disponíveis, muito da pesquisa de dados publicados pode ser feita rapidamente.

Se no término da busca na literatura química um composto não pode ser identificado, o químico deve então supor que se trata de um novo composto e se defronta com a tarefa de caracterizá-lo, elucidar sua estrutura e publicar resultados. A partir dos dados disponíveis que agora possui, o químico pode estar seguro sobre a estrutura em questão. Para confirmar a estrutura postulada, o químico pode preferir sintetizar o composto a partir de materiais de estrutura conhecida através de reações bem compreendidas, e então verificar se o material sintético e o composto desconhecido são idênticos. Outro modo de ataque ao problema da postos menores que podem ser identificados inequivocamente. Pode-se frequentemente chegar a uma estrutura desconhecida a partir das estruturas de compostos menores, se as reações usadas nas etapas de degradação são bem entendidas. É importante ser lembrado que uma estrutura proposta é aceita somente se for consistente com todos os dados conhecidos sobre o composto.

Espectroscopia no infravermelho (IV), espectroscopia de ressonância magnética nuclear (RMN), espectroscopia no ultravioleta (UV) e espectrometria de massas têm tido grande impacto na química e são largamente usadas por químicos orgânicos. Estas técnicas serão discutidas neste livro-texto com enfase em suas aplicações como ferramentas para a identificação de compostos orgânicos. Espectroscopia de ressonância magnética nuclear, que é a técnica mais informativa e mais largamente usada que o químico orgânico dispõe hoje para estudar a estrutura molecular, será apresentada neste capítulo. O capítulo 9 é dedicado à espectrometria no infravermelho; espectroscopia no ultravioleta é discutida no capítulo 29 e espectrometria de masses, no capítulo 32 (Seção 32.4). Além disto, no capítulo 32 RMN é discutido em maior detalhe e são apresentadas ilustrações de como dados de diferentes técnicas espectrais são reunidos para solucionarem-se problemas de identificação.

O espectro de RMN de um composto pode ser determinado diretamente no liquido puro. Se o composto é um solido, o espectro é determinado em solução. A razão para isto é que o espectro deve ser determinado em moléculas que tenham movimento livre e deste modo equalizem suas interações com as moléculas vizinhas. Espectros do tipo aqui discutidos não podem ser determinados em sólidos. Uma grande variedade de materiais são solventes satisfatórios para a determinação de espectros de RMN. Caso se queira estudar prótons, que é a situação usual, prefere-se usar um solvente que não contenha prótons que possam interferir. Tetracloreto de carbono e solventes deuterados, como D$_{2}$O ou clorofórmio deuterado, são também comumente usados.

\section{Orientação de um núcleo em um campo magnético externo}

Todos os núcleos têm carga e massa. Aqueles que têm um número de massa ímpar ou um número atômico impar também possuem um spin; isto é, têm momento angular. Por exemplo, $^1_1$H, $^2_1$H, $^{13}_6$C, $^{14}_7$N e $^{17}_8$O possuem spins, enquanto o mesmo não ocorre com $^{12}_6$C e $^{16}_8$O. Qualquer núcleo que tenha um spin pode ser estudado por RMN, porém este capítulo será limitado à discussão do núcleo do $^1_1$H (o próton), que, na pratica, tem dado a informação mais útil.

\emph{O isótopo mais comum do carbono ($^{12}_6$C) não apresenta espectro de RMN, porque ele não tem spin nuclear. Isto é lamentável, porque o esqueleto da maioria dos compostos orgânicos é, enfim, o carbono. O próximo isótopo mais abundante do carbono ($^{13}_6$C) apresenta espectro de RMN que pode ser de uso considerável. Visto que este isótopo se constitui em apenas cera de 1 por cento do carbono natural, tem havido uma dificuldade técnica na obtenção de espectros de $^{13}$C de intensidade suficiente para ser útil. Esta dificuldade tem sido solucionada recentemente e é esperado que espectros de 13C sejam cada vez mais importantes no futuro (Ver Seção 32.3).}

Uma carga girando, como a do núcleo 11H, gera um campo magnético que tem um momento magnético ($\mu$) a ele associado. Tal núcleo pode ser considerado como a barra de um imã. Quando um campo magnético externo ($H_0$) é aplicado, o núcleo tenta alinhar seu momento magnético ao longo da direção do campo, assim como a barra do ímã de uma bussola se alinha com o campo magnético da Terra. O numero quântico de spin do núcleo é designado I; há $2I + 1$ orientações e correspondentes níveis de energia possíveis para um núcleo magnético relativo ao campo externo.

Um próton, \ch{^1_1H} tem um numero quântico de spin de $\frac{1}{2}$; assim, ele tem $2\times\frac{1}{2}+1=2$ orientações possíveis; paralelo ($\uparrow$) e antiparalelo ($\downarrow$) ao campo magnético externo. Na ausência de um campo magnético, cada próton tem a mesma energia nuclear de spin e os spins são orientados em direções caóticas. Na presença de um campo magnético, os spins do próton são alinhados paralelos ou antiparalelos ao campo e a diferença de energia entre estas duas orientações ($\Delta E$) é proporcional à intensidade do campo magnético externo, $H_0$.

\begin{equation}
    \centering
    \Delta E = k H_0
\end{equation}

\noindent onde $k = h\gamma/2\pi$; $\gamma$ é razão giromagnética (uma constante para um dado núcleo); $H_0$ é a intensidade do campo magnético externo; e h é a constante de Planck. Esta variação no espaçamento entre níveis de energia ($\Delta E$) como função da intensidade do campo aplicado ($H_0$) é mostrado na Figura 5.1 para o núcleo $^1_1H$. O nível de energia mais baixo na Figura 5.1 corresponde ao alinhamento paralelo ($\uparrow$) ao campo aplicado e o nível de energia mais elevado corresponde ao alinhamento antiparalelo ($\downarrow$).

\emph{Figura 5.1 Separação de nível de energia de spin para o núcleo do hidrogênio como função de um campo magnético externo ($H_0$).}

Para uma dada intensidade de campo, o próton pode ir de um nível de energia para o outro por absorção ou emissão de uma quantidade discreta de energia.

\begin{equation}
    \centering
    \Delta E = hv
\end{equation}

onde $v$ é a frequência de radiação que está sendo absorvida ou emitida. Combinando as equações 1 e 2, encontramos

\begin{equation}
v = \frac{\gamma}{2\pi} H_0
\end{equation}

Como pode ser visto nestas equações, quando prótons são colocados em um campo magnético que tem uma intensidade fixa, haverá uma frequência definida separando os dois níveis de energia. Na prática, um campo de cerca de 14.100 Gauss (G) requer uma frequência de 60 megahertz (MHz) de energia (da região da radiofrequência do espectro eletromagnético) para a transição entre as orientações. Em um campo de 23.500 G, 100 MHz são necessários; para um campo de 47.000 G, 200 MHz são necessários (Figura 5.1). É importante notar que 60 MHz correspondem a uma quantidade muito pequena de energia ($6\times10^{-3} cal/mol$). Isto quer dizer que o número de moléculas no estado fundamental é ligeiramente superior ao número de moléculas no estado excitado.

Para obter-se um espectro de RMN de uma amostra, esta é colocada no campo magnético no espectrômetro, e um campo de radiofrequência é aplicado, passando-se uma corrente por uma serpentina que envolve a mostra (Figure 5.2). O campo magnético ($H_0$) é aumentado aos poucos e a excitação ou a 'oscilação' dos núcleos de uma orientação para a outra é detectada como uma voltagem induzida, resultando da absorção de energia do campo de radiofrequência. Um espectro de RMN (Figure 5.3) é um gráfico de voltagem induzida contra a varredura do campo magnético. A área sob um 'pico' depende do número total de núcleos que estão 'oscilando'.

A energia absorvida por um núcleo pode ser liberada por relaxamento spin-spin, no qual a energia de spin é transferida a um núcleo vizinho, ou por relaxamento spin-rede, no qual a energia de spin é convertida em energia térmica. Os núcleos são, deste modo, excitados do estado de spin mais baixo ao mais alto por um campo de radiofrequência. Eles retornam espontaneamente ao estado de energia mais baixo para serem excitados novamente, e assim por diante.

\section{Blindagem dos núcleos de hidrogênio}
Se todos os núcleos de hidrogênio absorvessem energia em uma mesmas intensidade de campo para uma dada frequência, a espectroscopia de RMN seria apenas um método para a análise quantitativa de prótons. Na realidade, é muito mais do que isto. A intensidade de campo necessária para a absorção de energia por um determinado próton depende de seu ambiente próximo, quer dizer, da estrutura molecular. Observando-se a intensidade de campo na qual um próton absorve, é possível deduzir-se algo sobre a estrutura molecular local.

\emph{Figura 5.3 Espectro de RMN do etano.}

Se nosso átomo de hidrogênio é parte de uma molécula e a molécula é colocada em um campo magnético, o campo induz uma circulação de elétrons em torno do próton em um plano perpendicular ao campo externo. Esta carga circulante, por sua vez, gera um campo magnético induzido na região do núcleo que está geralmente oposta ao campo externo (Figura 5.4). Os elétrons que envolvem o próton são ditos blindarem o próton se o campo induzido se opõe ao campo externo, como na Figura 5.a. Neste caso, os elétrons blindam o núcleo dos efeitos do campo externo. Por outro lado, os elétrons são ditos desblindarem o núcleo se o campo induzido aumenta o campo externo. Núcleos de hidrogênio em ambientes diferentes têm densidade eletrônica maior ou menor em torno deles, e são blindados ou desblindados em proporções diferentes por elétrons que circulam. O resultado é que o próton é sujeito a um campo magnetítico liquido ou efetivo.

\begin{equation}
H_{efetivo} = H_{externo}-H_{induzido}
\end{equation}

\emph{Figura 5.4 Blindagem ($H_{induzido}$) causada por elétrons circulando em torno do núcleo em plano perpendicular ao campo externo ($H_0$).}

Quando ocorre a varredura do campo magnético, nem todos os prótons oscilam na mesma intensidade de campo. A intensidade de campo na qual oscilam depende de quanto eles são blindados, o que, por sua vez, depende dos ambientes químicos.

Da Figura 5.4 pode ser visto que a blindagem dependerá da densidade eletrônica envolvendo o núcleo do hidrogênio. Por exemplo, os prótons no iodeto de metila (\ch{CH3I}) são mais blindados do que os prótons metílicos no metanol (\ch{CH3OH}) porque a eletronegatividade do oxigênio é maior do que a do iodo. O efeito elétron-atraente do oxigênio é maior e reduz a densidade eletrônica em torno do metila mais do que iodo. Do espectro de RMN de um composto, portanto, podemos dizer o número relativo de prótons diferentes presentes pelas áreas relativas sob os picos e também podemos dizer algo sobre o ambiente de cada próton.

\section{O DESLOCAMOENTO QUÍMICO}

Para ser mais utilizável, o fenômeno da blindagem discutido na seção anterior deve ser colocado numa base quantitativa. Pelo fato da blindagem ser dependente do ambiente químico, as intensidades de campo necessárias à absorção de energia por diferentes prótons são ditas serem deslocadas quimicamente, em relação a algum padrão. Tetrametil-silano, [TMS, (CH3)4Si] é o padrão usual. Ele é dissolvido numa solução de amostra a ser estudada e usada como referência interna. Visto que todos os prótons no TMS são quimicamente equivalentes, ele tem apenas uma frequência de absorção e como o silício é mais eletropositivo do que os átomos normalmente encontrados em compostos orgânicos (C, N, O, P, S, halogênios), muito poucos núcleos de hidrogênio absorvem em frequência tão alta como no TMS. O pico do TMS é, portanto, encontrado em um lado do espectro e não misturado no meio de um espectro, o que o torna um composto de referência conveniente. 

o deslocamento químico ed um certo núcleo de hidrogênio é a diferença entre a intensidade de campo na qual o próton absorve e a intensidade de campo na qual os prótons do padrão TMS absorvem. A escala delta ($\delta$) tem sido largamento químico observado (em unidades hertz, Hz) é dividido pela frequência (em Hz) do espectrômetro usado, dando $\delta$ em partes por milhão (ppm):

\begin{equation}
\delta = \frac{\textrm{deslocamento } (Hz)\times10^6}{\textrm{frequência do espectrômetro } (Hz)}(ppm)
\end{equation}

\emph{A escala tau ($\tau$) foi muito usada anteriormente: unidades $\tau$ podem ser convertidas a unidades $\delta$ pela equação $\delta = 10 - \tau$}

O deslocamento químico de um certo próton sob as mesmas condições (mesmo solvente, temperatura, etc.) é uma constante e não depende da frequência do espectrômetro de RMN particular usado para a medida. Ao pico de TMS é atribuído o valor $\delta$ de 0,000 e picos de uma amostra em estudo são relacionados a ele e apresentados em partes por milhão. 

Foi mencionado anteriormente que o pico de RMN de um grupo metila não ocorre sempre no mesmo lugar. Isto é ilustrado na Figura 5.5 para grupos metila ligados a oxigênio, carbono e silício. Embora todos os núcleos de hidrogênio que tenham um deslocamento químico maior que 0,000 sejam menos blindados do que os prótons do TMS, na prática os termos blindado e desblindado são usados para indicar que um núcleo absorve a um valor mais baixo ou mais alto de $\gamma$, respectivamente, do que um outro núcleo. Por exemplo, os prótons metílicos do CH3O- são ditos serem mais 'desblindados' do que os do CH3-C- (veja Figura 5.5). Por definição, a região de campo baixo de um espectro de RMN contém valores mais altos de $\gamma$ e a região de campo alto os valores mais baixo de $\gamma$: A maioria dos núcleos de hidrogênio absorve entre $\gamma$ 0,5 e $\gamma$ 12.

Se um grupo metila é ligado ao carbono, o deslocamento é $\gamma$ 0,95-0,85, o valor exato dependendo de vários outros fatores estruturais. Se, porém, a metila é ligada ao oxigênio, o oxigênio, mais eletronegativo, atrai os elétrons, distanciando-os dos prótons e desloca a absorção para $\delta$ 3,8-3,5. No Quadro 5.1 são dados alguns deslocamentos químicos para núcleos de hidrogênio ligados ao carbono, porém em ambientes químicos diferentes.

os deslocamentos químicos para os prótons hidroxílicos em álcoois geralmente sao encontrados na região $\delta$ 5,0-0,5. A posição é muito dependente da concentração por causa de ligação hidrogênio e é deslocada para valores menores de $\delta$ com a diluição. Prótons ligados a nitrogênio e enxofre podem também formar ligações hidrogênio e seus deslocamentos químicos sao dependentes da concentração, porém em um grau menor do que os ligados ao oxigênio. No Quadro 5.2 sao apresentados alguns exemplos de deslocamentos químicos de prótons ligados a átomos diferentes do carbono.

O espectro de RMN (Figura 5.6) do 2,2-dimetil-propanol ilustra o efeito da estrutura na frequência de absorção. Os prótons metílicos (-CH3) sao encontrados em $\delta$ 0,92. Os prótons metilênicos (-CH2-), que estão adjacentes a um átomo de oxigênio eletronegativo, sai mais 'desblindados' e absorvem em $\delta$ 3,20. A frequência na qual o próton da hidroxila (-OH) absorve é dependente da concentração e aqui aparece em $\delta$ 4,20.

As áreas sob os picos sao difíceis de serem medidas graficamente com precisão quando os picos sao tao estreitos. À medida que o espectro de RMN está sendo obtido, um integrador é geralmente acoplado ao registrador; este integrado mede a área sob a curva (a integral da fincão) e fornece esta informação na forma de uma série de platôs que podem ser vistos pelo espectro. A altura total de um platô é proporcional à área do pico varrido no platô. Na Figura 5.6 os platôs da esquerda para a direita têm alturas na razão 1:2:9, correspondendo ao número relativo de prótons de um platô para o próximo. Estas áreas nos dão as razões dos números de prótons que contribuem em cada platô.

Em comparação com a Figura 5.6, o espectro de RMN do metil-terbutil-éter (Figura 5.7), um isômero do 2,2-dimetil-propanol, apresenta apenas dois tipos de prótons equivalentes dando, por integração, uma razão 1:3. Os prótons do grupo metilico ligados ao oxigênio representam o pico em $\delta$ 3,12 e os do grupo terbutila representam o pico em $\delta$ 1,12. As Figuras 5.6 e 5.7 ilustram como espectros de RMN sao caraterísticos das estruturas dos compostos dos quais eles foram obtidos, em termos tanto de deslocamento químico quanto de áreas relativas sob os picos. Note que, se tivéssemos uma amostra de 2,2-dimetil-propanol em um frasco e uma amostra de metil-terbutil-éter em outro frasco, poderíamos muito facilmente identificá-los através da determinação de seus espectros do RMN.

\begin{table}[H]
    \centering
    \caption{Dados de deslocamento químico típicos para prótons ligados ao carbono.}
    %\label{my-label}
    \begin{tabular}{ccccc}
        \toprule
        Tipo do próton & Deslocamento químico, $\delta$ & Tipo de próton & Deslocamento químico, $\delta$ \\
        \midrule
        \ch{Si(CH3)4} & 0,000 & \chemfig[][]{-[,0.5]CH(-[2,0.5])-[,0.5]} & 1,6-1,4 \\ [1ex]
        \ch{CH4} & 0,22 & \chemfig[][]{CH_3-[,0.7]C(-[2,0.5])(-[6,0.5])-[,0.7]X} & 1,9-1,2 \\
        \chemfig[][]{CH_3-[,0.7]C(-[2,0.5])(-[6,0.5])-[,0.5]} & 0,95-0,85 & \chemfig[][]{CH_3-[,0.7]X} & 5,0-2,8 \\
        \chemfig[][]{-[,0.5]CH_2-[,0.7]} & 1,35-1,20 &  &  \\ [1ex]
         \bottomrule
    \end{tabular}
\end{table}

\begin{table}[H]
    \centering
    \caption{Dados de deslocamento químico típicos de prótons ligados a oxigênio, nitrogênio e enxofre.}
    %\label{tab:my_label}
    \begin{tabular}{ccc}
        \toprule
        Tipo do próton & Deslocamento químico, $\delta$ & Concentração \\
        \midrule
        Álcoois alifáticos (\chemfig[][]{-[,0.5]O-[,0.7]H}) & 0,5 & Monômero, diluição infinita \\
         & 5,0-0,5 & Em ligação hidrogênio, dependente da concentração \\
         Alquil-aminas (\chemfig[][]{-[,0.5]N(-[2,0.7]H)-[,0.7]H}) & 1,6-0,6 & Diluição infinita \\
          & 0,5-0,3 & Diluição infinita \\
         Tiois alifáticos (\chemfig[][]{-[,0.5]S-[,0.7]H}) & 1,7-1,3 & Diluição infinita \\
         \bottomrule
    \end{tabular}
\end{table}

\section{ACOPLAMENTO SPIN-SPIN}

Quando prótons tentam se alinhar com respeito ao campo magnético externo, uma interação chamada \textit{acoplamento spin-spin} pode ocorrer. O efeito de spin de um núcleo ($H_a$) é transferido aos núcleos quimicamente diferentes ($H_b$), geralmente através dos elétrons de ligação. Isto faz com que os núcleos adjacentes ($H_b$) sintam campos magnéticos líquidos, ou efetivos, diferentes daqueles que sentiriam na ausência de $H_a$. A influência de $H_a$ na intensidade do campo magnético efetivo sentido por $H_b$ depende da orientação do spin de $H_a$, relativo ao campo magnético externo. Um efeito do acoplamento spin-spin é complicar o espectro e tornar a interpretação muito mais difícil. Por outro lado, tal acoplamento produz uma informação muito útil sobre o número e tipo de prótons dos átomos de carbono adjacentes ao átomo que possui o próton em observação.

\begin{figure}[H]
    \centering
    \begin{tikzpicture}
        \begin{axis}[
            xlabel={ppm ($\delta$)},
            xmin=-0.4, xmax=8,
            x dir=reverse,
            width=\textwidth,
            height=8cm,
            ytick=\empty,
        ]
        \addplot table [mark=none] {plot_data/figure_5.6.dat};
    
        \end{axis}
    \end{tikzpicture}
    \caption{Espectro de RMN do 2,2-metil-propanol.}
    \label{fig:my_label}
\end{figure}

Suponha que consideremos um próton simples $H_b$, que está em ressonância a uma certa intensidade do campo aplicado e apresenta um pico simples a um certo valor $delta$ (Figura 5.8a). Considere, então, o que acontece a $H_b$ quando outro próton ($H_a$) é colocado na vizinhança (Figura 5.8b) $H_b$ sentirá não apenas o campo aplicado, mas também o campo de $H_a$. O núcleo $H_a$ poderá ter qualquer uma das duas orientações com respeito ao campo aplicado, paralelo ($\uparrow$) e antiparalelo ($\downarrow$). O pequeno campo gerado por $H_a$ pode aumentar ou diminuir o campo total sentido por $H_b$. Para manter a ressonância, então, será necessário reduzir ou aumentar o campo aplicado de $H_a$. O sinal $H_b$ na presença de $H_a$ é assim visto como um dublete. Como as diferenças de energia entre os estados paralelo e antiparalelo são extremamente pequenas, há praticamente um número igual de moléculas em cada estado e, portanto, os componentes do dublete têm igual intensidade.

Como estas idéias são algo complicadas, vamos repeti-las de um modo diferente. O $H_b$ isolado apresenta um pico único de área 1,00 para uma certa intensidade de campo aplicado $\delta$ (Figura 5.8a). Quando $H_a$ está próximo, ele poderá ter duas orientações igualmente possíveis, de modo que, em metade das moléculas, o campo de $H_a$ irá se adicionar a $H_0$ (a pequena seta na esquerda da Figura 5.8b). Para obter o mesmo campo efetivo em $H_b$, então, o campo aplicado terá de ser reduzido daquele teor, de modo que veremos um pico a campo baixo de $\delta$, com uma área metade da área do pico original. Com a outra metade das moléculas, o momento magnético de $H_a$ irá se opor a $H_0$ (a pequena seta à direita). Neste caso, será necessário aumentar-se $H_0$ para compensar o campo de $H_a$. Assim, o sinal original do $H_b$ isolado é dividido em um dublete simétrico pelo vizinho $H_a$. A área sob cada pico no dublete é 0,5 e os picos são equidistantes de $\delta$ em direções opostas.

A magnitude do desdobramento entre os componentes do dublete de $H_b$ é independente da intensidade do campo aplicado. Isto está em contraste com o deslocamento químico em hertz, que tem de ser convertido a unidades de $\delta$ (Equação 5), que sao então independentes do campo. A separação entre os componentes do dublete depende apenas da distância de $H_a$ e do ambiente. A separação entre os componentes do dublete é chamada constante de acoplamento, J, e é expressa em hertz. O acoplamento é essencialmente zero se $H_b$ e $H_a$ estão separados por mais de três ligações, exceto em casos especiais. Nos espectros discutidos neste capítulo, a magnitude de J é aproximadamente 5 Hz. O significado dos valores de J será considerado nos capítulos subsequentes. 

Agora, vamos considerar o caso mais complicado onde dois prótons quimicamente equivalentes ($H_a$) sao adjacentes a um próton quimicamente diferente ($H_b$), e este se encontra em observação. Quatro arranjos de spin dos dois prótons $H_a$ adjacentes sao possíveis e podem ser transmitidos a $H_b$:

\begin{enumerate}
    \item ambos com spin paralelo a $H_0$ ($\uparrow\uparrow$),
    \item um spin paralelo, um antiparalelo ($\uparrow\downarrow$),
    \item um antiparalelo, um paralelo ($\downarrow\uparrow$),
    \item ambos antiparalelos ($\downarrow\downarrow$)
\end{enumerate}

\noindent (v. Figura 5.9). Ambos os núcleos $H_a$ podem aumentar o campo aplicado, o que quer dizer que o campo aplicado $H_0$ deve ser reduzido. Do mesmo modo, ambos os núcleos $H_a$ aplicado deve ser aumentado. Finalmente, um núcleo $H_a$ pode se opor a $H_a$, enquanto o outro o aumenta, e há dois possíveis modos disto ocorrer (arranjos 2 e 3). Neste caso, não há efeito resultante dos dois núcleos $H_a$ em $H_b$ e o $H_0$ necessário é o mesmo. O próton $H_b$ é, portanto, visto como um triplete, com o pico central duas vezes maior (visto que os arranjos correspondentes dos núcleos são duas vezes mais prováveis) do que os picos das extremidades. As áreas estarão deste modo na proporção 1:2:1. A constante de acoplamento J corresponde a uma mudança na orientação do núcleo e é dada pela distância entre os picos adjacentes.

Continuando, suponhamos que existam três prótons equivalentes $H_a$, adjacentes a $H_b$. Qual será o espaçamento para $H_b$? Há oito arranjos de spin diferentes possíveis para os $H_a$, $\uparrow\uparrow\uparrow$, $\uparrow\uparrow\downarrow$, $\uparrow\downarrow\uparrow$, $\downarrow\uparrow\uparrow$, $\uparrow\downarrow\downarrow$, $\downarrow\uparrow\downarrow$, $\downarrow\downarrow\uparrow$, $\downarrow\downarrow\downarrow$, e estes levarão a quatro picos observados, com áreas 1:3:3:1 (Figura 5.10).

Dos casos discutidos, uma regra geral pode ser deduzida: se um próton ($H_b$) tem n prótons equivalentes ($H_a$) em carbonos adjacentes, sua absorção será dividida em (n + 1) picos. O valor de (n + 1) é chamado de multiplicidade. Este modo de determinar a multiplicidade é valido se a diferença de deslocamento químico entre os próton for consideravelmente químico e da constante de acoplamento se tornem próximos, este quadro se torna mais complexo.

Temos usado a expressão prótons equivalentes sem realmente defini-la. Dos prótons são ditos equivalentes se eles ocupam ambientes idênticos, como é visto com espectrômetro de RMN, A partir deste ponto de vista, os prótons no metano são, certamente, todos equivalentes, já que têm ambientes intermoleculares idênticos. Ao girarem as moléculas, os ambientes intermoleculares se equalizam. Para decidir se os prótons são equivalentes necessitamos, apenas, decidir se por rotações internas ou externas da molécula os prótons, na média, ocupam a mesma região.

A equalização dos ambientes intermoleculares por este movimento giratório é muito importante: passa-se, todavia, nas condições que acabamos de descrever e nao cause problemas. Em um cristal, o movimento giratório das moléculas geralmente é impedido e em soluções muito viscosas, vagaroso. Nestes casos há muito tipos de interações intermoleculares e isto leva, em princípio, a espectros muito complexos, na prática, a espectros de bandas largas, onde os detalhes nao podem ser resolvidos, sendo, portanto, de pouca utilidade para fins de identificação de estruturas. Deste modo, as medidas devem ser restritas a soluções de baia viscosidade, líquidos puros ou gases. As rotações internas das moléculas do tipo que temos discutido são suficientemente rápidas para que os três prótons do grupo metila, por exemplo, ocupem sempre ambientes equivalentes. Como veremos adiante, em moléculas mais complicada, dois prótons podem ser equivalentes em uma temperatura, mas deixam de sê-lo a temperatura mais baixas, em que ficam restritos a um dado ambiente.

Os prótons no etano são equivalentes entre si (mas são diferentes dos do metano). No propano há dois conjuntos de prótons nao equivalentes: os dois prótons secundários são equivalentes entre si e diferentes dos seis prótons metílicos, enquanto que os prótons metílicos são todos equivalentes entre si. Comparando-se com modelos, poderia parecer que os dois prótons do grupamento metila têm ambientes diferentes do terceiro, já que dois são vici em relação ao outro grupamento metila e um é trans. Isto é verdade para um dado instante. Entretanto, o espectrômetro de RMN opera uma certa frequência e gasta-se um tempo finito para se fazer uma medida. Os prótons metílicos perdem sua individualidade e tornam-se equivalentes quando o grupamento metila executa uma rotação de frequência muito rápida em relação à da medida. O espectrômetro de RMN nao vê, assim, os prótons metílicos separadamente. Os grupos metila no propano contêm, portanto, seis prótons equivalentes. Os picos destes prótons equivalentes ocorrem no mesmo valor de $\delta$, e nao são separados por acoplamento um com o outro,. Assim, o etano, com seis prótons equivalentes, apresenta apenas um único pico. 

É importante podermos decidir quando prótons são equivalentes, porque prótons equivalentes apresentam o mesmo deslocamento químico e seus picos nao apresentarão separação por acoplamento entre si. Nos exemplos seguintes, os prótons são divididos em conjuntos pelas letras a, b, .... Aqueles com a mesma letra são equivalentes entre si.

De posse desta introducao um tanto abstrata dos princípios de acoplamento spin-spin, vamos agora observar o espectro de RMN de um composto real, o álcool etílico, na Figura 5.11. Começando pelo lado direito do espectro, vemos o pico padrao do TMS em $\delta$ 0,00. O triplete em $\delta$ 1,20 é devido aos prótons metílicos, e a área total relativa sob eles é 3. O singlete em $\delta$ 4,80 é devido ao próton hidroxílicos, área = 1, e finalmente, o quarteto em $\delta$ 3,63 é devido aos prótons metilênicos, área = 2. Se observarmos inicialmente o grupo metila ($\delta$ 1,20) veremos que os três prótons são equivalentes e nao acoplam entre si. Eles têm dois prótons vizinhos no grupo metileno. A multiplicidade da metila é, portanto, 2 + 1 = 3, de modo que a metila é observada como um triplete. Os prótons metilênicos também nao acoplam entre si, mas acoplam com os prótons metílicos, dos quais há três. A multiplicidade do metileno é, portanto, 3 + 1 = 4, um quarteto. Note que o acoplamento é algo recíproco: se $H_a$ está acoplado com $H_b$, então $H_b$ está, do mesmo modo, acoplado com $H_a$. As constantes de acoplamento $J_{ab}$ e $J_{ba}$ devem ser sempre iguais. Os espaçamentos entre os componentes do triplete são, portanto, iguais aos espaçamentos entre os componentes do quarteto, neste caso 7.5 Hz.

Seria esperado que o \ch{-OH} acoplasse com o \ch{-CH2-} e vice-versa. Normalmente, tal acoplamento nao é visto porque o próton no oxigênio está rapidamente trocando ou sendo passado de uma molécula para a outra. Ele não fica no mesmo ambiente tempo suficiente para que seu acoplamento com os prótons metilênicos seja detectado e é, portanto, visto como um singlete. Os prótons metilênicos são vistos como um quarteto, resultante do acoplamento com o \ch{CH3-} apenas.

Finalmente, devemos lembrar que em espectros reais, os multipletes são menos simétricos do que nas Figuras 5.8 e 5.10. Multipletes simétricos são o caso limite quando o deslocamento químico é muito grande comparado com $J$. Se as diferenças de deslocamento químico entre prótons são de valor razoável, os multipletes tornam-se assimétricos de modo a 'crescerem' entre eles, como na Figura 5.11. Note que os picos do lado direito no quarteto são mais altos do que os correspondentes da esquerda. Do mesmo modo, o pico da esquerda no triplete é mais alto do que o da direita. Em espectros complicados algumas vezes é preciso algum esforço para decidir-se que átomos estão acoplados entre si. Ao interpretarmos tais espectros, é bom lembrar que, se dois conjuntos de picos correspondem a prótons acoplados entre si, os $J$ têm de ser os mesmos em ambos os conjuntos e os picos mais próximos um do outro, em cada conjunto, geralmente crescem como na Figura 5.11.

Espectros de RMN podem ser muito uteis para a identificação de compostos desconhecidos. As Figura 5.6 e 5.7 facilmente identificam dois isômeros de \ch{C5H12O}. O espectro de RMN do éter dimetílico contém um pico simples em $\delta$ 3,47, originário dos seis prótons, todos quimicamente equivalentes. Compare este espectro com o outro isômero de \ch{C2H6O}, etanol (Figura 5.11).

O meio mais efetivo de se diferenciarem moléculas simples é, geralmente, através dos seus espectros de RMN. Para ver isto melhor, vamos considerar a fórmula \ch{C3H7Cl}. Há dois isômeros com esta fórmula. Seus espectros de RMN nos permitirão decidir qual é qual? Os dois espectros são mostrados nas Figura 5.12 e 5.13. Eles são claramente diferentes e nós deveremos facilmente determinar qual espectro representa qual isômero. 

O multiplete perto de $\delta$ 4,0 na Figura 5.13 é baixa intensidade e difícil de ser visto com clareza; Esta porção do espetro necessita, portanto, ser corrida novamente, aumentando o tamanho dos picos, como mostramos.

Vamos começar com o 2-cloro-propano e prever como será o espectro de RMN. A fórmula estrutural:

mostra-nos que os seis hidrogênios das metilas são equivalentes e diferentes do hidrogênio em C-2. O espectro consistirá de dois grupos de picos: os prótons metílicos com um área de 6 em cerca de $\delta$ 1,5 (do Quadro 5.1) e próton C-2 em cerda de $\delta$ 4,0 com uma área de 1. Os prótons metílicos têm apenas um próton adjacente e apresentarão uma multiplicidade de 1 + 1 = 2. O próton C-2 tem seis prótons adjacentes e apresentará uma multiplicidade 6 + 1 = 7. Se olharmos as Figuras 5.12 e 5.13 veremos que a Figura 5.13 corresponde muito bem ao que é previsto, o que não ocorre com a Figura 5.12. 

Agora, vamos prever o espectro do 1-cloro-propano:

Há três conjuntos diferentes de prótons. Os prótons em C-1 devem dar um pico a cera de $\delta$ 3,5 (Quadro 5.1) com uma área de 2 e uma multiplicidade de $2 + 1 = 3$, os prótons em C-2 devem dar um pico a cerda de $\delta$ 1,5 com uma área de 2 e multiplicidade $(3 + 2) + 1 = 6$. Esta multiplicidade supõe que os prótons em C-1 acoplem com os de C-2 de modo equivalente aos prótons de C-3 com os de C-2. Embora isto não seja exatamente verdadeiro, a aproximação é satisfatória. Finalmente, os prótons metílicos devem apresentar ressonância a cerca de $\delta$ 0,9 com uma área de 3 e multiplicidade de $2 + 1 =3$. Recordando as Figuras 5.12 e 5.13 vemos que o espectro na Figura 5.12 é coerente com esta dados e o da Figura 5.13 não. Poderemos, portanto, atribuir estruturas a estes isômeros, de um modo inequívoco, partir de espectros de RMN.

Para resumir - o espectro de RMN de um composto nos dá três tipos de informação. O deslocamento químico de um multiplete nos diz algo sobre o ambiente do próton envolvido, a área sob o pico nos diz quantos prótons estão envolvidos, e a multiplicidade nos diz quantos prótons vizinhos existem. Para moléculas simples, esta informação geralmente é o bastante para se deduzir a estrutura total. Em moléculas complicadas, muitos dos multipletes se misturam e não podem ser resolvidos. Normalmente, entretanto, um certo número de fatores estruturais pode ser detectado e porcões da estrutura podem se deduzidas a partir do espectro de RMN.

Este capitulo é uma introdução ao espectro de RMN. Muito mais será dito nos capítulos seguintes sobra aplicações de RMN a problemas químicos.


