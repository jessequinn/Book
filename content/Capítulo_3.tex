\chapter{Os Alcanos}
\section{ESTRUTURA E NOMENCLATURA}
Existe um grande número de hidrocarbonetos com a fórmula \ch{C_{n}H_{2n+2}}. Estes compostos são chamados \textit{alcanos} ou \textit{parafinas}, e metano, \ch{CH4} (Seção 2.5), é o mais simples deles. Pelo aumento de \emph{n} obteremos as fórmulas de uma família de compostos, uma \textit{série homóloga}. Os quatro primeiros termos da série são:

\begin{tightcenter}
    \chemnameinit{}
    \chemname{\setchemfig{atom sep=2em}\chemfig[][]{H-C(-[2]H)(-[6]H)-H}}{\footnotesize Metano}
    \qquad
    \chemname{\setchemfig{atom sep=2em}\chemfig[][]{H-C(-[2]H)(-[6]H)-C(-[2]H)(-[6]H)-H}}{\footnotesize Etano}
    \qquad
    \chemname{\setchemfig{atom sep=2em}\chemfig[][]{H-C(-[2]H)(-[6]H)-C(-[2]H)(-[6]H)-C(-[2]H)(-[6]H)-H}}{\footnotesize Propano}
    \qquad
    \chemname{\setchemfig{atom sep=2em}\chemfig[][]{H-C(-[2]H)(-[6]H)-C(-[2]H)(-[6]H)-C(-[2]H)(-[6]H)-C(-[2]H)(-[6]H)-H}}{\footnotesize Butano}
    \chemnameinit{}
\end{tightcenter}

\noindent Os compostos acima são todos gasosos à temperatura ambiente, e os dois últimos, liquefeitos e colocados em cilindros, são muito usados como combustível. Os homólogos de maior peso molecular - pentano, hexano, octano, nonano, etc. - são líquidos (Tabela \ref{tab3_1}).

\begin{table}[H]
\centering
\caption{Os hidrocarbonetos normais.}
\label{tab3_1}
\begin{tabular}{cccP{4cm}cc}
\toprule
N.$\degree$ de carbonos & Fórmula & Nome & Número total de isômeros possíveis & P.E. ($\degree$C) & P.F. ($\degree$C) \\
\midrule
1 & \ch{CH4} & Metano & 1 & -162 & -183 \\ 
2 & \ch{C2H6} & Etano & 1 & -89 & -172 \\
3 & \ch{C3H8} & Propano & 1 & -42 & -187 \\
4 & \ch{C4H10} & Butano & 2 & 0 & -138 \\
5 & \ch{C5H12} & Pentano & 3 & 36 & -130 \\
6 & \ch{C6H14} & Hexano & 5 & 69 & -95 \\ 
7 & \ch{C7H16} & Heptano & 9 & 98 & -91 \\
8 & \ch{C8H18} & Octano & 18 & 126 & -57 \\
9 & \ch{C9H20} & Nonano & 35 & 151 & -54 \\
10 & \ch{C10H22} & Decano & 75 & 174 & -30 \\ 
11 & \ch{C11H24} & Undecano & - & 196 & -26 \\ 
12 & \ch{C12H26} & Dodecano & - & 216 & -10 \\
20 & \ch{C20H42} & Eicosano & 366.319 & 34 & +36 \\
30 & \ch{C30H62} & Tricontano & $4,11 \times 10^9$ & 446 & +66 \\
\bottomrule
\end{tabular}
\end{table}

\par\bigskip
\noindent\emph{NOMENCLATURA TRIVIAL. A nomenclatura de moléculas orgânicas simples não é completamente sistemática porque os compostos já eram conhecidos e tinham seus nomes próprios antes que suas estruturas fossem conhecidas. Por exemplo, o nome butano veio de um composto a ele relacionado (ácido butírico) que foi isolado pela primeira vez de manteiga rançosa. Os homólogos superiores têm uma nomenclatura mais sistemática, baseada em números gregos. Esses compostos e alguns outros são algumas vezes chamados de hidrocarbonetos alifáticos (do grego aleifar, gordura).}
\par\bigskip

Homólogos de cadeia linear contendo 18 ou mais carbono são sólidos graxos de baixo ponto de fusão, e sua mistura, comercialmente conhecida por graxa de parafina, já foi muito usada para selar garrafas de geléia, mas agora é mais usada como moderador em reatores nucleares. 

As ligações em todos os alcanos são fundamentalmente as mesmas que em metano. Os átomos são ligados por pares de elétrons em orbitais híbridos sp3 dos átomos de carbono e orbitais 1s dos átomos de hidrogênio. Cada carbono tem seus substituintes (átomos ou grupos de átomos a ele ligados) dispostos em um arranjo tetraédrico.

Frequentemente é difícil visualizar-se a estrutura tridimensional de uma molécula. Torna-se então muito útil ter 'modelos moleculares' que ajudem a compreender as estruturas. Existem três tipos principais de modelos, mostrado na Figura \ref{fig3_1}, que são uma aproximação da estrutura  e que tentam dar uma idéia do tamanho físico dos vários átomos. Embora nenhum dos tipos de modelos seja perfeitamente exato, eles têm alguma utilidade, dependendo da propriedade molecular que estejamos examinado. O leitor deverá familiarizar-se como cada tipo de modelo. Nós usaremos os diferentes tipos neste texto, dependendo das necessidades de cada caso. 

\begin{figure}[H]
    \centering
    \adjustbox{margin=1em,width=\textwidth,set height=4cm,set depth=4cm,frame,center}{Dummy}
    \caption{Os vários modelos moleculares representado o etano.}
    \label{fig3_1}
\end{figure}

Não há, em principio, limites para o numero de átomos de carbono que podem ser ligados em uma cadeia de hidrocarboneto. Nos conhecemos atualmente hidrocarbonetos lineares de mais de coem átomos de carbono. Uma das razoes para a existência de tantos compostos orgânicos é que não há verdadeiramente um limite para o tamanho das moléculas que podem ser formadas. Porém, a razão principal para a existência de tantos tipo diferentes de compostos orgânicos é o fenômeno conhecido como isomeria. Você deve lembrar-se que dois compostos diferentes são chamados isômeros se tiverem a mesma formula molecular. Embora propano e os homólogos inferiores não tenham isômeros, existem dois compostos com a formula molecular \ch{C4H10} três com a formula \ch{C5H12} e cinco com a formula \ch{C6H14}. O numero de isômeros aumenta enormemente com o aumento do número de carbonos (Tabela \ref{tab3_1}).

Se escrevemos propano, \ch{C3H8} com valências 4 para o carbono e 1 para o hidrogênio, só uma estrutura é possível. As duas formulas:

\begin{tightcenter}
    \setchemfig{atom sep=2em}\chemfig[][]{CH_3-C(-[2]H)(-[6]H)-CH_3}
    \qquad e \qquad
    \setchemfig{atom sep=2em}\chemfig[][]{H-C(-[2]CH_3)(-[6]CH_3)-H}
\end{tightcenter}

\noindent correspondem à mesma molécula vista de suas posições diferentes e representam a mesma estrutura. A fórmula:

\begin{tightcenter}
    \setchemfig{atom sep=2em}\chemfig[][]{CH_3-C(-[2]H)(-[6]CH_3)-H}
\end{tightcenter}

\noindent é também equivalente às duas outras, embora isto não seja evidente na reapresentação bidimensional, porque os quatro lados do tetraedro regular são equivalentes. Na Figura \ref{fig3_2} mostramos três vista diferentes de um modelo do propano, correspondendo às orientações das três formulas precedentes.

\begin{figure}[H]
    \centering
    \adjustbox{margin=1em,width=\textwidth,set height=4cm,set depth=4cm,frame,center}{Dummy}
    \caption{Três vistas de um modelo do propano \ch{C3H8}.}
    \label{fig3_2}
\end{figure}

À fórmula \ch{C4H10} correspondem duas estruturas moleculares que podem ser escritas:

\begin{tightcenter}
    \chemname{\setchemfig{atom sep=2em}\chemfig[][]{H-C(-[2]H)(-[6]H)-C(-[2]H)(-[6]H)-C(-[2]H)(-[6]H)-C(-[2]H)(-[6]H)-H}}{\footnotesize\emph{n}-butano}
    \qquad\qquad
    \chemname{\setchemfig{atom sep=2em}\chemfig[][]{H-C(-[2]H)(-[6]H)-C(-[2]H)(-[6,1.6]C(-[4]H)(-[6]H)-H)-C(-[10]H)(-[6]H)-C(-[2]H)(-[6]H)-H}}{\footnotesize Isobutano}
    \chemnameinit{}
\end{tightcenter}

\noindent Estas estruturas são diferentes, como pode ser visto na Figura \ref{fig3_3}. Um dos isômeros, isobutano, tem um carbono ligado a três outros carbonos, o que não acontece com o outro isômero, \emph{n}-butano.

\begin{figure}[H]
    \centering
    \adjustbox{margin=1em,width=\textwidth,set height=4cm,set depth=4cm,frame,center}{Dummy}
    \caption{À formula \ch{C4H10} correspondem o $n$-butano (à esquerda) e o isobutano (`a direita).}
    \label{fig3_3}
\end{figure}

Um composto que tem seus átomos de carbono colocados em uma cadeia linear é sempre chamado normal (\emph{n}-). Quando há uma substituição no segundo carbono, o composto é chamado \emph{iso}. Existem três alcanos com cinco carbonos: pentano normal (\emph{n}-pentano), isopentano e neopentano.

\begin{tightcenter}
    \chemname{\setchemfig{atom sep=2em}\chemfig[][]{CH_3-CH_2-CH_2-CH_2-CH_3}}{\footnotesize\emph{n}-pentano}\qquad
    \chemname{\setchemfig{atom sep=2em}\chemfig[][]{CH_3-CH(-[6]CH_3)-CH_2-CH_3}}{\footnotesize isopentano}\qquad
    \chemname{\setchemfig{atom sep=2em}\chemfig[][]{CH_3-C(-[2]CH_3)(-[6]CH_3)-CH_3}}{\footnotesize neopentano}
    \chemnameinit{}
\end{tightcenter}

Ficou claro, muitos anos atrás, que, uma vez que o número do isômeros possíveis para uma molécula de grande número de carbonos é enorme, designar pura e simplesmente cada isômero por um nome particular arbitrário não seria razoável. Uma nomenclatura sistemática foi então adotada pela União Internacional de Química Pura e Aplicada (UIQPA), uma organização internacional criada para tratar deste tipo de problemas.

Na nomenclatura da UIQPA a cadeia mais longa que se pode encontrar na molécula determina sua nomenclatura. O hidrocarboneto linear de mesmo número de carbonos, chamado composto principal, fornece a base para a nomenclatura do composto. Os átomos dessa cadeia (cadeia principal) são numerados em sequência de um extremo ao outro de maneira tal que os substituintes na cadeia recebam os menores números possíveis. Quando uma porção de um hidrocarboneto é considerada como substituinte da cadeia principal é chamada um grupo alquila e o nome geral do composto é alquil-alcano.

Para obter-se o nome de um grupo alquila substitui-se o sufixo \textit{-ano} do alcano por ila. Assim, \ch{CH3CH3} é etano e \ch{CH3CH2} é etila. Os nomes e abreviaturas de alguns grupamentos alquila mais comuns estão na Tabela \ref{tab3_2}. Um átomo de carbono é, às vezes, chamado de primário, secundário, terciário ou quaternário quando está ligado, respetivamente, a um, dois, três ou quatro átomos de carbono. Os nomes $s$-butila e $t$-butila na Tabela \ref{tab3_2} indicam que, nestes grupamentos alquila, os pontos de ligação à cadeia principal estão nos átomos ligados, respetivamente, a dois e três átomos de carbono.

\begin{table}[H]
    \centering
    \caption{Grupos alquila comuns (R-) e fragmentos relacionados.$^a$}
    \label{tab3_2}
    \begin{tabular}{ccc}
        \toprule
        Grupo & Nome$^b$ & Abreviatura \\
        \midrule
        \setchemfig{atom sep=2em}\chemfig[][]{CH_3-} & Metila & Me \\ 
        \setchemfig{atom sep=2em}\chemfig[][]{CH_3-CH_2-} & Etila & Et \\
        \setchemfig{atom sep=2em}\chemfig[][]{CH_3-CH_2-CH_2-} & \emph{n}-Propila & \emph{n}-Pr \\
        \setchemfig{atom sep=2em}\chemfig[][]{CH_3-CH(-[6]CH_3)-} & Isopropila & \emph{i}-Pr \\ [5ex]
        \setchemfig{atom sep=2em}\chemfig[][]{CH_3-CH_2-CH_2-CH_2-} & \emph{n}-Butila & \emph{n}-Bu \\ [1ex]
        \setchemfig{atom sep=2em}\chemfig[][]{CH_3-CH(-[6]CH_3)-CH_2-} & Isobutila & \emph{i}-Bu \\ [5ex]
        \setchemfig{atom sep=2em}\chemfig[][]{CH_3-CH_2-CH(-[6]CH_3)-} & \emph{s}-Butila & \emph{s}-Bu \\ [5ex]
        \setchemfig{atom sep=2em}\chemfig[][]{CH_3-C(-[2]CH_3)(-[4]CH_3)(-[6]CH_3)-} & \emph{t}-Butila & \emph{t}-Bu \\ [5ex]
        \setchemfig{atom sep=2em}\chemfig[][]{CH_3-CH_2-CH_2-CH_2-CH_2-} & \emph{n}-Pentila (ou n-amila) & \emph{n}-Am \\
        \setchemfig{atom sep=2em}\chemfig[][]{CH_3-CH(-[6]CH_3)-CH_2-CH_2-} & Isopentila (ou isoamila) & -\\
        \setchemfig{atom sep=2em}\chemfig[][]{-CH_2-} & Metileno & - \\ [2ex]
        \setchemfig{atom sep=2em}\chemfig[][]{-C(-[2])(-[4])(-[6])-H} & Metino & - \\ [4ex] 
        \bottomrule
        \multicolumn{3}{p{0.88\textwidth}}{\footnotesize $^a$Com frequência o símbolo R- é usado para representar um agrupamento alquila indeterminado (ou um radical). Assim, R-H é um alcano qualquer.} \\
        \multicolumn{3}{p{0.88\textwidth}}{\footnotesize $^b$Em português, ao formar-se o nome composto, a vogal a final dos nomes dos grupamentos desaparece por eufonia.}
    \end{tabular}
\end{table}

Vejamos agora como se nomeia o composto seguinte pelo sistema UIQPA.

\begin{tightcenter}
    \vspace{1em}
    \chemname{\setchemfig{atom sep=2em}\chemfig[][]{(!\nobond\chemabove[2ex]{}{1})CH_3-(!\nobond\chemabove[2ex]{}{2})CH(-[6]CH_3)-(!\nobond\chemabove[2ex]{}{3})CH_2-(!\nobond\chemabove[2ex]{}{4})CH_2-(!\nobond\chemabove[2ex]{}{5})CH_3}}{\footnotesize 2-Metil-pentano}
    \chemnameinit{}
\end{tightcenter}

\noindent A cadeia linear mais longa contém cinco átomos de carbono, logo o composto será um alquil-pentano. O grupamento alquila é um grupamento metila e deve-se numerar a cadeia principal em sequência de modo a dar ao átomo de carbono ao qual se liga o grupamento metila o menor número possível. Se numerássemos a cadeia em sequência da esquerda para a direita o grupamento metila estaria no carbono 2, enquanto que, se o fizéssemos da direita para a esquerda, no carbono 4. A numeração correta é portanto a primeira. O nome do composto é 2-metil-pentano. A aplicação das regras da UIQPA aos seguintes compostos produz os nomes apresentados em cada caso.

\begin{tightcenter}
    \chemname{\setchemfig{atom sep=2em}\chemfig[][]{CH_3-C(-[2]CH_3)(-[6]CH_3)-CH_2-CH_3}}{\footnotesize 2,2-Dimetil-butano}
    \qquad\qquad
    \chemname{\setchemfig{atom sep=2em}\chemfig[][]{CH_3-CH(-[2]CH_3)-CH(-[2]CH_3)-CH_3}}{\footnotesize 2,3-Dimetil-butano}
    \qquad\qquad
    \chemname{\setchemfig{atom sep=2em}\chemfig[][]{CH_3-CH_2-CH(-[6]CH_2CH_3)-CH_2-CH_3}}{\footnotesize 3-Etil-pentano}
    \chemnameinit{}
\end{tightcenter}

Deve ficar bem claro que a cadeia principal, correspondendo ao composto principal, é a cadeia mais longa possível, mesmo que na representação gráfica não esteja numa linha reta. Assim, o composto:

\begin{tightcenter}
    \setchemfig{atom sep=2em}\chemfig[][]{CH_3-CH_2-CH(-[6]CH_2CH_2CH_3)-CH_2-CH_2-CH_3}
    \chemnameinit{}
\end{tightcenter}

\noindent não é nomeado como um derivado do hexano, mas sim do heptano. O nome correto do composto é \textit{4-etil-heptano}. 

Se existirem dois ou mais tipos diferentes de grupamentos alquila ligados à cadeia principal, seus nomes são normalmente colocados em ordem alfabética (sem levar em conta os prefixos designativos de quantidade, di, tri, etc., e os prefixos $t$-, $s$-, $n$-). Assim, teremos:

\begin{tightcenter}
    \chemname{\chemname{\setchemfig{atom sep=2em}\chemfig[][]{CH_3-C(-[2]CH_3)(-[6]CH_3)-CH_2-CH_2-CH(-[6]CH_2CH_3)-CH_2-CH_3}}{\footnotesize 5-Etil-2,2-dimetil-heptano}}{\footnotesize (não 2,2-dimetil-5-etil-heptano)}
    \qquad\qquad
    \chemname{\chemname{\setchemfig{atom sep=2em}\chemfig[][]{CH_3-C(-[2]CH_3)(-[6]CH_3)-CH_2-CH_2-CH(-[6]CH(-[6]CH_3)-CH_3)-CH_2-CH_2-CH_3}}{\footnotesize 5-Isopropil-2,2-dimetil-octano}}{\footnotesize (não 2,2-dimetil-5-isopropil-octano)}
    \chemnameinit{}
\end{tightcenter}

As regras para a nomenclatura de hidrocarbonetos podem ser sumariadas como se segue:

\begin{description}
\item \textit{Regra 1}: Ache a cadeia de carbonos mais longa. Isto fornecerá o nome do composto principal.
\item \textit{Regra 2}: Identifique os grupamentos alquila ligados à cadeia principal.
\item \textit{Regra 3}: Nomeie cada um destes grupamentos e coloque-se em ordem alfabética (sem levar em consideração os prefixos designativos de quantidade ou os prefixos t-, s-, n-) diante do nome do hidrocarboneto principal.
\item \textit{Regra 4}: Numere o composto principal de tal maneira que os substituintes recebam os menores números possíveis.
\end{description}

Usando estas regras, mesmo uma molécula complicada pode ser nomeada de maneira clara e inambígua. Por exemplo, o hidrocarboneto:

\begin{tightcenter}
    \setchemfig{atom sep=2em}\chemfig[][]{CH_3-CH(-[2]CH_3)-CH_2-CH(-[6]CH(-[6]CH_3)-CH_3)-CH_2-C(-[2]CH_3)(-[6]CH_2CH_2CH_3)-CH_3}
    \chemnameinit{}
\end{tightcenter}

\noindent é chamado 4-isopropil-2,6,6-trimetil-nonano.

No caso de estruturas muito complicadas pode ser difícil decidir entre diversos nomes igualmente possíveis. A maior parte dos químicos aceita qualquer nome plausível que define sem ambiguidade uma estrutura. Qualquer nome corretamente elaborado deve levar a uma única estrutura mas, às vezes, uma estrutura pode ter vários nomes corretos.

\noindent\textbf{Aprendendo nomenclatura}\\
Nomenclatura em química é como o vocabulário em qualquer língua. É essencial saber manipular corretamente a nomenclatura sob pena de não se poder discutir química. Nomenclatura, como as palavras do vocabulário, deve ser às vezes memorizada. Nos capítulos subsequentes serão apresentadas as regras básicas da nomenclatura das diversas classes de compostos orgânicos. O leitor deverá recorrer a um texto especializado para se familiarizar com os detalhes da nomenclatura em língua portuguesa. 

\section{PETRÓLEO}
A principal fonte de compostos orgânicos no mundo de hoje é o petróleo. O óleo que se extrai da Terra é uma mistura complicada de compostos com a predominância de hidrocarbonetos. Os alcanos, de metano até os de cerca de 30 carbonos, são os componentes principais da fração hidrocarbônica. Predominam os hidrocarbonetos de cadeia linear.

\par\bigskip
\noindent REFINAÇÃO. \emph{O processamento de óleo cru, chamada refinação é uma operação extremamente complexa. A refinação começa com a separação do óleo cru em várias frações pelo processo da destilação fracionada. O material a ser destilado é colocado em um recipiente adequado e a temperatura é aumentada gradualmente. Os constituintes de mais baixo ponto de ebulição destilam primeiro e são seguidos progressivamente pelos materiais de mais alto pouco de ebulição. As frações normalmente destiladas na refinação de óleo cru estão na Tabela \ref{tab3_3}.}
\par\bigskip

\begin{table}[H]
    \centering
    \caption{Frações obtidas pela destilação de óleo cru.}
    \label{tab3_3}
    \begin{tabular}{P{5cm}cc}
        \toprule 
        Nome de fração & Conteúdo em carbono & Faixa aproximada de ebulição ($\degree$C) \\
        \midrule
        Gás natural & C$_1$-C$_4$ & Abaixo da temperatura ambiente \\
        Eter de petróleo & C$_5$-C$_6$ & 20-60 \\
        Ligroína (nafta leve) & C$_6$-C$_7$ & 60-100 \\
        Gasolina & C$_6$-C$_12$ & 50-200 \\
        Querosene & C$_12$-C$_18$ & 175-275 \\
        Óleo combustível (óleo de fornalha, óleos diesel) & Acima de C$_18$ & Acima de 275 \\
        Óleos lubrificantes & & Não destila à temperatura ambiente \\
        Graxas & & Não destila à temperatura ambiente \\
        Asfalto & & Resíduo \\
        \bottomrule
    \end{tabular}
\end{table}

\par\bigskip
\emph{O éter de petróleo e a ligroína são muito usados como solventes. A fração correspondente aos óleos lubrificantes é destilada sob pressão reduzida para dar óleos lubrificantes leves, médios e passados. A fração de querosene fornece o combustível necessário para o funcionamento de turbinas a gás e motores a jato. As necessidades de nossa civilização com respeito a estes produtos têm mudado muito nos últimos cinquenta anos. Antes que o uso dos automóveis se generalizasse, o querosene era muito importante para a iluminação e a gasolina quase sem valor. Depois de 1950, a situação já era totalmente ao inverso. Mais recentemente a procura de querosene tem aumentado graças ao desenvolvimento de turbinas a gás e motores a jato para aviação. Uma importante função da indústria de petróleo é a conversão daquelas frações economicamente pouco importantes naquelas de que a demanda é maior. Os processos químicos envolvidos são bastante elaborados e discutiremos alguns deles nos próximos capítulos.}

\emph{O petróleo foi, e ainda é, uma fonte muito importante e conveniente de energia. Entretanto, o fato de a quantidade disponível de petróleo ser limitada é suficientemente claro nos dias de hoje mesmo ao cidadão comum. Na velocidade com que vêm sendo usadas, as reservas conhecidas não durarão mais que cinquenta anos. Como ficará evidente partir de derivados de petróleo, e as futuras gerações provavelmente lamentarão profundamente a maneira pela qual a população humana deste século desperdiçou, pela queima pura e simples, esta riqueza valiosa e insubstituível.}
\par\bigskip

\section{COMPOSTOS ACÍCLICOS: ANÁLISE CONFORMACIONAL}
Até agora nós aceitamos tacitamente que a rotação em torno de uma ligação simples era completamente livre. Assim, um dos grupamentos metila do etano poderia girar em relação ao outro sem alterar a energia da molécula. Se a rotação em torno da ligação central do butano não fosse livre, deveríamos esperar que existissem dois isômeros rotacionais, talvez como estes: 

\begin{tightcenter}
    \chemfig[][scale=0.7]{C(-[2]H)(-[5]H)(-[7]CH_3)-[1,2]C(-[1]CH_3)(-[3]H)(-[6]H)}
    e
    \chemfig[][scale=0.7]{C(-[2]H)(-[5]CH_3)(-[7]H)-[1,2]C(-[1]CH_3)(-[3]H)(-[6]H)}
\end{tightcenter}

\noindent como tais isômeros nunca foram isolados, pensava-se que houvesse rotação livre em torno das ligações simples. 

Em 1935, E. Teller e B. Topley sugeriram que está idéia talvez não estivesse correta. Eles estavam estudando a capacidade calorífica do etano e observaram que era mais baixa do que a teoria indicava, se a rotação fosse livre em torno da ligação C-C. Eles sugeriram, então, que, se a rotação fosse dificultada por uma barreira de energia, a teoria e a experiência poderiam coincidir.

Posteriormente, alguns estudos mostraram que existem barreiras de energia que dificultam a rotação em torno de ligações \ch{C-C}, em geral, assim como de ligações \ch{C-N}, \ch{C-O}, e a maioria das ligações simples. Portanto, a possibilidade da existência de isômeros diferentes apenas nos seus arranjos rotacionais internos teria de ser considerada. Este tipo de isomeria mereceu a atenção de muitos químicos de 1935 até 1950. Muitos dos princípios básicos foram descobertos por físico-químicos, principalmente por Pitzer e Mizushima. 

As barreiras de energia para a rotação em torno de uma ligação simples são baixas (algumas quilocalorias por mol) na maior parte dos casos, de modo que, à temperatura ambiente, as moléculas possuem energia térmica suficiente para passá-las sem problemas. Não é surpreendente, pois, que os isômeros correspondentes aos diferentes arranjos rotacionais nunca tivessem sido isolados pelos antigos químicos. Estes isômeros existem mas convertem-se uns nos outros muito rapidamente, não permitindo assim a separação. Em 1950, Barton (então na Universidade de Glasgow) mostrou que muitas das propriedades químicas e físicas de moléculas complicadas podiam ser mais facilmente compreendidas se interpretadas em termos de arranjos rotacionais específicos ou preferidos. Mesmo que não possamos isolar os isômeros rotacionais, as propriedades da molécula dependerão das proporções relativas dos diversos isômeros rotacionais presentes. 

Moléculas que diferem entre si apenas pela rotação em torno de ligações simples (isômeros rotacionais) são chamadas habitualmente isômeros conformacionais ou confôrmeros. A interpretação das propriedades dos compostos em termos de suas conformações é chamada análise conformacional. Este é um ramo da química orgânica que continua sendo estudado pelos químicos, mas, agora, os princípios fundamentais são suficientemente claros.

No etano podemos imaginar dois extremos no arranjo de um grupamento metila com respeito ao outro durante a rotação em torno da ligação C-C. São os arranjos conhecidos como em coincidência e em oposição. As fórmulas em perspectiva e usando as projeções frontais segundo as ligações carbono-carbono, conhecidas como projeções de Newman, são: 

\begin{figure}[H]
    \centering
    \setchemfig{cram width=2pt}
    \chemname{\chemfig[][scale=0.7]{C(<[2]H)(<[5]H)(<[7]H)>[1,2]C(-[:90]H)(-[:210]H)(-[7]H)}}{Em coincidência}
    \quad e \quad
    \setchemfig{cram width=2pt}
    \chemname{\chemfig[][scale=0.7]{C(<[2]H)(<[5]H)(<[7]H)>[1,2]C(-[1]H)(-[3]H)(-[6]H)}}{Em oposição}
    \quad\quad\quad
    \chemnameinit{}
    \chemname{\newman[scale=0.7](43){H,H,H,H,H,H}}{Em coincidência}
    \quad\quad\quad
    \chemname{\newman[scale=0.7](120){H,H,H,H,H,H}}{Em oposição}
    \chemnameinit{}
\end{figure}

Em uma projeção de Newman o ângulo entre a direção C-H do grupamento metila que se vê primeiro na projeção frontal e a direção do C-H do outro grupamento metila (o ângulo $\omega$) vária de $0\degree$ a $360\degree$ com a rotação do grupamento metila. Se definirmos como sendo $\omega = 0$, o arranjo em que os hidrogênios do carbono que estão atrás estão escondidos pelos hidrogênios do carbono de frente, isto é, em coincidência na projeção (embora na Figura acime estejam ligeiramente deslocados para ficar mais claro), então, $\omega = 60\degree$ corresponde a uma conformação em oposição (porque, neste caso, existe um centro de simetria em relação ao qual os seis hidrogênios estão em oposição, dois a dois). $\omega=120\degree$ corresponde a uma nova conformação em coincidência, e assim por diante. A energia da molécula vária com $\omega$ de uma forma aproximada a uma senóide, com três máximos e três mínimos. A altura da barreira rotacional no etano, calculada a partir de dados de capacidade calorifica, e de 2,8 kcal mol$^{-1}$. Estas informações estão na Figura 3.4. 

A altura da barreira é tão que, à temperatura ambiente, as moléculas passam a maior parte do tempo no possou de potencial e apenas ocasionalmente têm energia suficiente para alcançar a energia mais alta. No entanto, como as coisas acontecem muito rapidamente na escala de tempo molecular, 'ocasionalmente' significa aqui que o fenômeno acontece muitas vezes por segundo.

\par\bigskip
\emph{O QUE CAUSA AS BARREIRAS ROTACIONAIS? A existência de barreiras rotacionais é coincida desde 1936, mas, apesar de muito trabalho, desde então, não temos certeza do que as causa, em termos de um modelo mecânico ou eletrostático simples. Muitas 'razões' foram imaginadas e depois descartadas. Uma ideia simples era que os hidrogênios estão muito perto uns dos outros na forma em coincidência e que existe uma repulsão de van der Waals (veja Seção 4.18) entre eles, mas a repulsão calculada e de apenas 10\% da que seria necessária para causar uma barreira tão grande como a observada. Como as demais explicações foram igualmente descartadas, pode-se perguntar se é possível dar uma explicação na base de um modelo físico simples para a barreira. Se a equação Schrodinger é resolvida (por métodos aproximados) para o etano na forma coincidência e na forma em oposição observa-se uma diferença de energia da ordem de 3 kcal mol$^{-1}$. O resultado concorda com a experiência é isto, afinal, e todo que equação de Schrodinger pode nos fornecer (e é o máximo que podemos esperar dela). No entanto, os químicos continuarão a procurar um modelo físico simples para este fenômeno, pois seria útil saber o que esperar, em termos de barreiras de rotação, naqueles novos casos onde a solução da equação de Schrodinger e extremamente difícil. De qualquer maneira, mesmo que seja incompleto nosso entendiamento das causa da existência de uma tão barreira, sabemos o que esperar, em situações normais, onde deverão existir barreiras de rotação e sua ordem de magnitude, o que é suficiente para que possamos fazer uso delas na análise conformacional.}
\par\bigskip

Se considerarmos a rotação em torno da ligação C$_2$-C$_3$ do butano devemos esperar uma barreira senoidal semelhante à do etano. Quando $\omega = 0\degree$, os dois grupos metila estão em coincidência e a energia será máxima. A $\omega = 60\degree$ existe um arranjo em oposição que corresponde a um mínimo de energia. Outro arranjo em coincidência ocorrerá a $\omega = 120\degree$ e em oposição a $180\degree$.

O butano é diferente do etano, entretanto, porque as conformações $\omega =60\degree$ e $\omega=180\degree$ não são idênticas. O arranjo onde $\omega=60\degree$ é chamado conformação vici (do latim vicinale, que deu origem a vicinal e do latim vicinu, que deu origem a vizinho) enquanto que a conformação $\omega=180\degree$ é chamada conformação anti (do grego anti, em oposição, contra) (veja, também, a Figura 3.5).

Os grupamentos metila estão mais próximos na conformação vici e uma repulsão de van der Waals se estabelece entre eles, o que é traduzido por uma diferença de energia de 0,8 kcal mol$^{-1}$. A energia é maior portanto na conformação vici do que na conformação anti.

\par\bigskip
\begin{leftbar}[cut=false]
\footnotesize

\noindent MATÉRIA OPCIONAL

\noindent\emph{Tensão de Pitzer}. A curva completa mostrando a energia do n-butano como uma função do ângulo de torção, $\omega$ é a curva cheia da Figura 3.6. Esta energia rotacional (algumas vezes chamada energia torsional ou tensão de Pitzer) é feita de três componentes. A primeira é a contribuição da energia torsional do etano, responsável por 2,8 kcal mol$^{-1}$ nos ângulos $\omega=0\degree$, $\omega=120\degree$ e $\omega=240\degree$. A isto deve ser somada a repulsão entre os dois grupos metila que atinge o máximo de 1,7 kcal mol$^{-1}$ a $\omega=0\degree$, mas torna-se desprezível quando os dois grupamentos metila afastem-se. A terceira aparece quando o hidrogênio e o grupamento metila estão em coincidência ($\omega=120\degree$ e $\omega=240\degree$) graças a uma pequena repulsão de van der Waals que aparece. A figura 3.6 ilustra o que acabamos de dizer.
\end{leftbar}
\par\bigskip

Os \emph{n}-alcanos, em fase cristalina, existem na forma de cadeias totalmente anti, tal como aparecem na Figura 3.7. Na fase líquida, entretanto, existe um número significante de moléculas na configuração vici em uma das ligações a cada momento, um número menor com duas configurações vici na mesma molécula, um menor número ainda com três configurações vici, e assim por diante. Um alcano de alto peso molecular é, então, em fase líquida, uma mistura complicada de confôrmeros, e suas propriedades conformacionais uma média matemática difícil de estudar. (O termo `confôrmero` é restrito habitualmente ao arranjo conformacional que corresponde a uma mínima de energia.)

\section{CICLO-HEXANO: ANÁLISE CONFORMACIONAL}

Os alcanos que discutimos até agora têm a fórmula geral $C_{n}H_{2n+2}$ e são chamados acíclicos (sem anéis). Os alcanos cíclicos conhecidos por alicíclicos (isto é, hidrocarbonetos cíclicos alifáticos) formam uma classe importante de hidrocarbonetos e têm fórmula $C_{n}H_{2n}$. Os hidrocarbonetos cíclicos podem ser olhados como alcanos que têm a cadeia de tal forma que suas duas extremidades se aproximam, dois dos hidrogênios são retirados e os carbonos formam uma ligação C-C. Por exemplo, n-hexano e ciclo-hexano (geralmente abreviado por um hexágono).

\noindent Embora a molécula pareça plana nas representações acima, os átomos de carbono no ciclo-hexano não estão todos no mesmo plano. Se estivessem, os ângulos CCC seriam de $120\degree$ (o ângulo interno do hexágono regular). Os ângulos CCC preferem ser próximo dos ângulos internos do tetraedro regular (o valor experimental é $112\degree$ para o propano), e forçar a abertura deste ângulo até $120\degree$ aumentaria a energia do sistema. Portanto, no estado fundamental - o estado de mais baixa energia - o ciclo-hexano não é planar. Nesta seção trataremos apenas do caso do ciclo-hexano. Discutiremos outros hidrocarbonetos alicíclos na próximo seção.

Temos a tenência de imagina que os alcanos normais seriam os casos mais simples, mas, pelo menos em matéria de análise conformacional, o sistema ciclo-hexano é o mais simples. O confômero mais estável desta molécula é a forma cadeira. Este confômero é rígido, no sentido em que uma mudança no ângulo diedral exige uma mudança simultânea nos ângulos de ligação da molécula. O ciclo-hexano, portanto, tem apenas uma conformação estável importante, em contraste com os alcanos normais que são misturas de arranjos anti e vici.

O leitor deveria acompanhar a discussão que se segue com a ajuda de modelos atômicos, de modo a facilitar a compreensão do texto.

O anel do ciclo-hexano é um sistema conformacional simples por duas razões: (1) porque exite em uma única conformação (menos de 1\% de outras conformações à temperatura ambiente); (2) porque a molécula tem muita simetria. A simetria é tal que todos os átomos de carbono são equivalente e os hidrogênios podem ser divididos em duas classes equivalentes, os hidrogênios axiais e equatoriais. As seis ligações C-H axiais são paralelas umas às outras, assim como ao eixo de simetria da molécula, três na parte superior e três na parte inferior. Os hidrogênios equatoriais são paralelos dois a dois e dispostos alternadamente acima e abaixo do plano médio da molécula. Se o hidrogênio axial, ligado a determinado carbono, está acima do plano médio da molécula o hidrogênio equatorial ligado ao mesmo átomo de carbono estará abaixo, e vice-versa.

Existe outra conformação possível para o anal do ciclo-hexano que tem os ângulos normais. É a chamada forma bote. A forma bote é energeticamente desfavorável por causa dos dois arranjos em coincidência que possui.

\noindent (Observe na figura e nos modelos moleculares que a porção mais escura corresponde a um butano vici na forma cadeira, e a um butano em coincidência na forma bote). Não obstante, uma pequena quantidade de moléculas na forma bote está em equilíbrio com as moléculas na forma cadeira à temperatura ambiente, e este equilíbrio se estabelece rapidamente. Se olharmos a forma cadeira (A) e prestarmos atenção aos hidrogênios axiais veremos que ao levantarmos a ponta esquerda da molécula converteremos a forma cadeira (A) na forma bote (B) (na qual os mesmos hidrogênios aparecem em suas novas posições). É possível recolocar a ponta esquerda da molécula na posição anterior e voltar a conformação cadeira (A). É possível, também, baixar a ponta direita da molécula na forma bote, e, neste caso, chegaremos à nova forma cadeira (C). Essas duas formas são idênticas exceto pelo fato que agora os hidrogênios que eram axiais são equatoriais e vice-versa. (Os hidrogênios axiais e equatoriais se interconvertem, portanto).

Quem estiver acompanhando a discussão com os modelos moleculares poderá sentir a tensão introduzida no sistema ao passar da forma cadeira para a forma bote. Existe maior tensão nas formas intermediárias (meia cadeira) e, em consequência, uma barreira de energia separa as formas cadeiras e bote.

A possibilidade de existência de uma forma bote para o ciclo-hexano foi reconhecida em 1890 pelo químico alemão Sachse. A forma bote tem os ângulos próximos aos do tetraedro regular como a forma cadeira, e são os dois únicos arranjos especiais onde isto ocorre. Entretanto, enquanto a forma cadeira é bastante rígida, a forma bote pode flexionar-se, sem nenhuma deformação angular, dando uma forma, algo mais estável, chamada conformação torcida:

\begin{figure}[H]
    \centering
    \schemestart
        \setchemfig{cram width=2pt}
        \chemname{\chemfig[][scale=0.7]{?<[:300,1]-[:0,2,,,line width=2pt]>[:60,1](!\nobond\chemabove[1ex]{}{1})-[:290,0.5](!\nobond\chemabove[1ex]{}{2})-[4,2.2]?}}{Bote}\arrow
        \setchemfig{cram width=2pt}
        \chemname{\chemfig[][scale=0.7]{?<[:315,1]-[:45,2,,,line width=2pt](!\nobond\chemabove[1ex]{}{1})>[:315,1](!\nobond\chemabove[1ex]{}{2})-[:225,1]-[:135,2]?}}{Torcida}\arrow
        \setchemfig{cram width=2pt}
        \chemname{\chemfig[][scale=0.7]{?<[:300,1]-[:0,2,,,line width=2pt](!\nobond\chembelow[1ex]{}{1})>[:60,1](!\nobond\chemabove[1ex]{}{2})-[:290,0.5]-[4,2.2]?}}{Bote}
    \schemestop
\end{figure}

\noindent Por um movimento contínuo de flexão, a proa (ou a popa) do bote move-se em torno do anel do carbono 1 (C-1) para C-2, depois C-3 etc. Este movimento é a razão de se chamar às vezes a forma bote de forma flexível. É necessário ao leitor acompanhar o movimento descrito com o auxílio de modelos. 

Uma vez que os modelos mecânicos indicavam que existiam duas formas sem tensão aparente para o ciclo-hexano, Sachse imaginou que elas pudessem ser separadas. Apenas um tipo de ciclo-hexano, entretanto, poderia ser isolado experimentalmente, o que levou os químicos da época a acreditarem, incorretamente, que o ciclo-hexano era uma molécula planar.

Nós sabemos, por outro lado, que nossos modelos mecânicas rígidos não são corretos em dois aspectos. Primeiro, que as formas cadeira e bote não são separáveis por técnicas como a destilação porque se interconvertem rapidamente. A barreira que separa as duas formas cadeira interconversíveis, (A) e (C), uma da outra é de aproximadamente 10 kcal mol$^{-1}$. Uma barreira deste tipo é suficientemente baixa para que muitas moléculas a ultrapassem a cada segundo à temperatura ambiente. Esta barreira de potencial relativamente baixo para a interconversão explica porque os químicos que nos precederam não foram capazes de isolar as duas conformações do ciclo-hexano. Segundo, porque os modelos mecânicos não permitem prever diferença de energia entre as formas cadeira e bote. Por causa da barreira rotacional em torno das ligações C-C, a forma cadeira é cerca de 5 kcal mol$^{-1}$ mais estável do que a forma bote.

A energética da interconversão das duas formas cadeira do ciclo-hexano através da forma flexível está sumariada no diagrama da Figura 3.9.

\noindent MATÉRIA OPCIONAL

\begin{small}
\noindent A inspeção dos desenhos anteriores e respectivos modelos mostra que a forma bote contém duas unidade butano em coincidência e duas em oposição. A conformação torcida da forma bote, entretanto, é mais estável do que a forma bote perfeita. Na conformação torcida nenhuma das unidades butano está exatamente em coincidência, porém, nenhuma está, propriamente, em oposição. Dos ângulos diedro e da Figura 3.6, pode-se estimar que a energia da confirmação torcida é cerca de 5 kcal mol$^{-1}$ acima da energia da conformação em cadeira. Esta diferença de energia é suficiente para que apenas 0,01\% da forma flexível esteja em equilíbrio com a forma cadeira. Então, não é apenas impossível isolar a forma flexível do ciclo-hexano, mas é quase impossível detectá-la pelos métodos analíticos diretos de que dispomos. Derivados simples do ciclo-hexano são encontrados, igualmente, quase exclusivamente na forma cadeira. Em moléculas mais complicadas as formas flexíveis podem, às vezes, ser isoladas.
\end{small}

Em um ciclo-hexano monossubstituído, tal como metil-ciclo-hexano, existe um equilíbrio no qual apenas duas conformações estão presentes de modo significativo. São duas formas cadeira, uma com o grupo metila axial e outra com o grupo metila equatorial: 

\noindent O equilíbrio é igualmente estabelecido através de uma forma flexível. Uma vez que metil-ciclo-hexano é uma mistura de dois confôrmeros, é preciso perguntar: Qual dos confôrmeros é mais estável e qual a diferença de energia entre eles?

Em nossa discussão da molécula do n-butano (Seção 3.3) afirmamos que a forma anti é mais estável do que a forma vici por causa da repulsão de van der Waals entre os dois grupos metila na forma vici. É possível analisarmos as confirmações do metil-ciclo-hexano em termos destes arranjos tipo butano anti e vici. Como no caso do n-butano, cada conformação vici deve ser responsável por um aumento de cerca de 0,8 kcal mol$^{-1}$ de energia da molécula em relação à conformação anti correspondente. A experiência mostra que isto é também verdade para derivados de ciclo-hexano em geral. Qualquer interação que envolva apenas átomos do anel é igual para ambas as conformações e, portanto, não precisa ser considerada. No caso do metil-ciclo-hexano é preciso olhar, apenas as interações que envolvem o grupo metila e que são duas para cada conformação. Se o grupamento metila é equatorial são ambas anti, enquanto que, se axial são ambas vici.

\section{OUTROS HIDROCARBONETOS ALICÍCLICOS}

Os ciclo-alcanos formam uma série homóloga, da qual o ciclo-propano é o menor membro. Temos, ainda, ciclo-butano, ciclo-pentano, ciclo-hexano, etc.

\begin{figure}[H]
    \centering
    \chemname{\chemfig[][scale=0.7]{*3(\ch{CH_2}-\ch{CH_2}-\ch{CH_2}-)}}{Ciclo-propano}
    \quad\quad\quad\quad
    \chemname{\chemfig[][scale=0.7]{*4(\ch{CH_2}-\ch{CH_2}-\ch{CH_2}-\ch{CH_2}-)}}{Ciclo-butano}
    \quad\quad\quad\quad
    \chemname{\chemfig[][scale=0.7]{*5(\ch{CH_2}-\ch{CH_2}-\ch{CH_2}-\ch{CH_2}-\ch{CH_2}-)}}{Ciclo-pentano}
    \quad\quad\quad\quad
    \chemname{\chemfig[][scale=0.7]{*6(\ch{CH_2}-\ch{CH_2}-\ch{CH_2}-\ch{CH_2}-\ch{CH_2}-\ch{CH_2}-)}}{Ciclo-hexano}
    \chemnameinit{}
\end{figure}

\noindent Estes compostos são frequentemente representados por uma abreviação pictórica.

\begin{figure}[H]
    \centering
    \chemfig[][scale=0.7]{*3(---)}
    \quad\quad\quad\quad
    \chemfig[][scale=0.7]{*4(----)}
    \quad\quad\quad\quad
    \chemfig[][scale=0.7]{*5(-----)}
    \quad\quad\quad\quad
    \chemfig[][scale=0.7]{*6(------)}
    \chemnameinit{}
\end{figure}

\noindent Anéis que contêm de 5 a 7 carbonos são com frequência chamados de anéis comuns, enquanto que os que contêm 3 ou 4 carbonos são chamados anéis pequenos, os que contêm de 8 a 11, anéis médios, e os que contêm mais de 12 átomos, anéis grandes ou macro-anéis. 

Os anéis comuns e grandes são semelhantes aos alcanos acíclicos na maior parte de suas propriedades físicas e química, enquanto que os anéis pequenos e médios têm um comportamento algo diferente. As características pouco usuais dos anéis pequenos aparecem porque o carbono sp3 nos alcanos abertos têm ângulos de ligação com valor próximo aos ângulos internos do tetraedro regular (109.5$\degree$) enquanto que as exigências geométricas dos anéis pequenos reduzem estes ângulos para valores muito menores.

Dois orbitais p no mesmo carbono têm um angulo de 90$\degree$ entre si. Se estes orbitais são hibridados pela adição de caráter s, o angulo entre eles aumenta ate atingir um valor de 180$\degree$ quando a percentagem de s chega a 50\%. Não existe combinação de orbitais possível que produza ângulos interorbitais menores do que 90$\degree$. Isto significa que no ciclo-propano os orbitais não se encontram segundo a mesma linha reta, e as ligações carbono-carbono são mai bem descritas como `curvas`. Tais ligações não são tão fortes como as ligações comuns, e as moléculas que as contêm possuem mais energia do que o normal. Moléculas deste tipo são ditas sob tensão. 

No caso do ciclo-butano, os ângulos interorbitais deveriam ser de 90$\degree$ e as ligações teriam de ser do tipo p e, portanto, muito fracas. A adição de algum caráter s torná-las-ia ciclo-propano têm, assim, ligações curvas estão sob tensão (Figura 3.10). Ciclo-propano possui tensão angular (porque os ângulos de ligação estão deformados) e também tensão tercional por cause dos hidrogênios em coincidência.

\noindent A \emph{expressão} tensão estérica (do Grego stereos, espaço) é, algumas vezes, utilizada para indicar a tensão imposta a uma molécula por sua geometria tridimensional (espacial).

Uma reação química que leve ao rompimento de um sistema sob tensão causa-lhe um alivio de tensão. Tal reação tende a ser mais exotérmica e mais rápida do que o mesmo tipo de reação envolvendo um anel de ciclo-hexano ou um sistema acíclico. 

A tensão em moléculas pode ser facilmente detectada a partir do calor de combustão, o qual é obtido pela queima do composto e a medida do calor evolvido. Para um alcano a reação é:

\begin{figure}[H]
    \centering
    \ch{C_{$n$}H_{2$n$+2} + (3 $n$ + 1/2 ) O_2 -> $n$ CO2 + (n + 1) H2O + calor}
\end{figure}

A ligação entre quaisquer dos átomos, por exemplo, uma ligação C-H, tem uma energia tal que, em primeira aproximação, é independente da natureza do resto da molécula. O calor de combustão do n-pentano é de 845,2 kcal mol$^{-1}$ e o do n-hexano 1.002,6 kcal mol$^{-1}$. O calor de combustão de um único grupo metileno (-CH$_{2}$-) é, então, a diferença entre esses dois valores, ou seja, 157,4 kcal mol${^-1}$. Ciclo-hexano não está sob tensão e seu calor de combustão deveria ser $6\times157,4 = 944,4$ kcal mol$^{-1}$, que é exatamente o encontrado (Quadro 3.4). Ciclo-propano, por outro lado, tem um calor de combustão 499,8 kcal mol${^-1}$, isto é, 27.6 kcal mol$^{-1}$ acima do valor esperado para uma estrutura sem tensão ($3\times157,4=472,2)$ kcal mol$^{-1}$). A energia de tensão para ciclo-alcanos ate o ciclo-decano foi determinada a partir dos calores de combustão respectivos e os valores encontrados estão no Quadro 3.4.

As conformações e as energias de tensão destes anéis serão examinadas rapidamente a seguir. Os átomos de carbono do ciclo-propano estão necessariamente em um plano (porque quaisquer três pontos determinam um plano). Esta molécula tem um alto grau de simetria e todos os seus hidrogênios são equivalentes.

A experiencia mostra que o ciclo-butano não é planar mas tem a forma de um cartão dobrado na diagonal.

\begin{figure}[H]
    \centering
    \chemname{\chemfig[][scale=0.7]{?-[:30,1]-[:330,1]-[:115,2]?}}{Ciclo-butano}
\end{figure}

A dobra reduz o angulo CCC do valor de 90$\degree$ que teria, se fosse o ciclo-butano planar, 88$\degree$ aproximadamente. Esta deformação do angulo, embora o afaste ainda mais do valor ideal de 109,5$\degree$, aumentando portanto sob este aspecto a energia do sistema, permite uma redução substancial da energia de torção, dando como resultado uma estabilização da molécula `dobrada` em relação à molécula planar. Como pode ser constatado na figura, os hidrogênios não mais estão em coincidência na forma dobrada. A energia total é um mínimo e a molécula tende naturalmente a adotar a conformação de energia miníma. Observe que embora a tensão causada pela curvatura da ligação seja menor do que no case do ciclo-propano, como o numero de ligações curvas é maior, a energia de tensão de ambas as moléculas é semelhante (Quadro 3.4).

O ciclo-pentano é dobrado pela mesma rezão e tem a conformação acima. No ciclo-pentano um ou dos átomos de carbono estão fora do plano dos demais. Como a estrutura é flexível, os que estão fora do plano estão mudando sucessivamente, ou seja, cada átomo forca sua entrada no plano, deslocando, assim, o átomo seguinte, de modo que a não-planaridade' movimenta-se no anel. Tal movimento é chamado uma 'pseudo-rotação', e é este mesmo movimento que possui a forma flexível do ciclo-hexano.

Os ângulos CCC no ciclo-pentano são, em media, de 105$\degree$, de modo que a tensão de curvatura das liga coes é muito menos severa do que no case do ciclo-butano enquanto que a tensão devida à coincidência dos hidrogênios é da mesma ordem de grandeza. Em consequência a tensão total é menor do que no caso do ciclo-butano.

\begin{figure}[H]
    \centering
    \setchemfig{cram width=2pt}
    \chemname{\chemfig[][scale=0.7]{?<[:0,1]-[:45,2,,,line width=2pt]>[:315,1]-[:225,1]-[:135,1.5]-[:180]?}}{Ciclo-heptano}
    \qquad
    \chemname{\chemfig[][scale=0.7]{?-[:30,1]-[:300,1]-[0,1]-[2,1.5]-[5,1]-[4,1]-[3,1]?}}{Ciclo-octano}
\end{figure}

\noindent\emph{A energia introduzida pela tensão no ciclo-heptano é devida principalmente aos ângulos torcionais desfavoráveis. Não é possível obter-se um arranjo em oposição perfeita em cada ligação C-C como no caso do ciclo-hexano. Em vez de ângulos diedro de 60$\degree$ tem-se cerca de 75$\degree$ o que, como mostra a Figura 3.6, leva a uma tensão torcional.}

\emph{Ciclo-octano e ciclo-nonano também contém considerável tensão torsional. Além disso nessas moléculas exitem repulsões hidrogênio-hidrogênio através do anel (o mesmo tipo de repulsão que existe entre os dois hidrogênios da proa e da popa da conformação bote do ciclo-hexano), isto é, repulsões de van der Waals entre hidrogênios muito próximos.}

O ciclo-decano tem uma conformação estranha na forma cristalina, como mostram estudos de difração de raios X. Se a molécula tivesse a conformação em 'coroa' haveria excessiva tensão torsional e de van der Waals. Na conformação adotada a molécula pode aliviar uma parte desta tensão, embora com isto aumente a tensão angular.

Os ângulos CCC no ciclo-decano são maiores do que os ângulos normais do \ch{C_{sp^3}}, sendo em media de 117$\degree$. Estes ângulos permitem à molécula reduzir as repulsões H-H no interior do anel. A diferença de energia entre a conformação da fase cristalina e a conformação em coroa é pequena e acredita-se que, em soluções, esta ultima seja a conformação predominante. Forcas de empacotamento no cristal, por outro lado, parecem favorecer a outra conformação.

Em cada um dos anéis de tamanho médio existe um compromisso entre as tensões torsional, angular e van der Waals, e cada uma dessas moléculas adota a conformação de tensão total miníma, isto é, possui a menor energia total. Por causa da tensão, todavia, anéis de tamanho médio são difíceis de sintetizar e são relativamente raros (ver a Seção 33.3).

Cada anel médio tem alguns hidrogênios que são aproximadamente equatoriais e outros que são aproximadamente axiais. Os problemas conformacionais são muito maiores do que no caso do ciclo-hexano e, por isso mesmo, menos compreendidos.

Anéis ainda maiores têm, usualmente, uma conformação preferida que consiste em duas cadeias longas paralelas, semelhantes a dois alcanos normais colocados um ao lado do outro em conformação totalmente anti, ligadas nas extremidades. São essencialmente sem tensão. 

Existem moléculas que contêm dois ou mais de dois anéis. Ciclo-pentil-ciclo-hexano é um exemplo deste tipo de moléculas.

Existem sistemas nos quais um ou mais carbonos são comuns a dois anéis. Assim, os sistemas espirano têm apenas um átomo de carbono comum dos dois anéis:

Nos sistemas biciclo- dois ou mais carbonos são comuns aos dois anéis:

O sistema decalina, sendo composto de dois anéis de seis membros, é muito estável e é encontrado facilmente na natureza. Sistemas mais complexos, tais como per-hidro-fenantreno e per-hidro-antraceno, são, também, muito comuns.

O adamantano contém um arranjo tridimensional de anéis de ciclo-hexano e constitui-se na unidade fundamental da estrutura do diamante. Mostramos uma parte da estrutura do diamante para comparação. A resistência e dureza do diamante vem do fato que o cristal é realmente uma molécula gigante onde os átomos são ligados por covalência.
