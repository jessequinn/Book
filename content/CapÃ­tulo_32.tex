RMN, espectrometria de massas e identificacao de compostos origanicos

introducao

O desenvolvimentos de metodods instrumentais de analise tem causado grande empacoto na quimica de nosso tempo. Muitos especilista usam hoje equipamentos electronicos complexos que eram desconhecidos ou desenvolvidos de rudementar antes da segunda garra. Pesagens, titulacoes, PH, pontos de fusao  e concentracoes sao exemplos de dados que podem ser determinados atualmente, de modo automatico por equipamento relativamente barato. Por outro lado, tecnicas como espectroscopia no ultravioleta, cristalografia de raios x, espectrometria de masas, espectrometria no infravermelho, espectroscopia de ressonaniccia de nuclear e cromatogria com fases gasoso, que nao eram conhecidas o estavam no inicial de seu desenvolvimento, sao metodos de uso corrente e indispensavel os quimicos organicos e analistas. Os dados geralados por estes metodos sao, ainda, objeto de analise detalhes pelos fisicos quimicos e quimicos teoricos. 

Qualquer quimico um dia o outro tera de infretar o problema identificar uma substancia de conhecida. O composto pode ser um produto simples, por em inspirado, de uma reacao quimico ou um substancia complexo obtida da natureza, cuja estrutura nunca foi determinada antes. Em ambos os casos, a primeria reacao do quimico organico moderno sera apelar para os metodos instrumentais. Somente apos ter obtido o maximo de informacoess ossivel instrumentaca disponivel e que ele tentara aplicar os metodos quimicos para completar a analise e comfirmar o nao a estutura pos do lado. 

A identificacao de substancias de conhecidas foi descrta em termos gerailze na 5.1 e ou lator deveria reve-la. Dos metodos disponiveis, a ressonanci magnetic nuclear (RMN, apresentada no Capilto 5, e ou mais util. A espectroscopia no infravermelho e no ultravioleta sao comuns e tambem muito uteis. 

Neste capitulo vamos discutir ressonancia magnetic nuclear mais detalhiamente e ver estrir mais informacoes dos espectros. Apresentamos entao a espectrometria de massas, explorando alguns dos espectros de interesse pratico que o metodo oferece. A seguir, vemos, de um modo geral, como resever o problema da identificacao de estruturas usando os metodos espectroscopias e, finalmente, aplicarmos simultanemente os metodos fisicos e quimicos ou problem estrutural. O metodo geral eseninente um mapa seguir todoso tipo informcoes disponivel para chegar a um objetivo final: estruture de compodoe em estudo.

E preciso mencionar estruduar exats muitos moleculas complicados tem sido determindas com presencsa atraves dos metodos de raiso x. Essencialmente, estes metodos permitem ou quicica retratar a molecula, localizando os atomos de manrier semelhante ao que fazemos para localizar objetos usando a luz visivel na fotograrfia conventlam. Uma existencia para que metodo de raois x seja possivel e que as substancia exista sob a forma cristlina. Para que se possa desvenolver uma restustura tridimensional, muitos fotografias deve ser obtidas diferentes angulos e, a seguir, correlacionadas por calculus trabalhosos. Um metodo foi desvenvolvido principalmente a partir de 1.930. Em um exemplo de determinicacao dea epica, um amino-acucar isolado da carapaca de carangueijos, a dois-amina-glicose, exigiu 3.000 fotografias raios x, tiradas manualmente, de tres anos de calculus complicados e antegiantes. Ja em um exemplo resante, a estrutura do metiodeto  




 ,